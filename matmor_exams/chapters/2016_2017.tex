\subsection{MFE, mock, 2016-10-28}

\textbf{SECTION A}

\begin{enumerate}

\item Consider the function $f(x,y,z)=x^3+2xz-3z^3-y^2$. Using the total differential find the approximate value of $f(1.01,0.99,1.02)$.

\item Consider the system
\[
\begin{cases}
x^3 + y^3 + z^3 = 3 \\
x + x^3 + 2y + 3y^2 + xyz + z^3 = 9 \\
\end{cases}
\]
\begin{enumerate}
  \item Check whether the functions $y(z)$ and $x(z)$ are defined at a point $(1, 1, 1)$
  \item Find $y'(z)$ and $x'(z)$
\end{enumerate}

\item Consider the function $f(u)=g(x,y)$ where $x=u^2$ and $y=\cos u$. The function $g$ has continuous second derivatives everywhere. Find $f'(u)$ and $f''(u)$.

\item The curve on the plane is defined by the equation $x^4 + y^2 + y^4 = 3$.
\begin{enumerate}
  \item Find a vector that is orthogonal to the curve at the point $(1, 1)$
  \item Find a vector that is parallel to the curve at the point $(1, 1)$
\end{enumerate}

\item The function $g$ is monotonic. The function $f$ is the inverse of the function $g$. Find $f'(1)$ if it is known that $g(10)=1$, $g'(10)=5$, $g(1)=-2$, $g'(1)=4$.

\item Let $x$ be a vector, $x\in \mathbb{R}^n$, and $A$ be $n\times n$ matrix of constants. Consider the function $f(x) = x^T A x$.
\begin{enumerate}
 \item Clearly state the Young's theorem
 \item Express the Hesse matrix of $f$ using $A$ and $A^T$.
\end{enumerate}





\end{enumerate}

\textbf{SECTION B}

\begin{enumerate}[resume]


\item Consider the function  $f(x, y) = \sqrt[3]{x^3 + y^3}$.

\begin{enumerate}

\item (7 points) Find $\frac{\partial f}{\partial x}(0,0)$ and $\frac{\partial f}{\partial y}(0,0)$. Is the function $f(x, y)$ continuously differentiable everywhere?
\item (3 points) Find equation of the tangent plane to the graph of $z= f(x,y)$ at the origin.

\item (10 points) Let $\Delta f = f(x,y) - f(0,0)$. Compare $\Delta f$ with the $df$ (total differential) at
the origin. Base your comparison on the existence of the limit $\lim_{x\to0, y\to 0} \frac{\Delta f - df}{\sqrt{x^2 + y^2}}$ as $x\to 0$ and $y\to 0$.
\end{enumerate}

\item Cournot duopoly produces good $Y$, where $Y=y_1 + y_2$. Here $y_1$ is the output of the
first firm and $y_2$ is the output of the second firm. The inverse demand on
good is given by the formula $p(Y)=1/Y$, where is $p(Y)$ the price per unit. The total
costs of the firms are $TC_1(y_1) = 2y_1$ and $TC_2(y_2) = y_2$, respectively. Let the profit of the first
firm be $\pi_1 = p(y_1 + y_2)y_1 - 2y_1$ and the profit of the second firm be $\pi_2 = p(y_1 + y_2)y_2 - y_2$.

\begin{enumerate}
\item (8 points) Write down the system of the first-order
conditions $\begin{cases} \frac{\partial \pi_1}{\partial y_1} =0 \\  \frac{\partial \pi_2}{\partial y_2} =0 \end{cases}$ and solve it.
\item (12 points) Government decides to impose a per unit tax $t$ on
both firms. It will increase costs for them by $ty_1$ and $ty_2$ respectively. Rewrite the system
of first-order conditions accounting for the tax. Find $\frac{dy_1}{dt}$ and $\frac{dy_2}{dt}$ by referring to the
appropriate IFT. Check that IFT conditions are verifiable here.
\end{enumerate}


\end{enumerate}



\subsection{MFE, fall exam, 2016-12-27}


\textbf{SECTION A}

\begin{enumerate}


\item Find the second order Taylor expansion of the function $f(x, y) = \sin( e^{2x} - e^{3y})$ at a point $x = 0$, $y=0$.


\item Find the limit
\[
\lim_{x \to 0, y \to 0} \frac{x^2 y^2}{x^2 + 3y^8}
\]


\item The function $f$ is defined by $f(x, y) = x^3 + 5xy^2$. Consider the graph $G$ of the function $f$
\begin{enumerate}
  \item Find a vector that is orthogonal to the surface of $G$ at the $x=1$, $y=1$.
  \item Find a vector that is parallel to the surface of $G$ at the $x=1$, $y=1$.
\end{enumerate}

\item Using Lagrange multiplier method find and classify the constrained extrema of $f(x, y, z) =  2x +3y + 9z$ subject to $x^2 + y^2 + z^2 = 1$.


\item Consider the sets $B_n = \left(-1/n, (n+1)/n \right)$, and the set $A = \cap_{n=1}^{\infty} B_n$.
\begin{enumerate}
\item Is the set $A$ bounded? Open? Closed? Compact? Convex?
\item Sketch the set $A \times A$.
\end{enumerate}

\item Find the local maxima of the function $f(x, y) =  (12 - x) x \sin y + x^2 \sin y \cos y$. 
Check whether these local maxima are the global ones.
% \item Find and classify the local extrema of $f(x,y) = 4 + x^3 + y^3 - 3xy$


\end{enumerate}

\textbf{SECTION B}

\begin{enumerate}[resume]


\item Let $u(c_t)$ be utility function of consumption $c_t$ at time $t$ which is discrete, $t \in \mathbb{N}$, ($\mathbb{N}$ --- set of natural numbers).
Function $u$ is continuously differentiable and strictly concave for $c>0$, $u(0)=0$, $u'(c)>0$, $\lim_{c\to 0+ } u'(c) = +\infty$.
\begin{enumerate}
\item (5 points) Consider maximization problem: $\sum_{t=1}^T u(c_t) \to \max$ subject to $\sum_{t=1}^T c_t = s$, $c_t\geq 0$, where the parameter $s$ is positive.
Let $T=2$. Show that if $(c_1^*, c_2^*)$ is the optimal bundle then $c_1^* = c_2^*$.
\item (7 points) Generalize this result for any natural $T$. You may refer to the Lagrange method.
\item (8 points) Let $(c_1^*, c_2^*, c_3^*, \ldots, c_T^*)$  be the optimal bundle. Find the limit of $\sum_{t=1}^T u(c^*_t)$  as $T \to \infty$ or show that it does not exist.
\end{enumerate}

\item In the method of least squares the straight line $a + bx$ is fit to the data $\{ (x_i, y_i), \; i \in 1, 2, \ldots, n \}$,
by minimizing the sum $S = \sum_{i=1}^n (y_i - (a + b x_i))^2$ with respect to $a$ and $b$.
\begin{enumerate}
\item (15 points) Using first-order conditions find optimal $a$ and $b$. Under what conditions does the solution for $a$ and $b$ exist?

Hint: you may find Cauchy-Schwartz inequality useful here.
\item (5 points) Show that the sufficient conditions are met.
\end{enumerate}

\end{enumerate}


\subsection{MFE, fall exam, 2016-12-27, solutions}

\begin{enumerate}
  \item 
  \item Let's bound the sequence from both sides:
  \[
    0 \leq \frac{x^2 y^2}{x^2 + 3y^8} \leq \frac{x^2 y^2}{x^2} = y^2
  \]
  So the limit is equal to $0$.
  \item 
  \item 
  \item Just look at first $B_1$, $B_2$ and $B_3$! $B_1 = (-1/1 ; 2/1)$, $B_2 = (-1/2; 3/2)$, $B_3 = (-1/3; 4;3)$.
  We see that only points from $[0;1]$ belongs to every $B_i$, so $A = \cap_i B_i = [0;1]$.
  The set $A$ is bounded, closed, compact, convex. The set $A \times A$ is a square on a plane. 
\end{enumerate}



\subsection{MFE, fall retake, 2017-01-20}


\textbf{SECTION A}

\begin{enumerate}


\item Find the second order Taylor expansion of the function $f(x, y) = \cos( e^{2x} - 1) - \cos(e^{3y}-1)$ at a point $x = 0$, $y=0$.


\item Find the limit or prove that it does not exist
\[
\lim_{x \to 0, y \to 0} \frac{x^2 y^2}{x^4 + 3y^4}
\]


\item Consider the sphere given by $x^2 + y^2 + z^2 = 1$. Find the equation of the tangent plane to the sphere at the point $x=1/\sqrt{3}$, $y=1/\sqrt{3}$, $z=-1/\sqrt{3}$.

\item Using Lagrange multiplier method find and classify the constrained extrema of $f(x, y, z) =  2x +3y + 9z$ subject to $x^2 + y^2 + 4z^2 = 1$.


\item Consider the set $A$ on the plane $(x, y)$ given by the inequality
\[
\frac{(x^2 + y^2 - 3)(x^2 + y^2 -10)}{x^2 + y^2 - 10} \geq 0
\]
\begin{enumerate}
  \item Is the set $A$ closed? open? bounded? convex? compact?
  \item If possible represent the set $A$ in the form $A=B_1 \times B_2$ where each set $B_i \subset \mathbb{R}$.
\end{enumerate}


\item Find and classify the critical poinst of the function $f(x, y) =  \exp(-x^2 -6y^2 + 2xy + 2y)$. Check whether these local extrema are the global ones.
% \item Find and classify the local extrema of $f(x,y) = 4 + x^3 + y^3 - 3xy$





\end{enumerate}

\textbf{SECTION B}

\begin{enumerate}[resume]
\item Short-run total costs of a firm are given by
\[
STC(q,K) = {q^2} + 3qK + 4{K^2} - K + \frac{1}{{16}},
\]
where  $q$ is the output and $K$ is the amount of capital fixed in the short-run. In the long-run the firm can always adjust the capital in order to minimize costs. Use the appropriate envelope theorem to find $MC = (TC)'$ — long-run marginal costs.

\item Solve the constrained minimization problem in two variables: $x^2 + y^2 \to \min$ subject to constraint $(x - 1)^3 = y^2$. Check firstly whether the method of Lagrange multipliers is valid to apply.


\end{enumerate}



\subsection{28 March 2017}

\textbf{SECTION A}

\begin{enumerate}

\item Find indefinite integrals
\begin{enumerate}
  \item  $\int {{e^{3x}}\sin 2x \, dx}$;
  \item  $\int {\frac{{x + 3}}{{{x^2} - 6x + 9}} \, dx}$.
\end{enumerate}

\item Solve the differential equation
\[
y''' - y'' + 6y' - 6 = 42\sin (x\sqrt 6 ).
\]

\item Solve the difference equation
\[
{y_{t + 2}} - 6{y_{t + 1}} + 9{y_t} = 5t.
\]

\item The function $f(x, y)$ is non-constant and homogeneous. It is also known that $h(x, y) = f'_x(x, y) + 3x^2y$ is homogeneous of degree 3. Find the value of $\frac{xf'_x(x, y) + yf'_y(x, y)}{f(x, y)}$.

\item Solve the following linear programming problem:
\[
\begin{cases}
2x_1 + 2x_2 + 3x_3 \to \min \\
x_1 \geq 0, x_2 \geq 0, x_3 \geq 0 \\
3x_1 + 5x_2 + x_3 \geq 8 \\
5x_1 + 3x_2 + x_3 \geq 9 \\
\end{cases}.
\]

\item Maximize the function
\[
11 + 10x_1 - x_1^2 -3x_2 + 8x_3 - x_3^2
\]
subject to constraints $2x_1 -x_2+4x_3 \leq 10$ and $x_2 \leq 100$.

\end{enumerate}

\textbf{SECTION B}

\begin{enumerate}[resume]


\item Consider the second-order differential equation
\[
xy'' - y' - 4{x^3}y = 0.
\]
It can be solved by an appropriate change of the variable $t = \phi (x)$.
\begin{enumerate}
  \item (10 points) Find this function $\phi (x)$ by setting the task of cancellation the term with $\frac{{dy}}{{dt}}$.
  \item (5 points) After the substitution the transformed equation has constant coefficients. Find its general solution.
  \item (5 points) Solve the original equation.
\end{enumerate}

\item Three players play the following game. Simulteneously each of them chooses one possible bet: either 1\$ or 2\$. A player is declared winner if his bet is unique and wins the amount of his bet. For example, if players have chosen 1, 2 and 1 then their corresponding payoffs are 0, 2 and 0.

\begin{enumerate}
  \item Find all Nash equilibria in pure strategies.
  \item Find symmetric Nash equilibrium in mixed strategies.
\end{enumerate}

\end{enumerate}


\subsection{marking 28 March 2017}

\begin{enumerate}
\item
1a = 5 points: 2 points - correct 1st integration by parts, 2 points - correct 2nd integration by parts, 1 point - calculations of the integral

2a = 5 points: 1 point - correct choice of solving method (knowledge that it's somehow connected to partial fractions), 3 poiints - correct decomposition to the sum of partial fractions,  1 point - calculations of the integral

\item

4 points - correct complementary function (1 point - characteristic equation, 2 points - roots, 1 point - answer)
6 points - correct particular integral (2 points - correct form, 2 points - all necessary derivatives, 2 points - calculations)

\item

4 points - correct complementary function (2 points - equation and roots, 2 points - form and answer)
6 points - correct particular solution (3 points - correct form, 3 points - calculations)

\item

5 points - correct degree of homogenuity of $f(x,y)$
1 point - understanding that the answer should be constant
1 point - recalling something about Euler's theorem
3 points - right answer

\item

2 points - Dual problem
2 points - graph
3 points - minimum
3 points - minimizer

\item

1 point - NDCQ
2 points - Lagrangean
3 points - FOC
4 points - solution of FOC

\item

7a = 10 points:

5 points - correct calculations of $dy/dt$ and $d^2y/dt^2$ for some function $t=\phi(x)$ or in general form using chain rule (2 points for the 1st derivative, 3 points - for the 2nd derivative)
5 points - correct guess that the substitution is $t= x^2$ and trying to do something with it (just correct guess without calculations - 2 points)

7b = 5 points:

1 point - characteristic equation
2 points - roots
2 points - solution

7c = 5 points

\item

8a = 6 points


Every player chooses 1 is not an equlibrium (every player wants to deviate)
Every player chooses 2 is not an equlibrium (every player wants to deviate)
Two players choose 1 and one player chooses 2 is an equilbrium.
Two players choose 2 and one player chooses 1 is an equilbrium.

8b = 14 points

Step 1. Idea that in symmetric equilibrium every player chooses 1 with some probability p and 2 with probability (1-p).

Step 2. First player should be indifferent between his pure strategies.

payoff if I choose 1 = payoff if I choose 2:
$(1-p)^2 = 2 p^2$ (equation = 8 points)

Step 3.  two roots $p = -1 + \sqrt{2}$ и $p= - 1 - \sqrt{2}$.
one root is impossible, so $p = \sqrt{2} - 1$.

Step 1 = 3 pts, step 2 = 6 pts, step 3 = 5 pts.
\end{enumerate}
