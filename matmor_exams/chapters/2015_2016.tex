\subsection{Mock 28.10.2015}


\textbf{Variant 1}. Please, don't forget to write you variant number. Sections A and B will make up 60\% and 40\% of the exam grade, respectively. Total duration of the exam is 120 min. Good luck! :)



\textbf{SECTION A}

\begin{enumerate}

\item Consider the function $f(x,y,z)=x^5+2xyz-3z^3$. Using the total differential find the approximate value of $f(1.01,0.99,1.01)$.

\item Consider the function $f(x,y)=2x^4-(x+y)^3$.
\begin{enumerate}
\item Find the Hesse matrix. Clearly state the Young theorem even if you don't use it.
\item Find the definiteness (positive definite, positive semidefinite, etc) of the Hesse matrix at the point $(1,2)$.
\end{enumerate}

\item Let the function $f(x,y)$ be defined by the formula
\[
f(x,y)=\begin{cases}
-1, \, \text{if} \, x>y \\
1, \, \text{if} \, x\leq y
\end{cases}
\]

\begin{enumerate}
\item Find the limits $\lim_{x\to\infty}\lim_{y\to \infty} f(x,y)$ and $\lim_{y\to\infty}\lim_{x\to \infty} f(x,y)$
\item Does the limit $\lim_{x\to\infty, \, y\to \infty} f(x,y)$ exist?
\end{enumerate}


\item The functions $f$ and $g$ are given: $f(x,y)=x^2+2xy+y^4$, $g(x,y)=-5x^2-xy-2y^4$. Find at least one direction from the point $(1,1)$ in which both functions will grow.

\item The function $z$ is defined by the formula $z(x,y)=f(x^3-y^2)$. Simplify the expression $2y\frac{\partial z}{\partial x}+3x^2\frac{\partial z}{\partial y}$.

\item Consider the function $f(x,y)=\sqrt{x+\sqrt{y+\sqrt{x+\sqrt{y + \ldots }}}}$.
\begin{enumerate}
\item Find the value of $(f^2(x,y)-x)^2-y-f(x,y)$
\item Find $\partial f/\partial x$ and $\partial f/\partial y$ at the point $(1,1)$
\end{enumerate}




%\item Given the system
%\begin{equation} \nonumber
%\begin{cases}
%u^2-w^2+x^2+y^2=2 \\
%uw+xy=2
%\end{cases}
%\end{equation}
%\begin{enumerate}
%\item Define a sufficient condition for functions $u(x,y)$ and $w(x,y)$ to be differentiable
%\item Find $\frac{\partial u}{\partial x}$
%\end{enumerate}

\end{enumerate}

\textbf{SECTION B}

\begin{enumerate}[resume]


\item (20 points) Let $S_1$ and $S_2$ be two sets from $\RR^2$: $S_1=\{(x,y) \in \RR^2 | xy=1 \}$, $S_2=\{(x,y) \in \RR^2 | xy=-1 \}$ and $S=S_1+S_2$. We denote by the sum $S_1+S_2$ the set
\[
S_1+S_2=\{ (x,y) \in \RR^2 | (x,y)=(x_1,y_1)+(x_2,y_2), \, (x_1,y_1) \in S_1, \, (x_2,y_2) \in S_2 \}
\]
\begin{enumerate}
\item Are the sets $S_1$ and $S_2$ closed? Justify your answer.
\item Does the origin belongs to the set $S$? % Show that the origin does not belong to $S$.
\item Is the set $S$ closed? % Show that $S$ is not closed.
\end{enumerate}
\item (20 points) Consider a Cournot duopoly of the two identical firms that compete by choosing outputs $y_1$  and $y_2$  simultaneously. Marginal costs of these firms are constant $MC_1=MC_2=c>0$. When the outputs $y_1$  and $y_2$ are set, the price of a good can be found by the formula  $p=a-b(y_1+y_2)$, where  $a>c$, $b>0$.
\begin{enumerate}
\item Find equations of the level curves for the profits of the firms $\pi_1(y_1, y_2)$ and  $\pi_2(y_1, y_2)$.
\item It is known that the point of equilibrium outputs $(y_1^*, y_2^*)$  in the coordinate plane $(y_1, y_2)$ can be found by drawing tangent lines to the level curves and these tangents should be parallel to the axes.
Then $(y_1^*, y_2^*)$ is the point of intersection of the tangents.
By finding corresponding gradients of $\pi_1(y_1, y_2)$,  $\pi_2(y_1, y_2)$ and using the hint stated above, find    $(y_1^*, y_2^*)$ in terms of $a$, $b$ and $c$.
\end{enumerate}

\end{enumerate}


\subsection{Mock 28.10.2015. Marking scheme}

\begin{enumerate}
\item Three partial derivatives = 1 pt for formula, 1 pt for value (6 pts). Value of function in the initial point = 1 pt, formula of differential = 1 pt, answer = 1 pt
\item First derivatives =  2 pts (1 pt for each), Hesse matrix = 3 pts, statement of the Young theorem = 2 pts, definiteness = 3 pts
\item 3 pts + 3 pts + 4 pts
\item Two gradients = 4 pts (2 pts each).

Directional derivative approach: two directional derivatives = 2 pts, condition = 2 pts, correct direction = 2 pts.

Visual approach on plane: observation that both gradients are above $y=x$ = 2 pts, observations that $\cos \alpha >0 $ if $\alpha < \pi/2$ = 2 pts, correct direction = 2 pts

\item Two derivatives = 8 pts (4 pts each). Final answer = 2 pts
\item Evaluation of expression = 2 pts, two derivatives = 8 pts (4 pts each)
\item a --- 8 pts, b --- 4 pts, c --- 8 pts
\item a --- 6 pts (3 pts + 3 pts), gradients of profit functions = 6 pts (3 pts each), remaining part of the proof = 8 pts
\end{enumerate}


\subsection{Fall Exam, 27.12.2015}

\textbf{Variant 1. Section A}

\begin{enumerate}

\item At the beginning James Bond is located at the point $(1, 1)$. To choose his new location he calculates the gradient of the function $f(x,y)=x^2+y^2 - 3xy+x$ from his current location and moves in the direction given by the gradient by its length. Where he will be after two movements?

\item Consider the system of equations
\[
\begin{cases}
x^4 + y^4 + z^4 = 3 \\
x + x^3 + y + 2y^3 + z + 3z^3 = 5
\end{cases}
\]

\begin{enumerate}
\item Are the function $x(z)$ and $y(z)$ defined around the point $A=(-1,1,1)$?
\item Find $dx/dz$ and $dy/dz$
\end{enumerate}


\item Find and classify unconstrained extrema of the function $f(x,y)=x^4 + y^8 - 2xy$

\item Find and classify constrained extrema of the function $f(x, y) =  xy$ subject to $x^2 + 4y^2= 9$


\item The function $u$ is defined by the equation $u^3(t) + u(t) = f(x,y)$, where $x=2-t$ and $y=1+2t$ and $f$ is in $C^2$. Find $du/dt$ and $d^2 u/dt^2$


\item Consider the function $f(x)=h(x)-ax$, where the function $h$ is twice differentiable and $h''(x)<0$ for all $x$. The global maximum of $f$ is denoted by $x^*(a)$.
\begin{enumerate}
\item Find $dx^*/da$
\item It is known that for $a=1$ the optimal point is $x^*=3$ and the value of maximum is $2015$. What is the approximate value of maximum for $a=1.01$?
\end{enumerate}

\end{enumerate}



\textbf{Variant 1. Section B.} Problems 7 and 8 can be solved separately.

\begin{enumerate}[resume]
\item A risk-averse Alex possesses $w$ dollars of wealth in money and property.  His house worth $L < w$ dollars can be completely destroyed by a landslide with the probability $p$, $0 < p < 1$. Let’s denote his wealth if landslide occurs by $x_L$ and $x_{NL}$ otherwise. Then his expected utility can be calculated by $E(u) = p \ln x_L + (1-p) \ln x_{NL}$, where $x_L, x_{NL} > 0$.
\begin{enumerate}
\item Show that $E(u)$ is a concave function in its domain.
\item Show that the set in  $(x_L, x_{NL})$ plane defined by the inequality $E(u)\geq const$  is convex.
\end{enumerate}

\item In order to reduce risk Alex buys insurance from a perfectly competitive company. By doing that he maximizes his expected utility $E(u)$ with respect to $(x_L, x_{NL})$ subject to constraint imposed by the company $p(w-L-x_L)+(1-p)(w-x_{NL})=0$.
\begin{enumerate}
\item  Find his optimal bundle  $(x_L^*, x_{NL}^*)$. Use bordered Hessian to check sufficiency. Is Alex better-off with the insurance? Explain.
\item Let $E(u)^*$ be the maximum value of $E(u)$ with insurance. By applying Envelope Theorem find $\partial E(u)^* / \partial p$. Express your answer in terms of $p$, $w$ and $L$ alone.
\end{enumerate}


\end{enumerate}

\subsection{Solution to Fall Exam, 27.12.2015}

\begin{enumerate}

\item Marking
\begin{itemize}
\item first movement 6 pt: 1 pt for the formula of gradient, 2 pt for the calculation of derivatives, 1 pt for the substitution of initial point, 2 pt for the location after first movement

\item second movement 4 pt: 2 pt for the calculation of gradient at the new point, 2 pt for the final location.

\item common mistake: movement by 2*(first gradient) --- 6 pt.
\end{itemize}
\textbf{Solution:}

Calculate the gradient of the function
\[
\nabla f(x,y):=(\frac{\partial f}{\partial x}, \frac{\partial f}{\partial y})=(2x-3y+1, 2y -3x).
\]
Then substitute the initial point (1, 1) to the formula above: $\nabla f(1,1)=(0,-1).$ This is the direction of James's first movement. He moves exactly by the length of this vector (problem conditions) so we don't need to normalize obtained result. Thus Bond's location after the first movement will be $(1,1)+(0,-1)=(1,0).$

Now calculate the gradient at the new point:$\nabla f(1,0)=(3,-3).$ This is the direction of the second movement. The final location will be $(1,0)+(3,-3)=(4,-3).$ This is the final answer.


\item Marking
\begin{itemize}
\item 2pt for IFT checking
\item each derivative 4 pt: 3 pt for the formula, 1 pt for the calculations
\end{itemize}
\textbf{Solution:}

Let $f_1 = x^4+y^4+z^4-3, f_2=x+x^3+y+2y^3+z+3z^3-5.$
\begin{enumerate}
\item Check the conditions of the IFT:
\begin{enumerate}
\item $f_1(-1,1,1) =0$, $f_2(-1,1,1)=0,$
\item $f_1, f_2 \in C^1,$
\item \begin{equation*}
\left|\frac {\partial(f_1,f_2)}{\partial(x,y)}\right|=\left|
\begin{array}{cc}
4x^3 & 4y^3 \\
1+3x^2 & 1+6y^2
\end{array}
\right|=\left|
\begin{array}{cc}
-4 & 4 \\
4 & 7
\end{array}
\right| = -44 \neq 0.
\end{equation*}
\end{enumerate}
\item
\begin{equation*}
\frac{d x}{d z} = -\frac{\left|\frac {\partial(f_1,f_2)}{\partial(z,y)}\right|}{\left|\frac {\partial(f_1,f_2)}{\partial(x,y)}\right|} = -\frac{
\left|
\begin{array}{cc}
4z^3 & 4y^3 \\
1+9z^2 & 1+6y^2
\end{array}
\right|
} {
\left|
\begin{array}{cc}
44x^3 & 4y^3 \\
1+3x^2 & 1+6y^2
\end{array}
\right|} = -\frac{
\left|
\begin{array}{cc}
4 & 4 \\
10 & 7
\end{array}
\right|
}{
\left|
\begin{array}{cc}
-4 & 4 \\
4 & 7
\end{array}
\right|
} = -\frac {3} {11};
\end{equation*}
\begin{equation*}
\frac{d y}{d z} = -\frac{\left|\frac {\partial(f_1,f_2)}{\partial(x,z)}\right|}{\left|\frac {\partial(f_1,f_2)}{\partial(z,y)}\right|} = -\frac{
\left|
\begin{array}{cc}
4x^3 & 4z^3 \\
1+3x^2 & 1+9z^2
\end{array}
\right|
} {
\left|
\begin{array}{cc}
4x^3 & 4y^3 \\
1+3x^2 & 1+6y^2
\end{array}
\right|} = -\frac{
\left|
\begin{array}{cc}
-4 & 	4 \\
4 & 10
\end{array}
\right|
}{
\left|
\begin{array}{cc}
-4 & 4 \\
4 & 7
\end{array}
\right|
} = -\frac {14} {11}.
\end{equation*}
\end{enumerate}


\item  FOC - 2 pts, all the critical points - 4 pts, SOC - 4 pts.
\item  NDCQ - 1 pt, FOC - 2 pts, correct solution - 4 pts, SOC - 3 pts.

\item marking
\begin{itemize}
\item first derivative: 3 pt for differentiating w.r.t. \(t\) OR applying IFT. 2 pt for correct answer
\item second derivative: 1 pt for the idea of differentiating the expression for the first derivative, 2 pt for correct application of chain rule, 2 pt for correct answer
\end{itemize}
\textbf{Solution:}

Differentiate both sides of the equality w.r.t. \(t\):
\[
3u^2 \frac{du}{dt} + \frac{du}{dt} = \frac{\partial f}{\partial x}\frac{\partial x}{\partial t}+\frac{\partial f}{\partial y}\frac{\partial y}{\partial t}
\]
\[
\frac{du}{dt} = \frac{\frac{\partial f}{\partial x}\frac{\partial x}{\partial t}+\frac{\partial f}{\partial y}\frac{\partial y}{\partial t}}{3u^2+1} = \frac{-\frac{\partial f}{\partial x}+2\frac{\partial f}{\partial y}}{3u^2+1}
\]
In order to find \(\frac{d^2u}{dt^2}\) differentiate the latter expression w.r.t.  \(t\) (don't forget to apply the chain rule for \(\frac{\partial f}{\partial x}\) and \(\frac{\partial f}{\partial y}\)):
\[
\frac{d^2u}{dt^2} = \frac{(-\frac{\partial ^2f}{\partial x^2}\frac{\partial x}{\partial t}-\frac{\partial ^2f}{\partial y\partial x}\frac{\partial y}{\partial t} + 2\frac{\partial ^2f}{\partial x\partial y}\frac{\partial x}{\partial t}+2\frac{\partial ^2f}{\partial y^2}\frac{\partial y}{\partial t})(3u^2+1) -
6u\frac{\partial u}{\partial t}(-\frac{\partial f}{\partial x}+2\frac{\partial f}{\partial y})}
{(3u^2+1)^2}
\]
Finally substitute \(\frac{du}{dt}\):
\[
\frac{d^2u}{dt^2} = \frac{(\frac{\partial ^2f}{\partial x^2}-4\frac{\partial ^2f}{\partial \partial dx}\frac{\partial y}{\partial t} + 4\frac{\partial ^2f}{\partial y^2})(3u^2+1)^2 -
6u(-\frac{\partial f}{\partial x}+2\frac{\partial f}{\partial y})^2}
{(3u^2+1)^3}
\]

\item marking
\begin{enumerate}
\item 1 pt for FOC, 1 pt for showing that FOC has unique solution, 1 pt for SOC, 2 pt for derivative
\item 3 pt for envelope theorem (idea + correct value) OR finding \(f^*(a)\) and differentiating, 2 pt for linear approximation (idea and correct \(f^*\))
\end{enumerate}
\textbf{Solution:}

\begin{enumerate}
\item Write down first order condition for maximization problem:
\[
f'(x) = h'(x)-a=0
\]

Let's denote \(g(x)=h'(x)\). Then \(g'(x)=h''(x)<0\) therefore \(g(x)\) is a strictly decreasing function and the FOC equation has the only solution:
\[
x^* = g^{-1}(a)
\]

As long as \(f''(x)=h''(x)<0\) \(f(x)\) is a concave function and this point is indeed a global maximum.
\[
\frac{dx^*}{da} = \frac{dg^{-1}(a)}{da} =\frac{1}{\frac{dg}{dx}(x^*)}=\frac{1}{h''(x^*)}
\]

In fact there is no need to find \(x^*\), but find it's derivative by differentiating FOC w.r.t. \(a\):
\[
h''(x^*) \cdot \frac{dx^*}{da} - 1 = 0 \; \Longrightarrow \; \frac{dx^*}{da} = \frac{1}{h''(x^*)}
\]

\item Apply envelope theorem to the maximization problem:
\[
\frac{df^*}{da} = \frac{\partial f^*}{\partial a} = -x^* = -3
\]
\[
f^*(1.01) \approx 2015 -3\cdot0.01 = 2014.97
\]
\end{enumerate}

\item marking: a --- 15 pts, b --- 5 pts
\[
H = \begin{pmatrix}
\frac{-p}{x_l^2} & 0 \\
0 & \frac{p-1}{x_{NL}^2} \\
\end{pmatrix}
\]

Here $\Delta_1 < 0$, $\Delta_2 > 0$, the function is concave.

\item marking: NDCQ --- 2 pts, Lagrangian function --- 1 pt, FOC --- 1 pt, solution of FOC --- 3 pts, SOC ---
4 pts, better-off --- 4 pts, Envelope theorem --- 5 pts

Short answers:

Optimal bundle: $(x_L, x_{NL})=(w-pL, w-pL)$

SOC: $\det H = \frac{p(1-p)}{(w-pL)^2}>0$

Envelope theorem: $\partial E(u)^*/\partial p = \frac{-L}{w-pL}<0$

Alex is better-off. Proof:

Utility without insurance: $E(u_0)=p \ln (w-L) + (1-p) \ln w$.

Utility with insurance: $E(u)^*=\ln (w-pL)$

Using concavity of $\ln$: $E(u)^* > E(u_0)$.
\end{enumerate}


\subsection{Fall-exam retake, 25.01.2016}

\begin{enumerate}

\item Find the total differential for the function $f(x,y)=x^2y^2+xy^2+2x+4y$. Using the total differential find approximately $f(1.001,1.999)$

\item The system of equations defines $x(z)$ and $y(z)$:
\begin{equation}
\begin{cases}
x^2+zxy+y^2+6z+y^3=10 \\
y^3x^2+3x+2y+z=7 \nonumber
\end{cases}
\end{equation}
Find $x'(z)$ at the point $x=1$ and $y=1$.

\item Find the local maxima and minima of the function $f(x,y)=x^4+2y^4-xy$. Determine whether the extrema you have found are global or local.


\item Calculate all partial derivatives of the first and second order of $u$ with respect to $x$ and $y$ if $u=f (\xi, \eta)$ and $\xi=x+xy$, $\eta=x/y$.

\item Use Lagrange multipliers to find the height and radius of a cylinder with the least possible
surface area among those with a volume of $6\pi$ m$^3$. Make sure you check the second order
condition for minimisation.

\item Consider the function $f(x)=h(x)-ax$, where the function $h$ is twice differentiable and $h''(x)<0$ for all $x$. The global maximum of $f$ is denoted by $x^*(a)$.
\begin{enumerate}
\item Find $dx^*/da$
\item It is known that for $a=1$ the optimal point is $x^*=3$ and the value of maximum is $2016$. What is the approximate value of maximum for $a=1.01$?
\end{enumerate}


\end{enumerate}

\textbf{Part B.}

\begin{enumerate}[resume]

\item Two simple independent problems :)
\begin{enumerate}
\item \textbf{(10 points) }Find all values of the parameter $\lambda $ such that the function $f=2x^{2} +3y^{2} +z^{2} +4xy-2xz-2\lambda yz$ is convex.

\item \textbf{(10 points) }Write down the equation of the tangent plane to a surface $z=x^{3} -y^{3} $ at the point $(-1;1;-2)$.
\end{enumerate}

\item \textbf{(20 points) }Consider a problem of finding the extremal values of the function $f(x,y)=e^{x} +e^{y} +cx+cy$ under the constraint $x+y=c$, where $c$ is a positive parameter.

\begin{enumerate}
\item  Find out what kind of a problem you need to set: a problem of maximization or minimization?

\item Let $f(x*(c),y*(c))$ be the \textit{value function} of the problem. If $c$ slightly increases and becomes $c+\Delta c$, estimate the change in $f(x*(c),y*(c))$. Your answer should contain $c$ and $\Delta c$ only.
\end{enumerate}

\end{enumerate}


\subsection{Final exam, 02 april 2016}

\begin{enumerate}

\item Solve differential equation $y''+2y'+y=2x+3e^{-x} \sqrt{x+1} $.

\item Given the difference equation $3y_{t+2} +2y_{t+1} +\gamma y_{t} =5$ with the real parameter $\gamma $, find all the values of  $\gamma$ for whose the time path of this equation is convergent.
Choose some value of $\gamma$ beyond the found range and show that the corresponding time path is divergent.


\item Find all mixed Nash equilibria of the following game

\begin{tabular}{c|ccc}
  & d & e & f \\
\midrule
a & 0;0  & 0;-1 & 6;-2 \\
b & 0;0  & -1;1 & 7;0 \\
c & -1;5 & -1;4 & 5;9
\end{tabular}

\item For non-negative $y_i$ minimize the function $11y_1 + 5 y_2 + 4y_3 +4y_4$ subject to constraints $y_1 + y_2 + y_3 +0.1y_4 \geq 2$ and $2y_1+y_2+0.1y_3+y_4 \geq 7$.


\item Use the Lagrange multiplier method to find the maximum value of $f(x,y,z)=(7+x)^2(1+y)^2(7+z)^2$ among positive numbers subject to $x^2 + 49y^2 + z^2 = 100$.

\item James Bond added $x^3+7x-8y - 6y^3$ to the unknown homogeneous function $f(x, y)$. The new function was homogeneous once again! Please help the Secret Service agent recover the function $f$ if $f(1, 1)=1$.


\item Solve initial-value problem for the difference equation
\[
y_{t+3} -4y_{t+2} +5y_{t+1} -2y_{t} =1
\]
with the initial values $y_{0} =y_{1} =y_{2} =0$.

\textit{Hint: one of the characteristic roots is 1.}

\item Using Lagrange multipliers maximize the function $f(x_1,x_2,x_3)=2x_1-x_2-3x_3$ subject to constraints:   $3(x_1-1)^2+(x_2+1)^2+2x_3^2\leq 4$ and $x_1, x_2, x_3 \geq 0$. Find the point(s) of maximum and maximum value of $f$.

\end{enumerate}

\subsection{Final exam, 02 april 2016 - Marking scheme}

\begin{enumerate}
\item A1
\begin{itemize}
\item 5pt: characteristic equation 2pt, characteristic roots 1pt, complementary function 2pt
\item 2pt: linear part of particular solution (1pt for the form, 1pt for coefficients)
\item 3pt: second part of particular solution by variation of parameters method
\end{itemize}

\item A2
\begin{itemize}

\item 2pt: characteristic roots 1pt, for convergence their absolute values are less than one 1pt
\item 5pt: distinct real roots case 2pt, same roots case 1pt, complex roots case 2pts
\item 3pt: example of time path (particular solution 1pt, general solution 1pt, divergence 1pt)

\end{itemize}

\item A3:
\begin{itemize}
\item 2pts: elimination
\item 2pts: $E(u_1)$, $E(u_2)$
\item 2pts: best responses
\item 1pt: plot
\item 3pts: all NE
\end{itemize}

\item A4:
\begin{itemize}
\item 2 pts: dual
\item 2 pts: plot of constraints
\item 1 pt: optimal point on the graph
\item 1 pt: value of max
\item 2 pts: non-binding ($y_i=0$)
\item 2 pts: binding constraints
\end{itemize}

\item A5
\begin{itemize}
\item 2 pt: NDCQ
\item 1 pt: Lagrangean
\item 2 pt: First-order condition
\item 2 pt: solution of FOC
\item 2 pt: Second-order condition and
\end{itemize}

\item A6
\begin{itemize}
\item 3 pt: any attempts
\item 2 pt: homogeneity condition
\item 5 pt: correct answer
\end{itemize}


\item B7
\begin{itemize}
\item order of equation 1pt, characteristic equation 3pt, solutions 3pt
\item complementary function 4pt, particular solution 4pt: 2pt for the form + 2pt for finding the coefficient
\item general solution 2pt, definite solution 3pt
\end{itemize}

\item B8
\begin{itemize}
\item NDCQ 2pt, Lagrangean 2pt
\item FOC 4pt, solution of FOC 6pt, max value of function 2pt
\item SOC 4pt
\end{itemize}


\end{enumerate}
