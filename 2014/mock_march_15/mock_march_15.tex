\documentclass[12pt,a4paper]{article}
\usepackage[utf8]{inputenc}
\usepackage[english]{babel}
\usepackage{amsmath}
\usepackage{amsfonts}
\usepackage{amssymb}
\usepackage{enumitem}
\usepackage[left=1cm,right=1cm,top=1cm,bottom=1cm]{geometry}
\begin{document}
\thispagestyle{empty}
\textbf{Variant 1}. Mock. 25 March 2015. Please, don't forget to write you variant number. Sections A and B will make up 60\% and 40\% of the exam grade, respectively. Total duration of the exam is 120 min. Good luck! :) 



\textbf{SECTION A}

\begin{enumerate}

\item Find all the complex roots of the equation $(z+i)^3=1+i$.

\item Solve the differential equation $-2x^2y'=x^2+y^2$ with initial condition $y(e)=e$.

\item Consider the equation $y^3+xy+3x^2+2x^3=7$. 
\begin{enumerate}
\item Does this equation define the implicit function $y(x)$ at a point $(x=1,y=1)$?
\item If the function $y(x)$ is defined find its second order Taylor expansion
\end{enumerate}

\item The function $f(x,y)$ for positive $x$ and $y$ is defined as
\[
f(x,y)=x^{42}y^a + x^{b+1}\sqrt{y+x}+\frac{1}{y^a x^b}
\]

\begin{enumerate}
\item Find the values of $a$ and $b$ such that $f$ is homogeneous
\item For the values of  $a$ and $b$ you have found find the degree of homegeneity of $x\frac{\partial^2 f}{\partial y^2} +y\frac{\partial^2 f}{\partial x^2}$
\end{enumerate}

\item Consider two vectors, $\vec{x}=(1,0,-1)$ and $\vec{y}=(1,1,-2)$. Find a vector $\vec{z}$ with maximal length (called \textit{first principal component}) such that $\vec{z}$ is a linear combination of $\vec{x}$ and $\vec{y}$, i.e. $\vec{z}=a \vec{x} + b \vec{y}$ with weights satisfying the condition $a^2+b^2=1$.

\item The Fibonacci sequence is defined as $F_n=F_{n-1}+F_{n-2}$ with initial conditions $F_0=0$ and $F_1=1$. 
\begin{enumerate}
\item Find explicit formula for $F_n$
\item Find the ``golden ratio'', $\phi=\lim_{n\to\infty} F_{n+1}/F_n$
\item Is it true that $F_n$ is the closest integer to $\phi^n/\sqrt{5}$?
\end{enumerate}



\end{enumerate}

\textbf{SECTION B}

\begin{enumerate}[resume]

\item It is known that functions $1$, $x$ and $x^2$  are particular solutions of the second-order linear differential equation  $a(x)y''+b(x)y'+y=1$,  where $a(x)$  and $b(x)$  are continuous functions.
\begin{enumerate}
\item  Find the general solution of this equation
\item  Find $a(x)$ and  $b(x)$
\end{enumerate}
\item Let $f(x)$ be a concave function defined on $[0;\infty)$  and $f(0)=0$. Is it true that for $k \geq 1$ the following inequality holds: $kf(x)\geq f(kx)$?


\end{enumerate}
\newpage

\thispagestyle{empty}
\textbf{Variant 2}. Mock. 25 March 2015. Please, don't forget to write you variant number. Sections A and B will make up 60\% and 40\% of the exam grade, respectively. Total duration of the exam is 120 min. Good luck! :) 



\textbf{SECTION A}

\begin{enumerate}

\item Find all the complex roots of the equation $(z-i)^3=1+i$.

\item Solve the differential equation $-2x^2y'=x^2+y^2$ with initial condition $y(e)=e$.

\item Consider the equation $y^3+xy+3x^2+3x^3=8$. 
\begin{enumerate}
\item Does this equation define the implicit function $y(x)$ at a point $(x=1,y=1)$?
\item If the function $y(x)$ is defined find its second order Taylor expansion
\end{enumerate}

\item The function $f(x,y)$ for positive $x$ and $y$ is defined as
\[
f(x,y)=x^{42}y^a + x^{b+2}\sqrt{y+x}+\frac{1}{y^a x^b}
\]

\begin{enumerate}
\item Find the values of $a$ and $b$ such that $f$ is homogeneous
\item For the values of  $a$ and $b$ you have found find the degree of homegeneity of $x\frac{\partial^2 f}{\partial y^2} +y\frac{\partial^2 f}{\partial x^2}$
\end{enumerate}

\item Consider two vectors, $\vec{x}=(-1,0,1)$ and $\vec{y}=(1,1,-2)$. Find a vector $\vec{z}$ with maximal length (called \textit{first principal component}) such that $\vec{z}$ is a linear combination of $\vec{x}$ and $\vec{y}$, i.e. $\vec{z}=a \vec{x} + b \vec{y}$ with weights satisfying the condition $a^2+b^2=1$.

\item The Fibonacci sequence is defined as $F_n=F_{n-1}+F_{n-2}$ with initial conditions $F_0=0$ and $F_1=1$. 
\begin{enumerate}
\item Find explicit formula for $F_n$
\item Find the ``golden ratio'', $\phi=\lim_{n\to\infty} F_{n+1}/F_n$
\item Is it true that $F_n$ is the closest integer to $\phi^n/\sqrt{5}$?
\end{enumerate}



\end{enumerate}

\textbf{SECTION B}

\begin{enumerate}[resume]

\item It is known that functions $1$, $x$ and $x^2$  are particular solutions of the second-order linear differential equation  $a(x)y''+b(x)y'+y=1$,  where $a(x)$  and $b(x)$  are continuous functions.
\begin{enumerate}
\item  Find the general solution of this equation
\item  Find $a(x)$ and  $b(x)$
\end{enumerate}
\item Let $f(x)$ be a concave function defined on $[0;\infty)$  and $f(0)=0$. Is it true that for $k \geq 1$ the following inequality holds: $kf(x)\geq f(kx)$?


\end{enumerate}
\newpage

\thispagestyle{empty}
\textbf{Variant 3}. Mock. 25 March 2015. Please, don't forget to write you variant number. Sections A and B will make up 60\% and 40\% of the exam grade, respectively. Total duration of the exam is 120 min. Good luck! :) 



\textbf{SECTION A}

\begin{enumerate}

\item Find all the complex roots of the equation $(z+i)^3=1-i$.

\item Solve the differential equation $-2x^2y'=x^2+y^2$ with initial condition $y(e)=e$.

\item Consider the equation $y^3+xy+3x^2+4x^3=9$. 
\begin{enumerate}
\item Does this equation define the implicit function $y(x)$ at a point $(x=1,y=1)$?
\item If the function $y(x)$ is defined find its second order Taylor expansion
\end{enumerate}

\item The function $f(x,y)$ for positive $x$ and $y$ is defined as
\[
f(x,y)=x^{42}y^a + x^{b+3}\sqrt{y+x}+\frac{1}{y^a x^b}
\]

\begin{enumerate}
\item Find the values of $a$ and $b$ such that $f$ is homogeneous
\item For the values of  $a$ and $b$ you have found find the degree of homegeneity of $x\frac{\partial^2 f}{\partial y^2} +y\frac{\partial^2 f}{\partial x^2}$
\end{enumerate}

\item Consider two vectors, $\vec{x}=(1,0,-1)$ and $\vec{y}=(-1,-1,2)$. Find a vector $\vec{z}$ with maximal length (called \textit{first principal component}) such that $\vec{z}$ is a linear combination of $\vec{x}$ and $\vec{y}$, i.e. $\vec{z}=a \vec{x} + b \vec{y}$ with weights satisfying the condition $a^2+b^2=1$.

\item The Fibonacci sequence is defined as $F_n=F_{n-1}+F_{n-2}$ with initial conditions $F_0=0$ and $F_1=1$. 
\begin{enumerate}
\item Find explicit formula for $F_n$
\item Find the ``golden ratio'', $\phi=\lim_{n\to\infty} F_{n+1}/F_n$
\item Is it true that $F_n$ is the closest integer to $\phi^n/\sqrt{5}$?
\end{enumerate}



\end{enumerate}

\textbf{SECTION B}

\begin{enumerate}[resume]

\item It is known that functions $1$, $x$ and $x^2$  are particular solutions of the second-order linear differential equation  $a(x)y''+b(x)y'+y=1$,  where $a(x)$  and $b(x)$  are continuous functions.
\begin{enumerate}
\item  Find the general solution of this equation
\item  Find $a(x)$ and  $b(x)$
\end{enumerate}
\item Let $f(x)$ be a concave function defined on $[0;\infty)$  and $f(0)=0$. Is it true that for $k \geq 1$ the following inequality holds: $kf(x)\geq f(kx)$?


\end{enumerate}
\newpage

\thispagestyle{empty}
\textbf{Variant 4}. Mock. 25 March 2015. Please, don't forget to write you variant number. Sections A and B will make up 60\% and 40\% of the exam grade, respectively. Total duration of the exam is 120 min. Good luck! :) 



\textbf{SECTION A}

\begin{enumerate}

\item Find all the complex roots of the equation $(z-i)^3=1-i$.

\item Solve the differential equation $-2x^2y'=x^2+y^2$ with initial condition $y(e)=e$.

\item Consider the equation $y^3+xy+3x^2+5x^3=10$. 
\begin{enumerate}
\item Does this equation define the implicit function $y(x)$ at a point $(x=1,y=1)$?
\item If the function $y(x)$ is defined find its second order Taylor expansion
\end{enumerate}

\item The function $f(x,y)$ for positive $x$ and $y$ is defined as
\[
f(x,y)=x^{42}y^a + x^{b+4}\sqrt{y+x}+\frac{1}{y^a x^b}
\]

\begin{enumerate}
\item Find the values of $a$ and $b$ such that $f$ is homogeneous
\item For the values of  $a$ and $b$ you have found find the degree of homegeneity of $x\frac{\partial^2 f}{\partial y^2} +y\frac{\partial^2 f}{\partial x^2}$
\end{enumerate}

\item Consider two vectors, $\vec{x}=(-1,0,1)$ and $\vec{y}=(-1,-1,2)$. Find a vector $\vec{z}$ with maximal length (called \textit{first principal component}) such that $\vec{z}$ is a linear combination of $\vec{x}$ and $\vec{y}$, i.e. $\vec{z}=a \vec{x} + b \vec{y}$ with weights satisfying the condition $a^2+b^2=1$.

\item The Fibonacci sequence is defined as $F_n=F_{n-1}+F_{n-2}$ with initial conditions $F_0=0$ and $F_1=1$. 
\begin{enumerate}
\item Find explicit formula for $F_n$
\item Find the ``golden ratio'', $\phi=\lim_{n\to\infty} F_{n+1}/F_n$
\item Is it true that $F_n$ is the closest integer to $\phi^n/\sqrt{5}$?
\end{enumerate}



\end{enumerate}

\textbf{SECTION B}

\begin{enumerate}[resume]

\item It is known that functions $1$, $x$ and $x^2$  are particular solutions of the second-order linear differential equation  $a(x)y''+b(x)y'+y=1$,  where $a(x)$  and $b(x)$  are continuous functions.
\begin{enumerate}
\item  Find the general solution of this equation
\item  Find $a(x)$ and  $b(x)$
\end{enumerate}
\item Let $f(x)$ be a concave function defined on $[0;\infty)$  and $f(0)=0$. Is it true that for $k \geq 1$ the following inequality holds: $kf(x)\geq f(kx)$?


\end{enumerate}


\end{document}