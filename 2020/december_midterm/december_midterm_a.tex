\documentclass[12pt]{article} % размер шрифта

\usepackage{tikz} % картинки в tikz
\usepackage{microtype} % свешивание пунктуации

\usepackage{array} % для столбцов фиксированной ширины

\usepackage{url} % для вставки ссылок \url{...}

\usepackage{indentfirst} % отступ в первом параграфе

\usepackage{sectsty} % для центрирования названий частей
\allsectionsfont{\centering} % приказываем центрировать все sections

\usepackage{amsthm} % теоремы и доказательства

\theoremstyle{definition} % прямой шрифт в условии теорем
\newtheorem{theorem}{Теорема}[section]


\usepackage{amsmath, amssymb} % куча стандартных математических плюшек

\usepackage[top=2cm, left=1.5cm, right=1.5cm, bottom=2cm]{geometry} % размер текста на странице

\usepackage{lastpage} % чтобы узнать номер последней страницы

\usepackage{enumitem} % дополнительные плюшки для списков
%  например \begin{enumerate}[resume] позволяет продолжить нумерацию в новом списке
\usepackage{caption} % подписи к картинкам без плавающего окружения figure


\usepackage{fancyhdr} % весёлые колонтитулы
\pagestyle{fancy}
\lhead{Math for economists}
\chead{Section A}
\rhead{2020-12-25, ICEF}
\lfoot{Variant $\mu$}
\cfoot{Good luck!}
\rfoot{}
\renewcommand{\headrulewidth}{0.4pt}
\renewcommand{\footrulewidth}{0.4pt}



\usepackage{todonotes} % для вставки в документ заметок о том, что осталось сделать
% \todo{Здесь надо коэффициенты исправить}
% \missingfigure{Здесь будет картина Последний день Помпеи}
% команда \listoftodos — печатает все поставленные \todo'шки

\usepackage{booktabs} % красивые таблицы
% заповеди из документации:
% 1. Не используйте вертикальные линии
% 2. Не используйте двойные линии
% 3. Единицы измерения помещайте в шапку таблицы
% 4. Не сокращайте .1 вместо 0.1
% 5. Повторяющееся значение повторяйте, а не говорите "то же"

\usepackage{fontspec} % поддержка разных шрифтов
\usepackage{polyglossia} % поддержка разных языков

\setmainlanguage{english}
\setotherlanguages{russian}

\setmainfont{Linux Libertine O} % выбираем шрифт
% если Linux Libertine не установлен, то
% можно также попробовать Helvetica, Arial, Cambria и т.Д.

% чтобы использовать шрифт Linux Libertine на личном компе,
% его надо предварительно скачать по ссылке
% http://www.linuxlibertine.org/index.php?id=91&L=1

% на сервисах типа sharelatex.com этот шрифт есть :)

\newfontfamily{\cyrillicfonttt}{Linux Libertine O}
% пояснение зачем нужно шаманство с \newfontfamily
% http://tex.stackexchange.com/questions/91507/

\AddEnumerateCounter{\asbuk}{\russian@alph}{щ} % для списков с русскими буквами
%\setlist[enumerate, 2]{label=\asbuk\cdot),ref=\asbuk\cdot} % списки уровня 2 будут буквами а) б) ...

%% эконометрические и вероятностные сокращения
\DeclareMathOperator{\Cov}{Cov}
\DeclareMathOperator{\Corr}{Corr}
\DeclareMathOperator{\Var}{Var}
\DeclareMathOperator{\E}{E}
\DeclareMathOperator{\grad}{grad}
\def \hb{\hat{\beta}}
\def \hs{\hat{\sigma}}
\def \htheta{\hat{\theta}}
\def \s{\sigma}
\def \hy{\hat{y}}
\def \hY{\hat{Y}}
\def \v1{\vec{1}}
\def \e{\varepsilon}
\def \he{\hat{\e}}
\def \z{z}
\def \hVar{\widehat{\Var}}
\def \hCorr{\widehat{\Corr}}
\def \hCov{\widehat{\Cov}}
\def \cN{\mathcal{N}}
\def \RR{\mathbb{R}}


\def \putyourname{\fbox{
    \begin{minipage}{42em}
      Name, group no:\vspace*{3ex}\par
      \noindent\dotfill\vspace{2mm}
    \end{minipage}
  }
}



\begin{document}

\begin{enumerate}

\item (10 points) Consider the function $f(x, y) = x^3 + 3y^2 x^2$ and the point $A(1,1)$.
\begin{enumerate}
  \item Calculate the gradient of the function $f$ at the point $A$.
  \item Find the second order Taylor approximation in the neighborhood of $A$.
\end{enumerate}

\item (10 points) Consider the equation $3x^3 + 5y^5 + z^3 + z=10$. 
\begin{enumerate}
  \item Check whether the equation defines the function $z(x, y)$ at a point $A(1,1,1)$.
  \item Find $dz$ at the point $A$.
\end{enumerate}


\item (10 points) Find all local extrema of the function $f(x, y) = x^2 y - 3xy^2 + 5xy +2$ such that $x\neq 0$ and $y\neq 0$. 
Classify them.

\item (10 points) Find all local constrained extrema 
of the function $f(x, y, z) = x + 2y + 3z$ subject to $\ln x + \ln y + \ln z = 0$.
Do not forget to classify extrema. 

\item (10 points) Consider the function $f(x, y) = h(x) g(y) + ax^3$, where $h$ and $g$ are twice differentiable
and $a$ is a parameter. Let's denote the maximum point by $x^*(a)$ and $y^*(a)$ and assume that second order conditions for maximization are met.

Find the sign of $dx^*/da$. 

\item (10 points) The level curves of the function $f(x,y)$ are given by the equation $y - x^2 = c$.

Draw two level curves of the function $g(x,y)=f(x-2, |y| + 1)$.

\end{enumerate}



\newpage
\chead{Section B}

\begin{enumerate}[resume]
\item Consider a problem 
\[
  \begin{cases}
xyz \to \max \\
\text{s.t. } x+ y+ z = c \\
x, y, z >0 
  \end{cases}
\]
where $c$ is a parameter and $c>0$.
\begin{enumerate}
  \item (15 points) Solve this problem using first-order conditions. Use bordered
Hessian for sufficiency.
\item (5 points) Use the result of part a) to show that arithmetic mean 
$(x + y + z)/3$ is 
no less than the geometric mean  $(xyz)^{1/3}$.
\end{enumerate}

\item 
\begin{enumerate}
  \item (10 points) Consider a utility maximization problem
\[
  \begin{cases}
    u(x,y) \to \max \\
    \text{s.t. } p_x x + p_y y  = I \\
      x, y > 0 \\
  \end{cases},
\]
where $u\in C^1$ and parameters $p_x$, $p_y$ and $I$ are positive. 
Let $(x^*, y^*)$ be the solution of this problem. Form the value function
$V(p_x, p_y, I) = u(x^*, y^*)$. 

Using appropriate envelope theorem show that 
\[
  x^* = - \frac{\partial V}{\partial p_x} / \frac{\partial V}{\partial I}, \quad 
  y^* = - \frac{\partial V}{\partial p_y} / \frac{\partial V}{\partial I}.
\]

% (so called Roy’s identity).

\item (10 points) Let $F(x,y)$ be a function such that $F\in C^2$ and $F'_y  \neq 0$. 
The equation $F(x, y)=0$ defines the implicit function $y(x)$.

Find the expression for $d^2 y / dx^2$.

The expression should include only derivatives of $F(x,y)$.
\end{enumerate}


\end{enumerate}


\end{document}
