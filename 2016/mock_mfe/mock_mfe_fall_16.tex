\documentclass[12pt,a4paper]{article}
\usepackage[utf8]{inputenc}
\usepackage[english]{babel}
\usepackage{amsmath}
\usepackage{amsfonts}
\usepackage{amssymb}
\usepackage{enumitem}
\usepackage[left=1cm, right=1cm, top=1cm, bottom=1cm]{geometry}

\def \RR{\mathbb{R}}

\begin{document}
\thispagestyle{empty}
\textbf{Variant 1}. 2016-10-28. Please, don't forget to write you variant number. Sections A and B will make up 60\% and 40\% of the exam grade, respectively. Total duration of the exam is 120 min. Good luck! :)



\textbf{SECTION A}

\begin{enumerate}

\item Consider the function $f(x,y,z)=x^3+2xz-3z^3-y^2$. Using the total differential find the approximate value of $f(1.01,0.99,1.02)$.

\item Consider the system
\[
\begin{cases}
x^3 + y^3 + z^3 = 3 \\
x + x^3 + 2y + 3y^2 + xyz + z^3 = 9 \\
\end{cases}
\]
\begin{enumerate}
  \item Check whether the functions $y(z)$ and $x(z)$ are defined at a point $(1, 1, 1)$
  \item Find $y'(z)$ and $x'(z)$
\end{enumerate}

\item Consider the function $f(u)=g(x,y)$ where $x=u^2$ and $y=\cos u$. The function $g$ has continuous second derivatives everywhere. Find $f'(u)$ and $f''(u)$.

\item The curve on the plane is defined by the equation $x^4 + y^2 + y^4 = 3$.
\begin{enumerate}
  \item Find a vector that is orthogonal to the curve at the point $(1, 1)$
  \item Find a vector that is parallel to the curve at the point $(1, 1)$
\end{enumerate}

\item The function $g$ is monotonic. The function $f$ is the inverse of the function $g$. Find $f'(1)$ if it is known that $g(10)=1$, $g'(10)=5$, $g(1)=-2$, $g'(1)=4$.

\item Let $x$ be a vector, $x\in \mathbb{R}^n$, and $A$ be $n\times n$ matrix of constants. Consider the function $f(x) = x^T A x$.
\begin{enumerate}
 \item Clearly state the Young's theorem
 \item Express the Hesse matrix of $f$ using $A$ and $A^T$.
\end{enumerate}





\end{enumerate}

\textbf{SECTION B}

\begin{enumerate}[resume]


\item Consider the function  $f(x, y) = \sqrt[3]{x^3 + y^3}$.

\begin{enumerate}

\item (7 points) Find $\frac{\partial f}{\partial x}(0,0)$ and $\frac{\partial f}{\partial y}(0,0)$. Is the function $f(x, y)$ continuously differentiable everywhere?
\item (3 points) Find equation of the tangent plane to the graph of $z= f(x,y)$ at the origin.

\item (10 points) Let $\Delta f = f(x,y) - f(0,0)$. Compare $\Delta f$ with the $df$ (total differential) at
the origin. Base your comparison on the existence of the limit $\lim_{x\to0, y\to 0} \frac{\Delta f - df}{\sqrt{x^2 + y^2}}$ as $x\to 0$ and $y\to 0$.
\end{enumerate}

\item Cournot duopoly produces good $Y$, where $Y=y_1 + y_2$. Here $y_1$ is the output of the
first firm and $y_2$ is the output of the second firm. The inverse demand on
good is given by the formula $p(Y)=1/Y$, where is $p(Y)$ the price per unit. The total
costs of the firms are $TC_1(y_1) = 2y_1$ and $TC_2(y_2) = y_2$, respectively. Let the profit of the first
firm be $\pi_1 = p(y_1 + y_2)y_1 - 2y_1$ and the profit of the second firm be $\pi_2 = p(y_1 + y_2)y_2 - y_2$.

\begin{enumerate}
\item (8 points) Write down the system of the first-order
conditions $\begin{cases} \frac{\partial \pi_1}{\partial y_1} =0 \\  \frac{\partial \pi_2}{\partial y_2} =0 \end{cases}$ and solve it.
\item (12 points) Government decides to impose a per unit tax $t$ on
both firms. It will increase costs for them by $ty_1$ and $ty_2$ respectively. Rewrite the system
of first-order conditions accounting for the tax. Find $\frac{dy_1}{dt}$ and $\frac{dy_2}{dt}$ by referring to the
appropriate IFT. Check that IFT conditions are verifiable here.
\end{enumerate}


\end{enumerate}



\newpage

\thispagestyle{empty}
\textbf{Variant 2}. 2016-10-28. Please, don't forget to write you variant number. Sections A and B will make up 60\% and 40\% of the exam grade, respectively. Total duration of the exam is 120 min. Good luck! :)



\textbf{SECTION A}

\begin{enumerate}

\item Consider the function $f(x,y,z)=x^3+2xz-3z^3-2y^2$. Using the total differential find the approximate value of $f(1.01,0.99,1.02)$.

\item Consider the system
\[
\begin{cases}
x^3 + y^3 + 2z^3 = 4 \\
x + x^3 + 2y + 3y^2 + xyz + z^3 = 9 \\
\end{cases}
\]
\begin{enumerate}
  \item Check whether the functions $y(z)$ and $x(z)$ are defined at a point $(1, 1, 1)$
  \item Find $y'(z)$ and $x'(z)$
\end{enumerate}

\item Consider the function $f(u)=g(x,y)$ where $x=2u^2$ and $y=\cos u$. The function $g$ has continuous second derivatives everywhere. Find $f'(u)$ and $f''(u)$.

\item The curve on the plane is defined by the equation $x^4 + y^2 + 2y^4 = 4$.
\begin{enumerate}
  \item Find a vector that is orthogonal to the curve at the point $(1, 1)$
  \item Find a vector that is parallel to the curve at the point $(1, 1)$
\end{enumerate}

\item The function $g$ is monotonic. The function $f$ is the inverse of the function $g$. Find $f'(2)$ if it is known that $g(10)=2$, $g'(10)=5$, $g(2)=-2$, $g'(2)=4$.

\item Let $x$ be a vector, $x\in \mathbb{R}^n$, and $A$ be $n\times n$ matrix of constants. Consider the function $f(x) = x^T A x$.
\begin{enumerate}
 \item Clearly state the Young's theorem
 \item Express the Hesse matrix of $f$ using $A$ and $A^T$.
\end{enumerate}



\end{enumerate}

\textbf{SECTION B}

\begin{enumerate}[resume]


\item Consider the function  $f(x, y) = \sqrt[3]{x^3 + y^3}$.

\begin{enumerate}

\item (7 points) Find $\frac{\partial f}{\partial x}(0,0)$ and $\frac{\partial f}{\partial y}(0,0)$. Is the function $f(x, y)$ continuously differentiable everywhere?
\item (3 points) Find equation of the tangent plane to the graph of $z= f(x,y)$ at the origin.

\item (10 points) Let $\Delta f = f(x,y) - f(0,0)$. Compare $\Delta f$ with the $df$ (total differential) at
the origin. Base your comparison on the existence of the limit $\lim_{x\to0, y\to 0} \frac{\Delta f - df}{\sqrt{x^2 + y^2}}$ as $x\to 0$ and $y\to 0$.
\end{enumerate}

\item Cournot duopoly produces good $Y$, where $Y=y_1 + y_2$. Here $y_1$ is the output of the
first firm and $y_2$ is the output of the second firm. The inverse demand on
good is given by the formula $p(Y)=1/Y$, where is $p(Y)$ the price per unit. The total
costs of the firms are $TC_1(y_1) = 2y_1$ and $TC_2(y_2) = y_2$, respectively. Let the profit of the first
firm be $\pi_1 = p(y_1 + y_2)y_1 - 2y_1$ and the profit of the second firm be $\pi_2 = p(y_1 + y_2)y_2 - y_2$.

\begin{enumerate}
\item (8 points) Write down the system of the first-order
conditions $\begin{cases} \frac{\partial \pi_1}{\partial y_1} =0 \\  \frac{\partial \pi_2}{\partial y_2} =0 \end{cases}$ and solve it.
\item (12 points) Government decides to impose a per unit tax $t$ on
both firms. It will increase costs for them by $ty_1$ and $ty_2$ respectively. Rewrite the system
of first-order conditions accounting for the tax. Find $\frac{dy_1}{dt}$ and $\frac{dy_2}{dt}$ by referring to the
appropriate IFT. Check that IFT conditions are verifiable here.
\end{enumerate}


\end{enumerate}



\newpage

\thispagestyle{empty}
\textbf{Variant 3}. 2016-10-28. Please, don't forget to write you variant number. Sections A and B will make up 60\% and 40\% of the exam grade, respectively. Total duration of the exam is 120 min. Good luck! :)



\textbf{SECTION A}

\begin{enumerate}

\item Consider the function $f(x,y,z)=x^3+2xz-3z^3-3y^2$. Using the total differential find the approximate value of $f(1.01,0.99,1.02)$.

\item Consider the system
\[
\begin{cases}
x^3 + y^3 + 3z^3 = 5 \\
x + x^3 + 2y + 3y^2 + xyz + z^3 = 9 \\
\end{cases}
\]
\begin{enumerate}
  \item Check whether the functions $y(z)$ and $x(z)$ are defined at a point $(1, 1, 1)$
  \item Find $y'(z)$ and $x'(z)$
\end{enumerate}

\item Consider the function $f(u)=g(x,y)$ where $x=3u^2$ and $y=\cos u$. The function $g$ has continuous second derivatives everywhere. Find $f'(u)$ and $f''(u)$.

\item The curve on the plane is defined by the equation $x^4 + y^2 + 3y^4 = 5$.
\begin{enumerate}
  \item Find a vector that is orthogonal to the curve at the point $(1, 1)$
  \item Find a vector that is parallel to the curve at the point $(1, 1)$
\end{enumerate}

\item The function $g$ is monotonic. The function $f$ is the inverse of the function $g$. Find $f'(3)$ if it is known that $g(10)=3$, $g'(10)=5$, $g(3)=-2$, $g'(3)=4$.

\item Let $x$ be a vector, $x\in \mathbb{R}^n$, and $A$ be $n\times n$ matrix of constants. Consider the function $f(x) = x^T A x$.
\begin{enumerate}
 \item Clearly state the Young's theorem
 \item Express the Hesse matrix of $f$ using $A$ and $A^T$.
\end{enumerate}




\end{enumerate}

\textbf{SECTION B}

\begin{enumerate}[resume]


\item Consider the function  $f(x, y) = \sqrt[3]{x^3 + y^3}$.

\begin{enumerate}

\item (7 points) Find $\frac{\partial f}{\partial x}(0,0)$ and $\frac{\partial f}{\partial y}(0,0)$. Is the function $f(x, y)$ continuously differentiable everywhere?
\item (3 points) Find equation of the tangent plane to the graph of $z= f(x,y)$ at the origin.

\item (10 points) Let $\Delta f = f(x,y) - f(0,0)$. Compare $\Delta f$ with the $df$ (total differential) at
the origin. Base your comparison on the existence of the limit $\lim_{x\to0, y\to 0} \frac{\Delta f - df}{\sqrt{x^2 + y^2}}$ as $x\to 0$ and $y\to 0$.
\end{enumerate}

\item Cournot duopoly produces good $Y$, where $Y=y_1 + y_2$. Here $y_1$ is the output of the
first firm and $y_2$ is the output of the second firm. The inverse demand on
good is given by the formula $p(Y)=1/Y$, where is $p(Y)$ the price per unit. The total
costs of the firms are $TC_1(y_1) = 2y_1$ and $TC_2(y_2) = y_2$, respectively. Let the profit of the first
firm be $\pi_1 = p(y_1 + y_2)y_1 - 2y_1$ and the profit of the second firm be $\pi_2 = p(y_1 + y_2)y_2 - y_2$.

\begin{enumerate}
\item (8 points) Write down the system of the first-order
conditions $\begin{cases} \frac{\partial \pi_1}{\partial y_1} =0 \\  \frac{\partial \pi_2}{\partial y_2} =0 \end{cases}$ and solve it.
\item (12 points) Government decides to impose a per unit tax $t$ on
both firms. It will increase costs for them by $ty_1$ and $ty_2$ respectively. Rewrite the system
of first-order conditions accounting for the tax. Find $\frac{dy_1}{dt}$ and $\frac{dy_2}{dt}$ by referring to the
appropriate IFT. Check that IFT conditions are verifiable here.
\end{enumerate}


\end{enumerate}



\newpage


\thispagestyle{empty}
\textbf{Variant 4}. 2016-10-28. Please, don't forget to write you variant number. Sections A and B will make up 60\% and 40\% of the exam grade, respectively. Total duration of the exam is 120 min. Good luck! :)



\textbf{SECTION A}

\begin{enumerate}

\item Consider the function $f(x,y,z)=x^3+2xz-3z^3-4y^2$. Using the total differential find the approximate value of $f(1.01,0.99,1.02)$.

\item Consider the system
\[
\begin{cases}
x^3 + y^3 + 4z^3 = 6 \\
x + x^3 + 2y + 3y^2 + xyz + z^3 = 9 \\
\end{cases}
\]
\begin{enumerate}
  \item Check whether the functions $y(z)$ and $x(z)$ are defined at a point $(1, 1, 1)$
  \item Find $y'(z)$ and $x'(z)$
\end{enumerate}

\item Consider the function $f(u)=g(x,y)$ where $x=4u^2$ and $y=\cos u$. The function $g$ has continuous second derivatives everywhere. Find $f'(u)$ and $f''(u)$.

\item The curve on the plane is defined by the equation $x^4 + y^2 + 4y^4 = 6$.
\begin{enumerate}
  \item Find a vector that is orthogonal to the curve at the point $(1, 1)$
  \item Find a vector that is parallel to the curve at the point $(1, 1)$
\end{enumerate}

\item The function $g$ is monotonic. The function $f$ is the inverse of the function $g$. Find $f'(4)$ if it is known that $g(10)=4$, $g'(10)=5$, $g(4)=-2$, $g'(4)=4$.

\item Let $x$ be a vector, $x\in \mathbb{R}^n$, and $A$ be $n\times n$ matrix of constants. Consider the function $f(x) = x^T A x$.
\begin{enumerate}
 \item Clearly state the Young's theorem
 \item Express the Hesse matrix of $f$ using $A$ and $A^T$.
\end{enumerate}



\end{enumerate}

\textbf{SECTION B}

\begin{enumerate}[resume]


\item Consider the function  $f(x, y) = \sqrt[3]{x^3 + y^3}$.

\begin{enumerate}

\item (7 points) Find $\frac{\partial f}{\partial x}(0,0)$ and $\frac{\partial f}{\partial y}(0,0)$. Is the function $f(x, y)$ continuously differentiable everywhere?
\item (3 points) Find equation of the tangent plane to the graph of $z= f(x,y)$ at the origin.

\item (10 points) Let $\Delta f = f(x,y) - f(0,0)$. Compare $\Delta f$ with the $df$ (total differential) at
the origin. Base your comparison on the existence of the limit $\lim_{x\to0, y\to 0} \frac{\Delta f - df}{\sqrt{x^2 + y^2}}$ as $x\to 0$ and $y\to 0$.
\end{enumerate}

\item Cournot duopoly produces good $Y$, where $Y=y_1 + y_2$. Here $y_1$ is the output of the
first firm and $y_2$ is the output of the second firm. The inverse demand on
good is given by the formula $p(Y)=1/Y$, where is $p(Y)$ the price per unit. The total
costs of the firms are $TC_1(y_1) = 2y_1$ and $TC_2(y_2) = y_2$, respectively. Let the profit of the first
firm be $\pi_1 = p(y_1 + y_2)y_1 - 2y_1$ and the profit of the second firm be $\pi_2 = p(y_1 + y_2)y_2 - y_2$.

\begin{enumerate}
\item (8 points) Write down the system of the first-order
conditions $\begin{cases} \frac{\partial \pi_1}{\partial y_1} =0 \\  \frac{\partial \pi_2}{\partial y_2} =0 \end{cases}$ and solve it.
\item (12 points) Government decides to impose a per unit tax $t$ on
both firms. It will increase costs for them by $ty_1$ and $ty_2$ respectively. Rewrite the system
of first-order conditions accounting for the tax. Find $\frac{dy_1}{dt}$ and $\frac{dy_2}{dt}$ by referring to the
appropriate IFT. Check that IFT conditions are verifiable here.
\end{enumerate}


\end{enumerate}


\end{document}
