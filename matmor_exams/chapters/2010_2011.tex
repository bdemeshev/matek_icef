\subsection{MFE, fall semester exam, 18.01.2011}

Variant 4, Classteachers: A.Arlashin, G.Sharygin, S.Provornikov

Section A.
Answer all 6 questions from this section (10 marks each)

\begin{enumerate}
\item Let $\bar{n}$ be the unit normal at a point $M_0( x_0 , y_0 , z_0 )$ to the sphere represented by the equation $x^2+y^2+z^2=16$ in $R^3$. The point $M_0$ belongs to that sphere.
\begin{enumerate}
\item Find the components of $\bar{n}$ as the functions of $M_0$.
\item Find the set of points on the sphere at which the directional derivative of the function $u( x, y, z )=x-z$ in the direction of $\bar{n}$ turns to zero.
\end{enumerate}

\item Suppose $f(x,y)$ is a twice differentiable function. Let $x$ and $y$ be defined in terms of $u$, $v$ as follows: $x(u,v)=ue^{2v}$, $y(u,v)=u^2-v^2$. Let $F(u,v)=f(x(u, v), y(u, v))$.
Calculate $F_{uu}''$ and $F_{uv}''$.
\item Find the linear approximation of the function $\cos(e^x+e^y)$ in the neighborhood of the point $(\ln (\pi/4), \ln (\pi/4))$.

\item In the economy producing two goods $x$ and $y$ , the production possibilities set (PPS) is given by the system of inequalities
$ \left\{ x^2-xy+2.5y^2\leq 450, \, x\geq 0, \, y\geq 0 \right\} $
By definition the marginal rate of transformation is defined as
$ MRT=-\frac{dy}{dx}$
, where derivative (if it exists) is taken at a point on the boundary of PPS which is called Production Possibilities Frontier.
\begin{enumerate}
\item  By completing perfect squares prove that PPS is a bounded set.
\item Using the Implicit Function Theorem find MRT for that economy in terms of $x$, $y$. Consider $x>0$ and $y>0$.
\item Find point(s) at which $MRT=1$.
\end{enumerate}

\item A two goods producer is a monopolist and it faces the inverse demand functions $p_x=60-2x$ and $p_y=50-3y+x$ where $x$, $y$ are produced positive quantities. Let the total cost function be $C(x,y)=x^2+2y^2+xy$. Find production levels that maximize firm’s profit. Use second-order conditions to verify maximization.

\item In the macroeconomic linear IS-LM model for the closed economy
$Y=\bar{C}+m(Y-T)+G+\bar{I}-ar$ and $\bar{L}+bY-cr=M_s$, where $M_s$ is money supply, $r$ --- interest rate, $G$ --- government expenditures, $T$ --- lump sum tax and the constant parameters $\bar{C}>0$, $0<m<1$, $\bar{I}>0$, $a>0$, $\bar{L}>0$, $b>0$, $c>0$. Find the formulas for $dr/dT$, $dY/dT$. Assume that government expenditures and money supply are fixed exogenous variables.
\end{enumerate}


Section B
Answer both questions from this section (20 marks each)

\begin{enumerate}[resume]
\item In the economy described by the production possibilities set in question four from section A, the world prices on goods are set as follows: $p_x=5$, $p_y=4$. In order to maximize its national income the economy maximizes the objective function $5x+4y$ on the constraint set
$ \left\{ x^2-xy+2.5y^2\leq 450, \, x\geq 0, \, y\geq 0 \right\} $.
Find the produced quantities in that economy. You may not use the Kuhn-Tucker formulation here. Can the Weierstrass Theorem on attainment of the greatest and the least values be applicable in that maximization problem? Explain.

\item A consumer splits her time $\bar{L}$ hours a week between labor and leisure. Her utility function is represented by $u(c,l)=c^{\alpha}l^{1-\alpha}$, where $c$ is the amount of consumption and $l$ is leisure in hours and $0<\alpha<1$. The weekly budget constraint is written as $pc+wl=w\bar{L}$,
where $p$ is the price of consumption, $w$ is the hourly wage rate.
\begin{enumerate}
\item Find the consumer’s optimal bundle $(c^*,l^*)$. Justify your answer by checking second-order conditions or otherwise.
\item Let $\alpha=3/4$, $\bar{L}=168$, $p=16$, $w=8$. Using Envelope Theorem estimate the change in the maximum value of her utility if the wage rate has decreased by $0.5$.
\end{enumerate}
\end{enumerate}
