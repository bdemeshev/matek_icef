
\subsection{MFE, 2018-10-26}

\begin{enumerate}
  \item (10 points) Consider the function $f(x, y) = x^3 + y^3 + 2xy$.
  Using the total differential find the approximate value of $f(1.98, 0.99)$.
  \item (10 points) Consider the system
  \[
  \begin{cases}
  x^3 + y^3 + z^2 = 3 \\
  x + x^3 + 2y^3x = 4 \\
  \end{cases}
  \]

  \begin{enumerate}
    \item Check whether the functions $z(y)$ and $x(y)$ are defined at a point $(1, 1, 1)$;
    \item Find $z'(y)$ if possible.
 \end{enumerate}
 \item (10 points) Consider the function $f(x, y, z) = x^2 + 9y^2 + 2xy + \alpha z^2$.
 \begin{enumerate}
   \item Find the Hesse matrix. Clearly state the Young theorem if you use it.
   \item For each value of $\alpha$ find the definiteness of Hesse matrix.
 \end{enumerate}
 \item (10 points) Consider the function $u(x)=f(a, b, c)$, where $a=\alpha(q,r)$, $b=\beta(x)$,
 $c=\gamma(x,q)$, $q=x^2$ and $r=x^3$. All the functions are differentiable.
 Find $u'(x)$.
  \item (10 points) Consider the function $f(x, y) = x^2 + y^2 + 4y$.
  The microbe Veniamin is standing at $(1,1)$ and is moving according to a simple rule.
  From a point $(a, b)$ he jumps into the point $(a, b) - 0.01\grad f(a,b)$.
  \begin{enumerate}
    \item Where Veniamin will be after two jumps?
    \item What will be the approximate location of Veniamin after $2018$ jumps?
  \end{enumerate}

  \item (10 points) Let $h(a, b) = \int_a^b \exp(-t^2) \cdot dt$. Find the $\grad h(1, 2)$.

\item The domain of the function $z = xy - \frac{2}{3}x\sqrt{x} - \frac{1}{3}y^3 +5x+3y$
is the nonnegative quadrant $\{x\geq 0, y \geq 0\}$.
\begin{enumerate}
  \item (10 points) Find the equation of the tangent plane to the graph of $z$ at $(1,1,8)$.
  \item  (10 points) Let $\grad z(1,1) = c$. Find all such points that $\grad z(x,y)=c$.
\end{enumerate}

\item Two drivers on a lonely island get utility from fast driving and money.
Let $0\leq x_1 \leq 1$ be the speed of the first car
and $0\leq x_2 \leq 1$ be the speed of the second car, respectively.
They have the same amount of wealth $I>1$. Utilities of the drivers are
$U_1(x_1, x_2) = x_1 + I\cdot (1 - x_1 x_2)$ and $U_2(x_1, x_2) = \ln x_2 + I\cdot (1 - x_1 x_2)$.
\begin{enumerate}
  \item (7 points) On $(x_1, x_2)$-plane draw the solutions of
  the equations $\frac{\partial U_1}{\partial x_1}=0$ and $\frac{\partial U_2}{\partial x_2}=0$.
  \item (10 points) Let $(x_1^*, x_2^*)=(1, 1/I)$.
  Show that the system of inequalities hold $U_1(x_1^*, x_2^*)\geq U_1(x_1, x_2^*)$
  and  $U_2(x_1^*, x_2^*)\geq U_2(x_1^*, x_2)$.
  \item (3 points) Explain why even the small bribe
  offered by the second driver will stop the first driver from using his car?
\end{enumerate}


\end{enumerate}



\subsection{MFE, 2018-10-26 — marking}


\begin{enumerate}
  \item Correct formula of the total differential (2).

Correct partial derivatives (2).

Correct calculation (6).

  \item All three IFT conditions (4).

Correct formula for derivative (2).

Correct calculations (4).

  \item \textbf{Problem 3.}

  Consider the function $f(x,y,z)=x^2+10y^2+2xy+\alpha z$\\
  (a) Find the Hesse matrix. Clearly state the Young theorem if you use it.\\
  (b) For each value of $\alpha$ find the definiteness of Hesse matrix.\\

  \textbf{Solution.}
  (a)\;
  \[
  \begin{pmatrix}
  2& 2 &0\\
  2 &20& 0\\
  0& 0& 2\alpha
  \end{pmatrix}
  \]

  (b) $\Delta_1=2>0$, $\Delta_2=40-4=36>0$, $\Delta_3=72\alpha$.\\
  1. If $\alpha >0$ then all corner principal minors are greater than 0. According to Silvester criterion the Hesse matrix is positive definite.

  2. If $\alpha<0$ then $\Delta_1>0$, $\Delta_2>0$, $\Delta_3<0$. According to Silvester criterion the matrix is indefinite.

  3. If $\alpha=0$ then we need to use generalized Silvester criterion (three corner determinants are not enough!).

  \begin{itemize}
  \item Minors of the 1st order: 2, 20, $2\alpha$. All are non-negative.
  \item Minors of the 2nd order: 36, 0, 0. All are non-negative.
  \item Minors of the 3rd order (the determinant of all matrix) is 0, also non-negative.
  \end{itemize}

  Thus, the matrix is positive semi-definite.
  \smallskip

  \textbf{Marking scheme}\\
  (a)\\
  2 pts for the Young theorem\\
  2 pts for all partial derivatives\\
  1 pt for the Hesse matrix

  OR

  4 pts for all partial derivatives if no Young theorem is used\\
  1 pt for the Hesse matrix

  (b)\\
  1 pt for all three corner principal minors\\
  1 pt for the case $\alpha > 0$\\
  1 pt for the case $\alpha <0$\\
  2 pts for the case $\alpha =0$


  \item

  Consider the function $u(x)=f(a,b,c)$, where $a=\alpha(q,r)$, $b=\beta(x)$, $c=\gamma(x,q)$, $q=-x^2$ and $r=x^3$. All the functions are differentiable. Find $u'(x)$.

  \textbf{Solution+Marking scheme}

  \[
  u'=\frac{\partial f}{\partial a}\underbrace{\frac{\partial \alpha}{\partial q}}_{1 pt}\underbrace{(-2x)}_{1 pt}+\frac{\partial f}{\partial a}\underbrace{\frac{\partial \alpha}{\partial r}}_{1 pt}\underbrace{3x^2}_{1 pt}+\frac{\partial f}{\partial b}\underbrace{\frac{\partial \beta}{\partial x}}_{1 pt}+\frac{\partial f}{\partial c}\underbrace{\frac{\partial \gamma}{\partial x}}_{1 pt}+\frac{\partial f}{\partial c}\underbrace{\frac{\partial\gamma}{\partial q}}_{1 pt}\underbrace{(-2x)}_{1 pt}
  \]

  + 2 pts for the correct form of the answer

  \textbf{Penalties}\\
  -1 pt for writing down each extra term in the answer\\
  -1 pt for using $a$ instead of $\alpha$ in $\frac{\partial \alpha}{\partial q}$ (or $\frac{\partial \alpha}{\partial r}$). Not more than 1 pt penalty even if the mistake appeared twice.\\
  -1 pt for using $b$ instead of $\beta$ (the same as above)\\
  -1 pt for using $с$ instead of $\gamma$ (the same as above)\\
  -1 pt for using $u$ instead of $f$ (the same as above)

 \item
   \begin{enumerate}
     \item  Correct expressions of partial derivatives - 2 pts.
      Correct values of partial derivatives at the point (1,1) - 1 pt.
      Correct coordinates of the point after the first jump - 1 pt. Gradient at new point
      - 1pt. Coordinates of the point after the second jump - 1 pt.
     \item 2 pts for each coordinate.
   \end{enumerate}

 \item
 Formula of derivative of definite integral - 2 pts.
 Expression of each partial derivative - 2 pts.
 Correct value of each derivative - 2 pts.

 \item
 \begin{enumerate}
   \item Derivatives = 6 pts (3 pts + 3 pts), formula of tangent plane = 4 pts.
   \item Gradient at (1, 1) = 4 pts, system of equation = 3 pts, answer = 3 pts.

   Answer: all points $(x, y)$ such that $y=\sqrt{x}$.
 \end{enumerate}
  \item


  \begin{enumerate}
    \item Derivatives = 4 pts (2 pts + 2 pts), plot = 3 pts. $x_2=1/I$, $x_2=1/Ix_1$.
    \item Inequality $I \geq I$ = 5 pts (4 pts for statement and 1 pt for proof),
    hard inequality = 5 pts (1 pt for statement and 4 pts for proof).
    \item 3 pts.
  \end{enumerate}

\end{enumerate}



\subsection{Midterm, 2018-12-27}

\begin{enumerate}
  \item (10 points) Find the limit or prove that it does not exist
  \[
  \lim_{x,y \to 0} \frac{1 - \cos(x+2y)}{\sin(xy)}
  \]

  \item (10 points) Using Lagrange multipliers find the extrema of the function $f(x,y) = x^2 + 4xy + y^2$ subject to $x^2 + 2y^2 = 16$.

 \item (10 points) Consider the function $u(x,y) = x^2 - 4xy + ay^2 -\ln(xy)$ for $x>0$ and $y>0$.
 For which values of $a$ the function $u$ is convex?

 \item (10 points) Find the second order Taylor approximation of a function $f(x,y) = x^5y^3 + 3x^2y$ at a point $x=1$, $y=2$.

  \item (10 points) Use Lagrange multipliers to find the height and radius of a cylinder
  with the maximal volume among those with a surface $S=10\pi$. Make sure you check the second order
condition for maximisation.

  \item (10 points) Let $h(x, y) = kx^2 + 6xy + 14y^2 + 4y + 10$.
  \begin{enumerate}
    \item Find the minimal value of the function $h$ for $k=2$.
    \item Using envelope theorem find approximate minimal value of $h$ for $k=1.98$.
  \end{enumerate}


\item This is a road construction costs minimization problem. Let the terrain profile be represented by the function $y(t)=\begin{cases}
3-3|t|, \text{ if } |t|\leq 1\\
0, \text{ otherwise}
\end{cases}$.
A road works start from the east of this hill (on the negative half-axis). Excavation costs can be found by the formula
\[
I(a, b) = \int_{-b/a}^0 (at + b - y(t))^2 \, dt
\]
where $at+b$  is the road profile we need to find with the constants $a>0$, $b>0$, and $b/a\geq 1$.
\begin{enumerate}
  \item (15 points) Find the Hessian matrix of $I(a,b)$ and check its sign-definitness.
  \item (5 points) Let $(a^*, b^*)$ be the solution of the first-order conditions for the minimization problem.
  Justify your choice for the $(a^*, b^*)$ values.
\end{enumerate}

\item (continuation of problem 7) (20 points)

Allowable grade of the road satisfies constraint $a\leq 1$. Under this constraint solve the problem $I(a, b) \to \min$ with respect to $b$.

\end{enumerate}


S

\subsection{Midterm, 2018-12-27, marking}

\begin{enumerate}
  \item 10 points
  \item 10 points
  \item Each derivative 1 point (total 5 points), Hesse matrix – 1
  point, conditions for positive (semi) definiteness – 1 points, correct answer ($a$ greater than some real number) – 3 points
 \item  Each derivative and its value 1 point (total 5 points), correct formula for Taylor approximation – 5
points
  \item 10 points
  \item 10 points 
  \item Many solutions are possible. First solution, without explicit $I(a,b)$. Each derivative $I'_a$, $I'_b$ — 4 points. Second solution, with explicit $I(a,b)$. Function $I(a,b)$ — 4 points, each first derivative — 2 points.
  Three second derivatives — 1 point each. Hesse matrix — 1 point. Definitness — 3 points.

  Point b. FOC — 2 points, SOC — 3 points.
  \item FOC for correct $I(a,b)$ — 10 points, SOC for correct $I(a,b)$ — 10 points.

FOC for incorrect $I(a,b)$ — 5 points, SOC for incorrect $I(a,b)$ — 5 points.

\end{enumerate}
