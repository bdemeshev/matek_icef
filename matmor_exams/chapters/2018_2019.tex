
\subsection{MFE, 2018-10-26}

\begin{enumerate}
  \item (10 points) Consider the function $f(x, y) = x^3 + y^3 + 2xy$.
  Using the total differential find the approximate value of $f(1.98, 0.99)$.
  \item (10 points) Consider the system
  \[
  \begin{cases}
  x^3 + y^3 + z^2 = 3 \\
  x + x^3 + 2y^3x = 4 \\
  \end{cases}
  \]

  \begin{enumerate}
    \item Check whether the functions $z(y)$ and $x(y)$ are defined at a point $(1, 1, 1)$;
    \item Find $z'(y)$ if possible.
 \end{enumerate}
 \item (10 points) Consider the function $f(x, y, z) = x^2 + 9y^2 + 2xy + \alpha z^2$.
 \begin{enumerate}
   \item Find the Hesse matrix. Clearly state the Young theorem if you use it.
   \item For each value of $\alpha$ find the definiteness of Hesse matrix.
 \end{enumerate}
 \item (10 points) Consider the function $u(x)=f(a, b, c)$, where $a=\alpha(q,r)$, $b=\beta(x)$,
 $c=\gamma(x,q)$, $q=x^2$ and $r=x^3$. All the functions are differentiable.
 Find $u'(x)$.
  \item (10 points) Consider the function $f(x, y) = x^2 + y^2 + 4y$.
  The microbe Veniamin is standing at $(1,1)$ and is moving according to a simple rule.
  From a point $(a, b)$ he jumps into the point $(a, b) - 0.01\grad f(a,b)$.
  \begin{enumerate}
    \item Where Veniamin will be after two jumps?
    \item What will be the approximate location of Veniamin after $2018$ jumps?
  \end{enumerate}

  \item (10 points) Let $h(a, b) = \int_a^b \exp(-t^2) \cdot dt$. Find the $\grad h(1, 2)$.

\item The domain of the function $z = xy - \frac{2}{3}x\sqrt{x} - \frac{1}{3}y^3 +5x+3y$
is the nonnegative quadrant $\{x\geq 0, y \geq 0\}$.
\begin{enumerate}
  \item (10 points) Find the equation of the tangent plane to the graph of $z$ at $(1,1,8)$.
  \item  (10 points) Let $\grad z(1,1) = c$. Find all such points that $\grad z(x,y)=c$.
\end{enumerate}

\item Two drivers on a lonely island get utility from fast driving and money.
Let $0\leq x_1 \leq 1$ be the speed of the first car
and $0\leq x_2 \leq 1$ be the speed of the second car, respectively.
They have the same amount of wealth $I>1$. Utilities of the drivers are
$U_1(x_1, x_2) = x_1 + I\cdot (1 - x_1 x_2)$ and $U_2(x_1, x_2) = \ln x_2 + I\cdot (1 - x_1 x_2)$.
\begin{enumerate}
  \item (7 points) On $(x_1, x_2)$-plane draw the solutions of
  the equations $\frac{\partial U_1}{\partial x_1}=0$ and $\frac{\partial U_2}{\partial x_2}=0$.
  \item (10 points) Let $(x_1^*, x_2^*)=(1, 1/I)$.
  Show that the system of inequalities hold $U_1(x_1^*, x_2^*)\geq U_1(x_1, x_2^*)$
  and  $U_2(x_1^*, x_2^*)\geq U_2(x_1^*, x_2)$.
  \item (3 points) Explain why even the small bribe
  offered by the second driver will stop the first driver from using his car?
\end{enumerate}

\end{enumerate}
