% arara: xelatex
\documentclass[12pt]{article} % размер шрифта

\usepackage{tikz} % картинки в tikz
\usepackage{microtype} % свешивание пунктуации

\usepackage{array} % для столбцов фиксированной ширины

\usepackage{url} % для вставки ссылок \url{...}

\usepackage{indentfirst} % отступ в первом параграфе

\usepackage{sectsty} % для центрирования названий частей
\allsectionsfont{\centering} % приказываем центрировать все sections

\usepackage{amsthm} % теоремы и доказательства

\theoremstyle{definition} % прямой шрифт в условии теорем
\newtheorem{theorem}{Теорема}[section]


\usepackage{amsmath, amssymb} % куча стандартных математических плюшек

\usepackage[top=2cm, left=1.5cm, right=1.5cm, bottom=2cm]{geometry} % размер текста на странице

\usepackage{lastpage} % чтобы узнать номер последней страницы

\usepackage{enumitem} % дополнительные плюшки для списков
%  например \begin{enumerate}[resume] позволяет продолжить нумерацию в новом списке
\usepackage{caption} % подписи к картинкам без плавающего окружения figure


\usepackage{fancyhdr} % весёлые колонтитулы
\pagestyle{fancy}
\lhead{Math for economists}
\chead{Section A+B}
\rhead{Retake 2022-02-01, ICEF}
\lfoot{Variant $\alpha$}
\cfoot{Good luck!}
\rfoot{}
\renewcommand{\headrulewidth}{0.4pt}
\renewcommand{\footrulewidth}{0.4pt}



\usepackage{todonotes} % для вставки в документ заметок о том, что осталось сделать
% \todo{Здесь надо коэффициенты исправить}
% \missingfigure{Здесь будет картина Последний день Помпеи}
% команда \listoftodos — печатает все поставленные \todo'шки

\usepackage{booktabs} % красивые таблицы
% заповеди из документации:
% 1. Не используйте вертикальные линии
% 2. Не используйте двойные линии
% 3. Единицы измерения помещайте в шапку таблицы
% 4. Не сокращайте .1 вместо 0.1
% 5. Повторяющееся значение повторяйте, а не говорите "то же"

\usepackage{fontspec} % поддержка разных шрифтов
\usepackage{polyglossia} % поддержка разных языков

\setmainlanguage{english}
\setotherlanguages{russian}

\setmainfont{Linux Libertine O} % выбираем шрифт
% если Linux Libertine не установлен, то
% можно также попробовать Helvetica, Arial, Cambria и т.Д.

% чтобы использовать шрифт Linux Libertine на личном компе,
% его надо предварительно скачать по ссылке
% http://www.linuxlibertine.org/index.php?id=91&L=1

% на сервисах типа sharelatex.com этот шрифт есть :)

\newfontfamily{\cyrillicfonttt}{Linux Libertine O}
% пояснение зачем нужно шаманство с \newfontfamily
% http://tex.stackexchange.com/questions/91507/

\AddEnumerateCounter{\asbuk}{\russian@alph}{щ} % для списков с русскими буквами
%\setlist[enumerate, 2]{label=\asbuk\cdot),ref=\asbuk\cdot} % списки уровня 2 будут буквами а) б) ...

%% эконометрические и вероятностные сокращения
\DeclareMathOperator{\Cov}{Cov}
\DeclareMathOperator{\Corr}{Corr}
\DeclareMathOperator{\Var}{Var}
\DeclareMathOperator{\E}{E}
\DeclareMathOperator{\grad}{grad}
\def \hb{\hat{\beta}}
\def \hs{\hat{\sigma}}
\def \htheta{\hat{\theta}}
\def \s{\sigma}
\def \hy{\hat{y}}
\def \hY{\hat{Y}}
\def \v1{\vec{1}}
\def \e{\varepsilon}
\def \he{\hat{\e}}
\def \z{z}
\def \hVar{\widehat{\Var}}
\def \hCorr{\widehat{\Corr}}
\def \hCov{\widehat{\Cov}}
\def \cN{\mathcal{N}}
\def \RR{\mathbb{R}}


\def \putyourname{\fbox{
    \begin{minipage}{42em}
      Name, group no:\vspace*{3ex}\par
      \noindent\dotfill\vspace{2mm}
    \end{minipage}
  }
}



\begin{document}

\begin{enumerate}

\item (10 points) Find the limit or prove that it does not exist
  \[
  \lim_{x \to 0, y \to 0} \frac{x^2 y^2}{x^4 + 3y^4}.
  \]
  

\item (10 points) Consider the function 
\[
f(x, y) = \int_0^{2x} 3e^{u^2} \, du  + \int_0^{3y} 2\cos(u^2) \, du.   
\]

Find the gradient $\grad f$ at the point $(0, 0)$.

\item (10 points) Consider the system 
  \[
  \begin{cases}
    x + y + z = 2 \\
    2x^2 + 2y^2 = z^2 \\
  \end{cases}.  
  \]
  \begin{enumerate}
    \item Are the functions $x(z)$ and $y(z)$ defined in a neighborhood of the point $A(x=1, y=-1, z=2)$?
    \item Find $dx/dz$ at the point $A$ if possible. 
  \end{enumerate}

\item (10 points)  
The set $S$ is defined by $S = \{(x, y) \in \mathbb{R}^2 \mid 0 \leq y \leq 2- x^2\}$. 
Two rectangles one on the top of the other are inscribed in $S$, 
thus they have the common side and the upper vertices lie on this parabola.
Let $A_1 + A_2$ be the sum of their areas, where $A_1 >0$, $A_2 >0$. 

Consider the maximization problem $A_1 + A_2 \to \max$.
\begin{enumerate}
  \item Solve the maximization problem or show that the maximum does not exist.
  \item Check whether the Weierstrass theorem is applicable.
\end{enumerate}

\item (10 points)  Using Lagrange multiplier method find and classify 
the constrained extrema of $f(x, y, z) =  5x +4y + 8z$ subject to $x^2 + y^2 + z^2 = 1$.

\item (10 points)  Find all stationary points of $f(x,y) = x^4 + y^4-4xy + 1$. 
Classify them as local minimum, maximum or saddle point.


%\item (10 points) The turtle starts at the point $(0, 0)$. 
%Every day she continues her way at constant speed in constant direction. 
%Every midnight she drops her speed by 30\% and turns $90^{\circ}$ counterclockwise. 
%On the first day she goes by 1 kilometer in the positive direction of $OX$ axis.

%What is the ultimate destination of the turtle?


\end{enumerate}

\begin{enumerate}[resume]

  \item (20 points) Let the demand and supply for an ice-cream on the sunny day be $q_D=D(p, T, d)$ 
  and $q_S=S(p,T)$ correspondingly. Here $p$ is the price, 
  $T$ is the temperature on this day, $d$ — distance of the selling place from the center of the park, 
  $D_p<0$, $D_T>0$, $S_p>0$, $S_T<0$, $D_d<0$. 

  \begin{enumerate}
  \item Find analytically how the equilibrium price $p^{\ast}$ changes with the increase of $T$. 
  How does it change with the increase of $d$? 
  \item Let $q^{\ast}$ be the equilibrium supply quantity. 
  Find $\frac{\partial{q^{\ast}}}{\partial T}$. 
  Find the condition when $\frac{\partial{q^{\ast}}}{\partial T}<0$.
  \end{enumerate}
    
  
\item (20 points) We wish to build a picnic zone for the travellers along a highway. The picnic zone should be rectangular 
with an area of 2000 m$^2$ and should have a fence on the three sides not adjacent to the highway. 
The price of one meter of fence is equal to \$ 10.
\begin{enumerate}
\item Find the dimensions of the picnic area that minimize the fencing costs.
\item Using hessian or otherwise check that you have found the costs-minimizing solution.
\item Using the Envelope theorem estimate the change in the costs if we decrease the area of the picnic zone by 1 m$^2$.
\end{enumerate}
\end{enumerate}





\end{document}
