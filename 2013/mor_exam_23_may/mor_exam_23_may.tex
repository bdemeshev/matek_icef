\documentclass[12pt,a4paper]{article}
\usepackage[utf8]{inputenc}
\usepackage[russian,british]{babel}
\usepackage[T2A]{fontenc}
\usepackage{amsmath}
\usepackage{amsfonts}
\usepackage{amssymb}
\usepackage{enumitem}
\usepackage[left=12mm,right=12mm,top=1cm,bottom=1cm]{geometry}



\setlength{\topsep}{0pt}


\begin{document}



\section*{Variant 1. Methods of optimal solutions, 23.05.14}

You need to solve all the problems from Sections A, B, C.

\pagestyle{empty}

\textbf{SECTION A}

\begin{enumerate}[resume]
\item (15 points) A problem with the mixed constraints is given: $3x_1 x_2-x_2^3 \to \max$

subject to $2x_1+5x_2 \geq 20$, $x_1-2x_2=5$, $x_1, x_2 \geq 0$.
\begin{enumerate}
\item Check NDCQ conditions.
\item  Form the Langrangian function.
\item Solve the maximization problem by the use of Langrange method or otherwise.
\item Justify that the maximum point was found.
\end{enumerate}
\item  (20 points)   Consider the linear programming problem with parameter $\beta$ and nonnegative $x_i$:
\begin{align*}
3x_1+x_2-x_3+4x_4 \to \max \\
x_1+x_2-x_3+x_4\leq 1 \\
\beta x_1+x_2+x_3+2x_4 \leq 2
\end{align*}

\begin{enumerate}
\item Find the optimal values of the primal variables for $\beta=6$.
\item Find the function $\phi(\beta)$, where $\phi(\beta)$  is the maximum value of the objective function for fixed value of $\beta$.
\item Sketch the graph of $\phi(\beta)$.
\end{enumerate}


\end{enumerate}

\textbf{SECTION B}

\begin{enumerate}[resume]
\item (15 points)  Consider the system of difference equation:
\[
\begin{cases}
x_{t+1}=x_t-y_t+6 \\
y_{t+1}=2x_t-y_t+3
\end{cases}
\]

\begin{enumerate}
\item Solve the system of difference equations 
\item Explore the stability of its solutions.
\end{enumerate}

\item   (15 points)   Solve the  Euler’s equation $x^2y''-4xy'+6y=0$ on the interval $(0,\infty)$ by using the substitution $x=e^t$ or otherwise. The answer should be written as a function of $x$.

\end{enumerate}


\textbf{SECTION C}

\begin{enumerate}[resume]
\item  (15 points)   Consider the following bimatrix game:


\begin{tabular}{c|ccc}
 & D & E & F \\ 
\hline 
A & 4;3 & 2;2 & 2;1  \\ 
B & -2;8 & 4;7 & 2;4  \\ 
C & 1;2 & 3;1 & 3;3  \\ 
\end{tabular} 

\begin{enumerate}
\item Find all the pure and mixed Nash equilibria
\item State whether the equilibria are Pareto-optimal
\end{enumerate}

\item  (20 points)  Three players play the following game. Simultenuously each of them chooses one of three numbers: $1$, $2$ and $3$. If all players choose the same number, then everyone gets nothing. Otherwise the player with smallest unique number receives two rubles, and other players pay one ruble each. Example: if $1$, $1$ and $3$ are chosen, then the winner is the player who chose $3$. She receives two rubles, and  each of the other two players pays one ruble.
\begin{enumerate}
\item Find all the pure and mixed Nash equilibria
\item State whether the equilibria are Pareto-optimal
\end{enumerate}



\end{enumerate}

\newpage


\section*{Variant 2. Methods of optimal solutions, 23.05.14}

You need to solve all the problems from Sections A, B, C.

\pagestyle{empty}

\textbf{SECTION A}

\begin{enumerate}
\item (15 points) A problem with the mixed constraints is given: $12x_1 x_2-5x_2^3 \to \max$

subject to $3x_1+2x_2 \geq 6$, $x_1-x_2=1$, $x_1, x_2 \geq 0$.
\begin{enumerate}
\item Check NDCQ conditions.
\item  Form the Langrangian function.
\item Solve the maximization problem by the use of Langrange method or otherwise.
\item Justify that the maximum point was found.
\end{enumerate}
\item  (20 points)   Consider the linear programming problem with parameter $\beta$ and nonnegative $x_i$:
\begin{align*}
4x_1+3x_2+2x_3-2x_4 \to \max \\
x_1+x_2+2x_3+2x_4\leq 1 \\
\beta x_1+2x_2+3x_3-x_4 \leq 5
\end{align*}

\begin{enumerate}
\item Find the optimal values of the primal variables for $\beta=4$.
\item Find the function $\phi(\beta)$, where $\phi(\beta)$  is the maximum value of the objective function for fixed value of $\beta$.
\item Sketch the graph of $\phi(\beta)$.
\end{enumerate}


\end{enumerate}

\textbf{SECTION B}

\begin{enumerate}[resume]
\item (15 points)  Consider the system of difference equation:
\[
\begin{cases}
x_{t+1}=x_t-y_t+2 \\
y_{t+1}=5x_t-y_t+1
\end{cases}
\]

\begin{enumerate}
\item Solve the system of difference equations 
\item Explore the stability of its solutions.
\end{enumerate}

\item   (15 points)   Solve the  Euler’s equation $x^2y''-xy'-3y=0$ on the interval $(0,\infty)$ by using the substitution $x=e^t$ or otherwise. The answer should be written as a function of $x$.

\end{enumerate}


\textbf{SECTION C}

\begin{enumerate}[resume]
\item  (15 points)   Consider the following bimatrix game:


\begin{tabular}{c|ccc}
 & D & E & F \\ 
\hline 
A & 4;4 & 2;3 & 2;2  \\ 
B & -2;9 & 4;8 & 2;5  \\ 
C & 1;3 & 3;2 & 3;4  \\ 
\end{tabular} 

\begin{enumerate}
\item Find all the pure and mixed Nash equilibria
\item State whether the equilibria are Pareto-optimal
\end{enumerate}

\item  (20 points)  Three players play the following game. Simultenuously each of them chooses one of three numbers: $1$, $2$ and $3$. If all players choose the same number, then everyone gets nothing. Otherwise the player with smallest unique number receives two rubles, and other players pay one ruble each. Example: if $1$, $1$ and $3$ are chosen, then the winner is the player who chose $3$. She receives two rubles, and  each of the other two players pays one ruble.
\begin{enumerate}
\item Find all the pure and mixed Nash equilibria
\item State whether the equilibria are Pareto-optimal
\end{enumerate}



\end{enumerate}

\newpage

\section*{Variant 3. Methods of optimal solutions, 23.05.14}

You need to solve all the problems from Sections A, B, C.

\pagestyle{empty}

\textbf{SECTION A}

\begin{enumerate}
\item (15 points) A problem with the mixed constraints is given: $3x_1 x_2-x_2^3 \to \max$

subject to $2x_1+5x_2 \geq 20$, $x_1-2x_2=5$, $x_1, x_2 \geq 0$.
\begin{enumerate}
\item Check NDCQ conditions.
\item  Form the Langrangian function.
\item Solve the maximization problem by the use of Langrange method or otherwise.
\item Justify that the maximum point was found.
\end{enumerate}
\item  (20 points)   Consider the linear programming problem with parameter $\beta$ and nonnegative $x_i$:
\begin{align*}
3x_1-x_2+x_3+4x_4 \to \max \\
x_1-x_2+x_3+x_4\leq 1 \\
\beta x_1+x_2+x_3+2x_4 \leq 2
\end{align*}

\begin{enumerate}
\item Find the optimal values of the primal variables for $\beta=6$.
\item Find the function $\phi(\beta)$, where $\phi(\beta)$  is the maximum value of the objective function for fixed value of $\beta$.
\item Sketch the graph of $\phi(\beta)$.
\end{enumerate}


\end{enumerate}

\textbf{SECTION B}

\begin{enumerate}[resume]
\item (15 points)  Consider the system of difference equation:
\[
\begin{cases}
x_{t+1}=x_t-y_t+4 \\
y_{t+1}=2x_t-y_t+2
\end{cases}
\]

\begin{enumerate}
\item Solve the system of difference equations 
\item Explore the stability of its solutions.
\end{enumerate}

\item   (15 points)   Solve the  Euler’s equation $x^2y''-3xy'+5y=0$ on the interval $(0,\infty)$ by using the substitution $x=e^t$ or otherwise. The answer should be written as a function of $x$.

\end{enumerate}


\textbf{SECTION C}

\begin{enumerate}[resume]
\item  (15 points)   Consider the following bimatrix game:


\begin{tabular}{c|ccc}
 & D & E & F \\ 
\hline 
A & 5;3 & 3;2 & 3;1  \\ 
B & -1;8 & 5;7 & 3;4  \\ 
C & 2;2 & 4;0 & 4;3  \\ 
\end{tabular} 

\begin{enumerate}
\item Find all the pure and mixed Nash equilibria
\item State whether the equilibria are Pareto-optimal
\end{enumerate}

\item  (20 points)  Three players play the following game. Simultenuously each of them chooses one of three numbers: $1$, $2$ and $3$. If all players choose the same number, then everyone gets nothing. Otherwise the player with smallest unique number receives two rubles, and other players pay one ruble each. Example: if $1$, $1$ and $3$ are chosen, then the winner is the player who chose $3$. She receives two rubles, and  each of the other two players pays one ruble.
\begin{enumerate}
\item Find all the pure and mixed Nash equilibria
\item State whether the equilibria are Pareto-optimal
\end{enumerate}



\end{enumerate}

\newpage

\section*{Variant 4. Methods of optimal solutions, 23.05.14}

You need to solve all the problems from Sections A, B, C.

\pagestyle{empty}

\textbf{SECTION A}

\begin{enumerate}
\item (15 points) A problem with the mixed constraints is given: $12x_1 x_2-5x_2^3 \to \max$

subject to $3x_1+2x_2 \geq 6$, $x_1-x_2=1$, $x_1, x_2 \geq 0$.
\begin{enumerate}
\item Check NDCQ conditions.
\item  Form the Langrangian function.
\item Solve the maximization problem by the use of Langrange method or otherwise.
\item Justify that the maximum point was found.
\end{enumerate}
\item  (20 points)   Consider the linear programming problem with parameter $\beta$ and nonnegative $x_i$:
\begin{align*}
4x_1+x_2+3x_3-x_4 \to \max \\
x_1+2x_2+x_3+2x_4\leq 1 \\
\beta x_1+3x_2+2x_3-x_4 \leq 5
\end{align*}

\begin{enumerate}
\item Find the optimal values of the primal variables for $\beta=4$.
\item Find the function $\phi(\beta)$, where $\phi(\beta)$  is the maximum value of the objective function for fixed value of $\beta$.
\item Sketch the graph of $\phi(\beta)$.
\end{enumerate}


\end{enumerate}

\textbf{SECTION B}

\begin{enumerate}[resume]
\item (15 points)  Consider the system of difference equation:
\[
\begin{cases}
x_{t+1}=x_t-y_t+7 \\
y_{t+1}=5x_t-y_t+3
\end{cases}
\]

\begin{enumerate}
\item Solve the system of difference equations 
\item Explore the stability of its solutions.
\end{enumerate}

\item   (15 points)   Solve the  Euler’s equation $x^2y''-6y=0$ on the interval $(0,\infty)$ by using the substitution $x=e^t$ or otherwise. The answer should be written as a function of $x$.

\end{enumerate}


\textbf{SECTION C}

\begin{enumerate}[resume]
\item  (15 points)   Consider the following bimatrix game:


\begin{tabular}{c|ccc}
 & D & E & F \\ 
\hline 
A & 3;3 & 1;2 & 1;1  \\ 
B & -3;8 & 3;7 & 1;4  \\ 
C & 0;3 & 2;2 & 2;3  \\ 
\end{tabular} 

\begin{enumerate}
\item Find all the pure and mixed Nash equilibria
\item State whether the equilibria are Pareto-optimal
\end{enumerate}

\item  (20 points)  Three players play the following game. Simultenuously each of them chooses one of three numbers: $1$, $2$ and $3$. If all players choose the same number, then everyone gets nothing. Otherwise the player with smallest unique number receives two rubles, and other players pay one ruble each. Example: if $1$, $1$ and $3$ are chosen, then the winner is the player who chose $3$. She receives two rubles, and  each of the other two players pays one ruble.
\begin{enumerate}
\item Find all the pure and mixed Nash equilibria
\item State whether the equilibria are Pareto-optimal
\end{enumerate}



\end{enumerate}

\section{Marking scheme }



General comment: arithmetic mistake not seriously affecting the correct path of solution should not lead to withdrawal of more than 20\% of the total score (i.e. either 4 or 2 points).
\begin{enumerate}

\item 2 points for checking NDCQ. Another 4 points for forming correct Lagrangian. 3 points for the proof that the critical point is maximum. 6 points for the solution of the problem (hopefully the majority of students will reduce problem to the one-dimensional).
\item 5 points for conversion to the dual program. 5 points for finding the optimal values of the primal variables. Another 8 points for correct analysis of the feasible region and finding $\phi$. 2 points for the graph. 
\item 3 points for the finding of the partial solution of the system. 7 points for the general solution of the homogeneous system. 5 points for the analysis of stability.
\item 7 points for conversion of the equation to the constant coefficients type. Another 8 points for finding general solution as a function of $x$.
\item 3 points for pure NEs, 3 points for strategy elimination, 6 points for mixed strategy equilibrium, 3 points for checking Pareto-optimality.
\item 6 points for all pure NEs, 2 points for the idea $U(1)=U(2)=U(3)$. Correct expressions for $U(1)$, $U(2)$, $U(3)$ --- 2 points each, solution --- 4 points, 2 points for Pareto-optimality.

All pure NEs. The number 1 should be chosen, otherwise any loser  will deviate. By brute force: $(1,2,3)$, $(1,3,3)$, $(1,2,2)$ and transpositions.

Symmetric mixed NE. First player should be indifferent between numbers, so:

\[
\begin{cases}
p_1+p_2+p_3=1 \\
U(1)=U(3) \\
U(2)=U(3) 
\end{cases}
\]

Here 
\begin{align*}
U(1)=2\cdot (1-p_1)^2+0\cdot p_1^2-1\cdot [1-(1-p_1)^2-p_1^2] \\
U(2)=2\cdot (p_1^2+p_3^2)+0\cdot p_2^2-1\cdot [1-(p_1^2+p_3^2)-p_2^2] \\
U(3)=2\cdot (p_1^2+p_2^2)+0\cdot p_3^2-1\cdot [1-(p_1^2+p_2^2)-p_3^2]
\end{align*}

Solving $U(2)=U(3)$ we obtain $p_2=p_3$. Finally we obtain $p_1=0.5$, $p_2=p_3=0.25$.

This is zero-sum game, so all profiles are Pareto-optimal.


\end{enumerate}


\end{document}