% !TEX root = ../matmor_exams.tex

\subsection{MFE, October mock, 2019-10-25}

The october exam was shorter than usual as it fell on the same day with linear algebra exam. 

\begin{enumerate}
    \item (15 points) Consider the function $f(x, y) = x^3 -3 y^3 + 2xy$.
    Using the total differential find the approximate value of $f(1.98, 0.99)$.
  

    \item (15 points) Consider the system
    \[
    \begin{cases}
    3x^3 + y^3 + z^2 = 5 \\
    x + x^3 + 2y^3x = 4 \\
    \end{cases}
    \]
  
    \begin{enumerate}
      \item Check whether the functions $z(y)$ and $x(y)$ are defined at a point $(1, 1, 1)$;
      \item Find $z'(y)$ if possible.
   \end{enumerate}
  

   \item (15 points) Consider the function $h(b) = f(f(f(b \cdot f(b))))$.
   Find $dh/db$ for $b=1$ if it is known that $f(1)=2$, $f(2)=3$, $f(3)=1$, 
   $f'(1)=3$, $f'(3)=2$, $f'(2)=1$.
  

  \item (15 points) Consider the function $f (x, y) = xyz^3$, the vector $v = (1, 2)$ and the point $A = (−1, −1)$.
  
  \begin{enumerate}
    \item  Find the gradient of $f$ at the point $A$.
  \item  Find the directional derivative of $f$ at the point $A$ in the direction given by $v$.
  \end{enumerate}
  
  \item (15 points) Provide an explicit example of a sequence in $\RR^2$ that is unbounded and has 
  exactly two accumulation points. 
  
  
  \item Two identical firms compete in a labor market with the supply function $w(L)= w_0 + aL$, 
  where $w_0>0$, $a>0$ and $L$ is the labor amount supplied at the wage rate $w$.
  
  In order to find equilibrium one has to solve the system of equations
  \[
  \begin{cases}
    f(L_1) - ME_1 = 0 \\
    f(L_2) - ME_2 = 0
  \end{cases},
  \]
  where $f'(L)<0$ for all $L>0$ 
  and $ME_1$, $ME_2$ are marginal expenses which are found by differentiation, 
  $ME_i = \partial (w(L)L_i)/\partial L_i$ for $i \in \{1, 2\}$ and $L = L_1 + L_2$. 
  
  Suppose the equilibrium exists.
  \begin{enumerate}
    \item (10 points) Prove that $L_1^*=L_2^*$.
    \item (15 points) Find $\partial L_1^*/\partial w_0$.
  \end{enumerate}
  
  \end{enumerate}
  

\subsection{MFE, October mock, 2019-10-25, marking}


\begin{enumerate}
  \item 1 - value of function at good point,
  2 - value of der-ve wrt x,
  2 - value of der-ve wrt y,
  4 - 1st Taylor series approximation,
  1 - final result
  
  \item 
  a) 1pt - functions are continuous, 3pts - Jacobean matrix, 1pt - point satisfies the system of eq-ns
  
  
  b) 2pts - Formula, 3pts - Final result
  
  \item 3pts chain rule, 2pts product rule, 2 pts each of 4 multiples, 2pts answer
  
  \item
  a) 2pts for partial derivatives, 2pts for derivatives in a point, 3 pts correct gradient
  
  b) 1pt correct formula, 3pts correct grad and direction vector, 2 pts normalization, 2 pts answer
  
  \item  any sequence in $R^2$ = +3 pts,
  unbounded = +6 pts,
  with two accumulation points = +6 pts

  \item
  \begin{enumerate}
\item correct $\partial w(L)L_i /\partial L_i$ = 4 pts, rest of proof = 6 pts
 \item   ratio of determinants formula = 5 pts, nominator = 5 pts, denominator = 5 pts
  
\end{enumerate}
  

\end{enumerate}



\subsection{MFE, midterm, 2019-12-28}

\begin{enumerate}
  \item (10 points) Find the limit or prove that it does not exist
  \[
  \lim_{x_1, x_2, x_3 \to 0} \frac{x_1^2 + 4x_1 x_2 - x_2^2 + 6 x_3x_2}{x_1^2 + x_2^2 +x_3^2}
  \]

  \item (10 points) Using Lagrange multipliers find the extrema of the function $f(x,y) = x^2 + 2 y^2$ subject to $x^2 + y^2 = 16$.
Check sufficiency conditions.

 \item (10 points) Consider the function $u(x,y) = x^2 + x^4 + axy + y^2$. 
 For which values of $a$ the function $u$ is convex? concave?

 \item (10 points) Using Envelope theorem find the approximate minimum of the function
 \[
 f(x, y) = x^4 + 0.001 (x^2 + y + y^2) + (y-1)^4.
 \]

  \item (10 points) Find and classify all the local extrema of the function $f = x^2 - 4xy +y^3 +4y$.


\item (10 points) Using optimization techniques prove for $x> 0$, $y>0$ the inequality 
\[
\frac{x + y}{2} \geq \frac{2}{x^{-1} + y^{-1}}  
\]

%  \item (10 points) The fully connected artificial neural network layer with ReLU activation may be represented as
%  a function from $\mathbb{R}^n$ to $\mathbb{R}^n$, $y = \sigma(Wx + b)$, 
%  where $W$ is a $n\times n$ constant matrix, $b$ is a constant vector. 
%  And scalar function $\sigma(t) = \max(t, 0)$ 
%  is applied element by element to the argument vector. 
  
%  Calculate maximal and minimal value of $dy_{2020}/dx_{2019}$ if $w_{i,j} = i+j$.



\item Consider a problem $f(x_1, x_2, \alpha) \to \max$ subject to $g(x_1, x_2) = 0$. 
The function $f$ is maximized with respect to $x_1$, $x_2$ and $\alpha$ is a real parameter. 
Both functions $f$ and $g$ are twice continuous differentiable. 
Let $(x_1^*, x_2^*)$ be a solution to this problem depending on $\alpha$ 
and $\phi(\alpha)$ be the value function of this problem. 

\begin{enumerate}
  \item (5 points) Formulate the envelope theorem that provides the value of $d\phi/da$.
  \item  (5 points) Introduce the function $F=f(x_1, x_2, \alpha) - \phi(\alpha)$. 
  Clearly state the second-order sufficiency condition applicable to $F$ 
  that guarantees the optimality of $(x_1^*, x_2^*)$. 
  \item (10 points) The SOC condition stated in b) should justify inequality
  $\frac{\partial^2 f}{\partial x_1 \partial \alpha} \frac{dx_1^*}{d\alpha} +
  \frac{\partial^2 f}{\partial x_2 \partial \alpha} \frac{dx_2^*}{d\alpha} > 0$. 
  Show this. 
\end{enumerate}


\item Consider a function 
\[
  f(x) = \frac{\sum_{i=1}^n \sum_{j=1}^n a_{ij}x_i x_j}
{\sum_{i=1}^n x_i^2}
\]
in $\mathbb{R}^n$ defined everywhere except at the origin. 
Here $a_{ij}$ are entries of the symmetric matrix $A$. 
\begin{enumerate}
\item (5 points) Show that $A=cI$ implies that $f(x)=c$. 
Here $c$ is a real number and $I$ is identity matrix.
\item (5 points) Let $A$ be a matrix other than in a). 
Show that then $f$ is discontinuous at the origin with the irremovable discontinuity. 
\textit{Hint: you may show this using part c) or otherwise}.
\item (10 points) In order to find the points of extremum of $f$ 
the optimal problem is reformulated as follows:
\[
\begin{cases}
  \sum_{i=1}^n \sum_{j=1}^n a_{ij}x_i x_j \to \max / \min \\
  \sum_{i=1}^n x_i^2 = 1
\end{cases}
\] 
Solve it by Lagrangian and find the maximum and minimum values of $f$ 
in terms of eigenvalues of matrix $A$.
\end{enumerate}


\end{enumerate}



\subsection{MFE, midterm, 2019-12-28, marking}

\begin{enumerate}
  \item Take two different directions, for instance, $(x,0,0)$ and $(0,y,0)$.
  
  Show that the limits of function along these directions are different. (3 pts for each direction)
  
  Conclude that the initial limit doesn't exist (reasoning - 3 pts, correct answer - 1 pt)
  
  OR
  
  Take the direction that depends on k and m (x, kx, mx). 
  
  Show that the limit turns out to be the function of two variables k and m (6 pts)
  
  Show that this expression of k and m possesses different values (not a constant) (3 pts)
  
  Write the answer that Limit doesn't exist (1 pt)
  
  \item
  I. Critical points (5 pts total):
  
  1 pt for proof that NDCQ holds without any exceptions
  
  1 pt for the Lagrangian function
  
  1 pt for the system of FOC
  
  2 pts for all 4 pts: (0; 4) (0; -4), (4; 0), (-4;0)
  
  II. Local extrema (3 pts total)
  
  1 pt for bordered Hess matrix
  
  1 pt for proof of existence of local minimum(-a)
  
  1 pt for proof of existence of local maximum(-a)
  
  III. Global exprema (2 pts total)
  
  1 pt for the mention of compactness of the constraint set+ mention of Weierstrass theorem
  
  1pt for finding the global extrema

  \item Hesse matrix - 2 points
  
  Leading principle minors - 1 point
  
  Criteria for concavity/convexity + Sylvestr criterion - 2 points
  
  Proof that function is never concave - 2 points
  
  Condition for convexity - 3 points
  
  
  \item 
  Case of $a=0:$ FOC + SOC (or different proof of min) - 3 points
  
  Envelope theorem and its application here - 3 points 
  
  Linear approximation and its application here - 3 points
  
  Correct answer - 1 points

  \item FOC - 2 points, solution - 2 points, Hesse matrix - 2 points, classification - 4 points (2 points for each critical point)

  \item
  Using optimization techniques:
  Objective function - 2 points, FOC - 2 points, solution - 2 points, SOC and concavity analysis - 4 points
  
  Other method:
  
  2 points for some correct simplification of the inequality, 
  
  4 points for considering the case of equality, 
  
  6 points if simplifications result in inequation like this: $x^2+y^2 \geq 2xy$, 
  
  8 points if it results in $(x-y)^2>=0$, 
  
  10 points if it is explained that all x and y are solutions of this inequality.
  
  -1 point if in solution there’s no mention that all values are positive so that one can multiply/divide the whole inequality by numerator/denominator of the right/left part

  \item a) correct derivative without point — 3 points
  b) 5 points
  c) 5 points for $\partial F/\partial \alpha$, 5 points for $\partial^2 F/\partial \alpha^2$

  \item a) correct identity matrix — 1 point

  b) only particular case of matrix $A$ considered — 3 points

  c) NDCQ — 1 point, Lagrange function — 1 point, FOC with correct derivatives — 2 points,
  interpretation of FOC with eigenvalues — 3 points, SOC — 3 points.

\end{enumerate}



\subsection{More problems\ldots}

These problems are from exam drafts and may be used for practice. 

\begin{enumerate}
\item A production function $y=f(x_1, x_2)$ exhibits constant returns to scale, 
that is $f(tx_1,tx_2)=tf(x_1,x_2)$ for every $t>0$, where $x_1, x_2 \geq 0$.
\begin{enumerate}
\item Let $f_1(x_1,x_2)=\frac{\partial f}{\partial x_1}$ and $f_2(x_1,x_2)=\frac{\partial f}{\partial x_2}$. 
Using the chain rule show that both partial derivatives of $f$ have the property $f_i(tx_1,tx_2)=f_i(x_1,x_2)$ for $i=1,2$.
\item Let $MRTS=f_1/f_2$. Show that its value remains the same along a straight line $x_2=\lambda x_1$,
 where $\lambda$ is a given positive number and $x_1, x_2>0$.
\end{enumerate}

\item Consider the function $f(x,y,z)=2x^5+2xyz-z^3$. 
Using the total differential find the approximate value of $f(1.02,0.99,1)$.

\item Consider the function $u(x,y)=4f(r)$ where $r=\sqrt{x^2+y^2}$. 
Is it possible to represent the function $x\frac{\partial u}{\partial x}+y\frac{\partial u}{\partial y}$ as a function of $r$ alone, i.e. $g(r)$? If yes, then find $g(r)$.

\item Given the system
\begin{equation} \nonumber
\begin{cases}
u^2-w^2+x^2+y^2=0 \\
uw+xy=0
\end{cases}
\end{equation}
\begin{enumerate}
\item Define a sufficient condition for functions $u(x,y)$ and $w(x,y)$ to be differentiable.
\item Find $\frac{\partial w}{\partial x}$.
\end{enumerate}

\end{enumerate}

