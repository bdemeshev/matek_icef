% !TEX root = ../matmor_exams.tex

\subsection{MFE, december exam, 2022-12-27}



\begin{enumerate}
    \item (10 points) Consider the sets $B_n=\left[ \frac{1}{3n},\frac{2n-1}{n}\right]\subset \mathbb{R}$. 
    Let $A=\bigcup \limits_{n=1}^{\infty}B_n$.  
      \begin{enumerate}
          \item Is $A$ bounded? open? closed? compact?
          \item Sketch $A\times A$. Is it open? closed?
      \end{enumerate} 


  \item (10 points) Find the limit or prove it doesn't exist 
  \[
  \lim_{x\to +\infty, y\to +\infty}x^2y^5\sin {\frac{1}{x^5y^2}}.
  \]

  \item (10 points) Using the chain rule calculate all partial derivatives of the first and second order of $g(\alpha,\beta)=f(x(\alpha,\beta),y(\alpha,\beta))$, 
  if $x=2\alpha-\beta^2, y=\alpha \beta$ and $f(x,y)$ is twice continuously differentiable. 



  \item (10 points) Two dimensional function is given by $f(x,y)=x^3-2x^2y$.
      \begin{enumerate} 
          \item Find the direction of maximal growth of $f(x,y)$ from the point $A(1,1)$.
          \item Find at least one direction from the point $A$ in which the function doesn't increase and doesn't decrease (doesn't change).
      \end{enumerate} 



  \item (10 points) Find and classify all the critical points of  $f(x,y)=(y^2-2xy+x)e^x.$



  \item (10 points) Using Lagrange multiplier method find and classify the constrained extrema of $f(x,y,z)=3x-y+2z$ subject to $4x^2+4y^2+z^2=13$.
\end{enumerate}



\begin{enumerate}[resume]
  \item (20 points) Let $h: \mathbb{R} \to \mathbb{R}$ be a $\mathbb{C}^1$ function and $h(0)=0$. Consider the system
\[
  \begin{cases}
    e^x+h(y)=u^2\\
    e^y-h(x)=v^2
  \end{cases}   
\]
          \begin{enumerate} 
              \item Prove that in a neighbourhood of $(x,y,u,v)=(0,0,1,1)$ we can define $x(u,v)$ and $y(u,v)$ as functions of $u$ and $v$.
              \item If it is also known that $h'(0)=1$ calculate $\frac{\partial ^2 x}{\partial v^2}$ at the point $(0,0,1,1)$.
          \end{enumerate}
\end{enumerate}

\begin{enumerate}[resume]
      \item  In a two-product economy the price-taking industry bears total costs $C(x,y)=c_1(x)+c_2(y)$, 
      where $x$  and $y$  are the outputs and $C_1$ and $C_2$ are the functions twice continuously differentiable with the positive first and second order derivatives. 
      Since only one factor of production is employed, the outputs are related $2x+3y=L$, 
      where $L$  is the total endowment of labor.
          \begin{enumerate}
              \item (5 points) Set the minimization problem for $C(x,y)=c_1(x)+c_2(y)$, 
              subject to constraint and write down first-order conditions applied to the Lagrangian of the problem $K(x,y,\lambda)$.
             \item (5 points) By checking the bordered Hessian determinant, 
             prove that the only critical point of $K(x,y,\lambda)$ is a minimizer.
              \item(3 points) Let this point be $(x^*,y^*,\lambda^*)$. 
              Find $\frac{dC(x^*,y^*)}{dL}$. Use one of the envelope theorems.
              \item(7 points) Industry economists have found out that $c_1''(x^*)=c_2''(y^*)$. 
              Use this information to calculate the values of $\frac{dx^*}{dL}$ and $\frac{dy^*}{dL}$.
          \end{enumerate}
  \end{enumerate}
  

\subsection{MFE, december exam, 2021-12-25, Marking scheme}
\begin{enumerate}

  \item 
      \begin{enumerate}

          \item Understanding which set is represented by $A$ – an open interval – 1 point.
   $A$ is open – 2 points.
   $A$   is not closed, bounded $\Rightarrow$ not compact – 2 points.
          \item Understanding which set is represented by  $A\times A$  – 1 point.
     $A\times A$  - open – 2 points, not closed – 2 points.  Only answers – 0 points.
      \end{enumerate}
  \item Consideration of different directions without solution – 2 points. 
  Finding incorrect limit for one of directions – 2 points. 
  Finding correct limit for one of directions – 5 points. 
  Arithmetic mistake – minus 1-2 points.

  \item Correct $\frac{\partial x}{\partial \alpha},\frac{\partial y}{\partial  \alpha},\frac{\partial x}{\partial \beta},\frac{\partial y}{\partial \beta}$ - 2 points. 
  Correct chain rule for $\frac{\partial g}{\partial \alpha}, \frac{\partial g}{\partial \beta}$ - 2 points. 
  Each second order derivative - 2 points.
  \item 
      \begin{enumerate} 
          \item Direction of maximal growth is a gradient vector – 1 point, correct partial derivatives of $f(x,y)$ – 2 points, correct values at the point $A$ – 1 point.
          \item Definition of directional derivative - 2 points, it must be equal to 0 – 2 points, final answer – 2 points.
      \end{enumerate}
  \item Correctly formulating and solving for FOC – 3 points. 
  Checking extremum via Hessian – 7 points. Partial or incorrect solution – 1-2 points. 
  Numerical errors in an overall correct solution – (1-3) points subtracted. 
  \item Correctly formulating Lagrangian and solving for FOC – 3 points.
Checking extremum via (bordered) Hessian or correctly applying Weirstrass theorem – 7 points. 
Partial or incorrect solution – 1-2 points.
Numerical errors in an overall correct solution – (1-3) points subtracted.
Graphical solution – up to 5 points, depending on detalization, if the Lagrangian is formulated.

  \item 
      \begin{enumerate} 
          \item
Continuity assumption - 4 points,
point check - 4 points,
Jacobian condition - 4 points.
          \item
Correct $\frac{\partial x}{\partial v}$  - 2 points,
$\frac{\partial ^2 x}{\partial v^2}$ (formula only) - 3 points,
correct result - 3 points
      \end{enumerate}
  \item 
      \begin{enumerate} 
              \item
FOC - 6 points
\item
Hessian matrix (formula only) - 2 points, correct result - 2 points
\item
correct result - 3 points
\item
correct result - 6 points (each derivative - 3 points)
      \end{enumerate}


\end{enumerate}





\subsection{MFE, final, 2023-05-17}
\begin{enumerate}
%\section{Variant $\Xi$}
\item Let $y=F(L,K)$ be a continuously differentiable function defined for $L,K>0$.
    \begin{enumerate}
        \item (5 points) What does it mean to say that $y=F(L,K)$ is a homogeneous function and provide rigorous definition.
        \item (5 points) Let  $\frac{\partial F}{\partial n}$ be a directional derivative of $F(L,K)$  in the direction  $n$, where $n$  is a unit vector. Show that  $\frac{\partial F}{\partial n}$   is also a homogeneous function.
    \end{enumerate}


\item (10 points) There is a square on the complex plane. 
Four complex numbers form the four vertices of this square. 
Three of them are:  $-19+32i$,  $-5+12i$,  $-22+15i$. Find the fourth complex number.


\item (10 points) Solve the first order differential equation $xy'+2y=xy^4$.


\item (10 points) Consider difference equation $y_{t+3}-7y_{t+1}-6y_{t}=\alpha (-1)^t+\beta \cdot 2^t$, where $\alpha$ and $\beta$ are real numbers.
\begin{enumerate}
    \item For which values of $\alpha$ and $\beta$ there is NO resonance?
    \item Find the general solution of the difference equation and express it it terms of $\alpha$ and $\beta$ for the values found in part a).
\end{enumerate}


\item (10 points) Solve the following linear programming problem (find both the minimal value of the objective function and the coordinates):
  \[
  \begin{cases}
  2x_1 + 3x_2 + 4x_3 \to \min \\
  x_1 \geq 0, x_2 \geq 0, x_3 \geq 0 \\
  x_1 + x_2 + 4x_3 \geq 2 \\
  x_1 + x_2 + x_3 \geq 1 \\
  \end{cases}.
  \]
  


\item (10 points) Find all pure and mixed Nash equilibria in the following game

\begin{center}
\begin{tabular}{@{}cccc@{}}
\toprule
  & d & e & f \\ \midrule
a & (6, 6) & (2, 6)  & (3, 2)   \\
b & (2, 1) & (5, 4) & (3, 2)   \\
c & (3, 4) & (3, 2) & (2, 3)   \\ \bottomrule
\end{tabular}
\end{center}

\end{enumerate}    



\begin{enumerate}[resume]

\item In a closed economy the production possibilities frontier is given by the system of inequalities in  $xy$--plane where  $x$ and $y$ represent the produced goods: 

$PPF=\{ x\geq 0,0\leq y\leq 4+2x-x^2,y\leq 4\}$.
\begin{enumerate}
    \item (7 points) Prove analytically that  $PPF$ is a convex set. 
    \item (10 points) Representative consumer has the utility function  $U(x,y)=\sqrt{6x}+y$. Maximize its value on the  $PPF$.
    \item (3 points) Use appropriate sufficiency condition justifying your solution in b).
\end{enumerate}


\item (20 points) Solve the system of differential equations by using the substitution method 

\[
  \begin{cases} 
    \dot{x}= 6x -3y -3e^{-t}, \\ 
    \dot{y}= 15x -6y-e^t.
  \end{cases}
\]


\end{enumerate}

\subsection{MFE, final, 2023-05-17, marking}
\begin{enumerate}

    \item 
    \begin{enumerate}
      \item  correct definition – 5 points, without domain/valid values of  $t$  minus 1-2 points;
      \item only facts about derivatives of homogeneous functions without proof -  minus 1 point, correct proof – 5 points.
    \end{enumerate}

    \item Let $A(-19,32), B(-22,15), C(-5,12)$. Point $D=C+\overrightarrow{BA}=(-5,12)+(3,17)=(-2,29).$
    
    Thus, the 4th point is: $-2+29i$


Any correct solution — 10 points,

Only image of points on the complex plane — 1 point,

Solution `with distances' without answer — 5 points

    \item $y=0$ is a particular solution — 1 point
    
    divide the whole equation by $y^4\Rightarrow x\frac{y'}{y^4}+\frac{2}{y^3}=x$

    Substitution: $z=\frac{1}{y^3},z'=-\frac{3y'}{y^4}$  — 3 points

    Then, $xz'-6z=-3x$ — 1 point

    Solve homogeneous equation first: $xz'-6z=0\Rightarrow z=C\cdot x^6$ — 2 points

    Variation of the constant: $z=C(x)\cdot x^6\Rightarrow z'=6x^5C+x^6C'\Rightarrow C'=-3x^{-6}\Rightarrow C(x)=\frac{0.6}{x^5}+A$

    Thus, $z(x)=Ax^6+0.6x$ — 2 points

    Final answer: $y=0$ and $\frac{1}{y^3}=Ax^6+0.6x$ — 1 point

    \item Characteristic equation $\lambda^3-7\lambda-6=0$ — 1 point

    The roots are $\lambda_1=-1, \lambda_2=-2, \lambda_3=3$ — 3 points

    $y_t^{hom}=C_1(-1)^t+C_2(-2)^t+C_3\cdot 3^t$ — 1 point

    There will be no resonance if $\alpha = 0, \beta \in \RR$ — 2 points

    $y_t^{part}=A\cdot 2^t\Rightarrow 8A-14A-6A=\beta\Rightarrow A=-\frac{\beta}{12}$ — 2 points

    $y_t^{general}=C_1(-1)^t+C_2(-2)^t+C_3\cdot 3^t-\frac{\beta}{12}$ — 1 point

    \item Dual problem — 2 points: 
    \[
          \begin{cases}
          2y_1+y_2 \to \max \\
          y_1+y_2 \leq 2 \\
          y_1+y_2\leq 3 \\
          4y_1+y_2 \leq 4 \\
          y_1,y_2\geq 0
          \end{cases}.
    \]

    Sketch — 3 points
    
    $(y_1^{*},y_2^{*})=\left( \frac{2}{3},\frac{4}{3}\right)$ — 1 point

    $\max(2y_1+y_2)=\min(2x_1 + 3x_2 + 4x_3)=\frac{8}{3}$ — 1 point

    Second dual constraint is passive thus $x_2^{*}=0$

    $y_1^{*}$, and $y_2^{*}$ are positive thus  both initial constraints are active
    
    \[
    \begin{cases}
        x_1  + 4x_3 = 2, \\
        x_1 +  x_3 = 1 
    \end{cases}
\]
    $(x_1^{*},x_2^{*},x_3^{*})=\left( \frac{2}{3},0,\frac{1}{3}\right)$ — 3 points

    \item Pure equilibria are: $(a,d),(b,e)$ — 2 points

    Mixed strategies. As the game is finite  never best response strategies are dominated. 
    Column $f$ is never best response for player 2 and $c$ is never best response for player 1.
    
    After eliminating these strategies:

    $E_1=6pq+2q(1-p)+2(1-q)p+5(1-q)(1-p)$ and $dE_1/dp=7q-3$

    Warning! The function $E_1$ is linear in $p$! Do not simply solve $dE_1/dp=0$! 
    We have corner solution here!

    \[
      p^*=\begin{cases}
        0, q<\frac{3}{7} \\
        \left[ \frac{3}{7},1 \right], q=\frac{3}{7} \\
        1, q>\frac{3}{7}
    \end{cases}
    \]

    
    $E_2=6pq+6q(1-p)+(1-q)p+4(1-q)(1-p)$ and $dE_2/dq=3p-3$

    Warning! The function $E_2$ is linear in $q$! Do not simply solve $dE_2/dq=0$! 
    We have corner solution here!

    \[
      q^*=\begin{cases}
        0, p<1 \\
         \left[ \frac{3}{7},1 \right], p=1
    \end{cases}
    \]

    Equilibria in mixed strategies: $(a,q\cdot d+(1-q)\cdot e)$, where $q\in\left[ \frac{3}{7}, 1\right]$.
    
    One mixed NE is found — 3 points. Infinity of mixed NE — 3 points.

    \item 
    \begin{enumerate}
      \item 5 points for any attempt, 2 points — function concavity is applied and stated;
      \item 2 points — Lagrangian, 4 points — FOC, 4 points — point identified;
      \item 3 points correct justification.
    \end{enumerate}
    
    \item  ODE is correctly structured — 5 points. Roots are found — 10 points.
    General solution for one function  — 2 points.
    General solution for both functions — 3 points.
  
\end{enumerate}