\subsection{MFE, fall semester exam, 19.01.2010}

Lecturer: K. A. Bukin, Classteachers: A. Arlashin, I. Kachkovski, M. Martinovic


Marks will be deducted for insufficient explanation within your answers. Total duration
of the exam is 120 min.


Variant I


SECTION A. Answer all six questions from this section (10 marks each)

\begin{enumerate}

\item Consider the function $f(x,y)=(1+x)/y^2$. On the $(x,y)$-plane, sketch several level curves of $f$. Calculate and sketch unit length vectors indicating directions of the most rapid growth of $f$ at points $(1, 1)$ and $(3, -1)$.

\item Use the first-order differential to approximate
$\sqrt[3]{4\cdot 0.9^2+2.2^2}$.

\item Find the equation of the tangent plane to the surface given by $x^3+z^3-3xz=y-1$ at the point $(1, 4, 2)$.

\item Find $\partial^2 u / \partial x \partial y$ and $\partial^3 u / \partial x \partial z^2$ if $u=f(s,t)$ and $s=x/y$, $t=y/z$. Assume that $f$ has continuous third-order partial derivatives.

\item Given the system
\[
\begin{cases}
xe^{u+v}+2uv=1 \\
ye^{u-v}-\frac{u}{1+v}=2x
\end{cases}
\]
find $du$ and $dv$ at $x_0 = 1$, $y_0 = 2$, $u_0 = 0$, $v_0 = 0$.

\item Use Lagrange multipliers to find the dimension of a rectangular box with the least
possible surface area among those with a volume of 27 m$^3$. Make sure you check the second
order condition for minimisation.
\end{enumerate}

SECTION B. Answer both questions from this section (20 marks each)

\begin{enumerate}[resume]
\item Dr. Cooper is a two-period consumer and his utility is $u(c_0 , c_1) = \ln c_0 + 0.8 \ln c_1$, with $c_0$ the initial period consumption and $c_1$ the second period consumption. He gets income of $Y_0$ and $Y_1$ in these two periods respectively, and can borrow and lend money at the same interest rate of $r$.


That means, if Dr. Cooper consumes less than $Y_0$ in the initial period, then the difference is saved and results in the second period additional income, equal to $(Y_0 - c_0)(1 + r)$, of course. If he decides to consume more than $Y_0$ in the initial period, he has to borrow and repay the borrowed amount together with interest payments in the second period, thus lowering his second-period consumption (to what amount?).
\begin{enumerate}
\item Show that irrespective of whether Dr. Cooper goes borrowing or lending, his budget
constraint is
\[
c_0 (1 + r) + c_1 = Y_0 (1 + r) + Y_1
\]
\item Sketch the budget constraint and several indifference curves on the $(c_0 , c_1)$-plane.
\item Suppose now that $r = 0.4$ and $Y_0 = Y_1 = 10$. Set up the utility optimisation problem with variables $c_0$ and $c_1$. Use Lagrange multipliers to solve it. Justify that the point you have found is the point of maximum.
\item Using the Envelope Theorem, approximate the change in Dr. Cooper’s maximum
utility if the interest rate rises to $0.45$ with unchanged income in both periods.
\end{enumerate}

\item Consider the Edgeworth exchange economy with two agents, Robinson and Friday,
and two goods, wheat and fish. Robinson has the utility of possessing $w$ units of wheat
and $f$ units of fish equal to $u_R(w, f) = wf$, Friday’s utility is $u_F(w, f) = w^2 f$. There is one unit of wheat and one unit of fish in the economy.
\begin{enumerate}
\item Set up (but do not solve) the problem of optimisation of Friday’s utility given the
fixed level of Robinson’s utility, say, $u_R(w, f ) = 1/3$. There should be 4 variables (the product levels, or allocations, for Robinson and Friday) and 3 equality constraints.
\item The problem above can also be set up in 2 variables with 1 constraint: maximize
$(1-w)^2(1-f)$ subject to $wf = 1/3$. Use Lagrange multipliers to solve it. Find the optimal product levels of both agents. Do not check the second-order condition.
\item In the context, explain the geometric significance of the equation
\[
\nabla u_R (w, f ) = \lambda\cdot \nabla u_F (1 - w, 1 - f )
\]
where $\lambda$ is some real number.
\item The optimal levels in (ii) are also the solution of the problem of optimising Robinson’s utility given some fixed level of Friday’s utility, $u_F(w,f) = a$. What is the value of $a$? What is the optimal value of Robinson’s utility under the given constraint?
\end{enumerate}


\end{enumerate}


\subsection{MFE, fall semester exam, solution, 19.01.2010}

\begin{enumerate}
\item
\item
\item
\item
\[
\frac{\partial u}{\partial x} = \frac{\partial f}{\partial s} \frac{1}{y}
\]

\[
\frac{\partial^2 u}{\partial x\partial y} = \left(\frac{\partial^2 f}{\partial s^2} \frac{-x}{y^2}+ \frac{\partial^2 f}{\partial s\partial t}\frac{1}{z}\right)\frac{1}{y} - \frac{1}{y^2}  \frac{\partial f}{\partial s}
\]

\[
\frac{\partial^3 u}{\partial x\partial z^2} = \frac{1}{z^4} \frac{\partial^3 f}{\partial s\partial t^2} + \frac{2}{z^3} \frac{\partial^2 f}{\partial s\partial t}
\]

\end{enumerate}

