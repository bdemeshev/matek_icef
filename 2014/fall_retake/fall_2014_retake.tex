\documentclass[12pt,a4paper]{article}
\usepackage[utf8]{inputenc}
\usepackage[english]{babel}
\usepackage{amsmath}
\usepackage{amsfonts}
\usepackage{amssymb}
\usepackage{enumitem}
\usepackage[left=2cm,right=2cm,top=2cm,bottom=2cm]{geometry}
\DeclareMathOperator{\grad}{grad}
\begin{document}
\thispagestyle{empty}
\textbf{Retake 24.01.2015. Good luck!}

\textbf{Part A}

\begin{enumerate}

\item The function $f(x,y)$ is given by $f(x,y)=u^2(x,y)+v^3(x,y)$. The values of $u$ and $v$ and their gradients at the point $(x,y)=(1,1)$ are also known, $u(1,1)=3$, $v(1,1)=-2$, $\grad u=(1,4)$, $\grad v=(-1,1)$. Find $\grad f$ at the point $(1,1)$ if $u,v \in C^1$.

\item Find the local maxima and minima of the function $f(x,y)=x^4+2y^4-xy$. Determine whether the extrema you have found are global or local.

\item For the function $f(x,y)=x^3y^5+x^2-y^3+xy$ find first order Taylor approximation at the point $(1,1)$ and second order Taylor approximation at the same point.

\item Use Lagrange multipliers to find the height and radius of a cylinder with the least possible
surface area among those with a volume of $6\pi$ m$^3$. Make sure you check the second order
condition for minimisation.

\item Find the equation of the tangent plane to the surface given by $x^3+z^3-3xz=y-1$ at the point $(1, 4, 2)$.

\item Suppose $f(x,y)$ is a twice differentiable function. Let $x$ and $y$ be defined in terms of $u$, $v$ as follows: $x(u,v)=ue^{2v}$, $y(u,v)=u^2-v^2$. Let $F(u,v)=f(x(u, v), y(u, v))$.
Calculate $F_{uu}''$ and $F_{uv}''$.

\end{enumerate}

\textbf{Part B}


\begin{enumerate}[resume]
\item A firm’s inventory $I(t)$ is depleted at a constant rate per unit time, i.e. $I(t) = x-\delta t$, where $x$ is an amount of good reordered by the firm whenever the level of inventory is
zero. The order is fulfilled immediately. The annual requirement for the commodity is
$200$ units and the firm orders the commodity $n$ times a year where $200 = nx$. The firm
incurs two types of inventory costs: a holding cost and an ordering cost. Since the average
stock of inventory is $x/2$, the holding cost equals $C_h x/2$, the cost of placing one order is
$C_o$, and with $n$ orders a year the annual ordering cost equals $C_o n$.
\begin{enumerate}
\item Minimize the cost of inventory $C = C_h x/2 + C_o n$ by choice of $x$ and $n$ subject to
the constraint $nx = 200$ by the Lagrange multiplier method.
\item Use the envelope theorem to approximate the change in the minimal cost if the
requirement for the commodity rises to $204$ units.
\end{enumerate}


\item A two-product firm produces outputs $y_1$ and $y_2$ from a single factor of production which is labor, in other words, there is a function $f$, such that $f(y_1,y_2)\leq \bar{L}$. Output prices are $p_1$ and $p_2$. The firm has a fixed amount of labor supply $\bar{L}>0$ that should be utilized in full.
\begin{enumerate}
\item (10 points) Set the problem of the revenue maximization under the labor constraint mathematically and derive first-order conditions. Assume that both outputs should be produced in positive amounts.
\item (10 points) Let the maximum value of the total revenue under the labor constraint be $TR(p_1,p_2,\bar{L})$. What are its derivatives with respect to the prices?
\end{enumerate}


\end{enumerate}

\begin{enumerate}
\item $\grad f =2u \grad u + 3v^2 \grad v = 6\cdot (1,4)+12\cdot (-1,1)=(-6,36)$. Maybe solved by computing $f'_x$ (formula + value, 3+1 pts) and $f'_y$ (3+1 pts) and putting them in a vector (2 pts). 
\item FOC statement --- 1 pt, FOC solution --- 4 pts, SOC --- 3 pts, globality --- 2 pts.

Critical points: $(x,y)=(0,0)$, $(2^{-9/8},2^{-11/8})$, $(-2^{-9/8},-2^{-11/8})$

\item 1 pt for each derivative (5 pts total $f_x$, $f_y$, $f_{xx}$, $f_{yy}$, $f_{xy}$), 2 pts for first order, 3 pts for second order approximation

\item problem formulation (target function, constraint) --- 2 pts, NDCQ --- 1 pt, FOC formulation --- 1 pt, FOC solution 4, SOC check --- 2 pts

\item $\grad f = (3x^2-3z,-1,3z^2-3x)=(-3,-1,9)$ (4 pts). So tangent plane equation is $-3x-y+9z=c_0$ (4 pts). Plugging in the coordinates of the point we obtain $c_0=11$ (2 pts).

\item $F_u$ 4 pts, $F_{uu}$ --- 3 pts, $F_{uv}$ --- 3pts

\item a) NDCQ --- 2 pts, Langrangean --- 1 pt,  FOC statement --- 1 pt, FOC solution --- 5 pts, SOC --- 5 pts, minimum value --- 2 pts, b) 4 pts

\item formulation --- 5, NDCQ --- 2 pts, FOC statement --- 3 points

\end{enumerate}



\end{document}