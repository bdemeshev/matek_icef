\subsection{MFE, mock, 31.10.11}
Marks will be deducted for insufficient explanation within your answers. Sections A and B will make up 60\% and 40\% of the exam grade, respectively. Total duration of the exam is 120 min. \\ \\
\textbf{SECTION A}

Answer SIX of the six questions from this section.

\begin{enumerate}

\item Estimate value of $\sin^2(61^{o})\cdot\tan(31^{o})$ by linear approximation using derivatives at $60^{o}$ and $30^{o}$. Convert degrees into radians first.

\begin{enumerate}
\item Degrees to radians. 2 points.
\item Appoximation. 8 points.
\end{enumerate}


\item Let $D$ be the domain of the function $f(x,y)=\ln(x)+\sqrt{y-x}$. Find $D$, the set $D^{o}$ of internal points of $D$, the set $\partial D$ of boundary points of $D$.

\begin{enumerate}
\item The set $D$. 4 points.
\item Internal points. 3 points.
\item Boundary points. 3 points.
\end{enumerate}


\item Consider the function $f(x,y)=x^2+y^3-xy+3y$ at the point $(2;1)$. Find all the directions in which the  growth rate of the function constitutes $60\%$ of the maximal possible growth rate at that point.

\begin{enumerate}
\item Gradient and its length. 3 points.
\item Equation for direction. 3 points.
\item Solution of the equation. 4 points. One missing solution implies penalty 2 point.
\end{enumerate}



\item Find the Hesse matrix of $f(x,y)=y\ln(x^2+y)$. Clearly state Young's theorem if you use it.

\begin{enumerate}
\item First derivatives. 2 points.
\item Second derivatives without the use of Young's theorem. 8 points .
\item Second derivatives with the use of Young's theorem. 6 points for derivatives. 2 points for the statement of the theorem.
\end{enumerate}



\item Consider the function
\begin{equation}
f(x,y)=
\begin{cases}
	\frac{xy}{\sqrt{x^2+y^2}}, \mbox{if} (x,y)\neq (0,0) \\
	0, \mbox{if} (x,y)=(0,0)
\end{cases}
\end{equation}
Is this function continuous at $(0;0)$?

\begin{enumerate}
\item 10 points. Answer without proof = 0 points.
\end{enumerate}


\item The value of $y$ is determined as a function of $t$ by the equation
\begin{equation}
\int_{0}^{t}f(x,y)\, dx=1
\end{equation}
Find $dy/dt$

\begin{enumerate}
\item Mention of the formula $dy/dt=-\frac{G_t}{G_y}$. 2 points.
\item All the rest. 8 points. Answer: $-f(t,y)/\int_0^tf'_y(x,y)\, dx$.
\end{enumerate}


\end{enumerate}

PLEASE TURN OVER

\textbf{SECTION B}

Answer TWO of the two questions from this section.


\begin{enumerate}
\item Find all critical points of the function $z=z(x,y)$ implicitly defined by the equation
\begin{equation}
x^2+y^2+z^2-xz-yz+x+y+4z+1=0
\end{equation}

\item Let the production function $q=F(K,L)$ be twice continuously differentiable. Marginal rate of technical substitution is defined by the formula
\begin{equation}
MRTS=\frac{\frac{\partial F}{\partial L}}{\frac{\partial F}{\partial K}}
\end{equation}
The derivatives in the denominator and numerator are taken while the same amount of the output $q$ is fixed.
Show that under the conditions $F_L>0$, $F_K>0$, $F_{LL}<0$, $F_{KK}<0$, $F_{LK}\geq 0$, the marginal rate monotonously declines with the growth of the factor $L$. In other words $\frac{\partial MRTS}{\partial L}<0$.
\end{enumerate}

\subsection{MFE, fall semester exam, 29.12.11}

Lecturer: K. Bukin \\
Classteachers: B. Demeshev, A. Kalchenko, S. Slavnov, D. Yesaulov \\

Marks will be deducted fro insufficient explanations within your solutions. Exam lasts for 120 minutes. \\

Section A. Answer all 6 questions from this section (10 marks each) \\

\begin{enumerate}

\item The function $f(x,y)$ is given by $f(x,y)=u^2(x,y)+v^3(x,y)$. The values of $u$ and $v$ and their gradients at the point $(x,y)=(1,1)$ are also known, $u(1,1)=3$, $v(1,1)=-2$, $\grad u=(1,4)$, $\grad v=(-1,1)$. Find $\grad f$ if $u,v \in C^1$.
\item Find the Hesse matrix of the function $f(x,y)=\int_{x-3y}^{2x+y}h(t)\,\dif t$ if $h\in C^1$.

\item Determine the values of $a$ for which the quadratic form $x^2 + 2axy + 2xz + z^2$ is positive definite, negative definite, positive semidefinite, negative semidefinite, and indefinite.

\item Given the system of two equations $x^2+y^2=\frac{1}{2}z^2$ and $x+y+z=2$, find $\od{x}{z}$ and $\od{y}{z}$ in the neighborhood of the point $(1,-1,2)$.

\item Find the maxima and minima of the function $f(x,y,z)=xyz$ subject to the constraints $x+y+z=5$ and $xy+yz+xz=8$, $x>0$, $y>0$, $z>0$.

\item If the function $y$ is given by $y(x)=\frac{1}{2}(e^x+e^{-x})$, check that $\od{x}{y}=\frac{1}{\sqrt{y^2-1}} $.
\end{enumerate}

Section B. Answer both questions from this section (20 marks each) \\

\begin{enumerate}
\addtocounter{enumi}{6}
\item Consider the following maximization problem $f(x,y,a,b)=ax^2-x+by^2-y$, where $a$ and $b$ are real numbers.
\begin{enumerate}
\item Derive $(x^*,y^*)$ --- the point that satisfies the first order conditions.
\item Specify the conditions for $a$ and $b$ to ensure that the point $(x^*,y^*)$ is the maximizer.
\item Use the Hessian to specify conditions for concavity and convexity of $f$ depending on the values of parameters.
\item Solve the comparative statics problem: compute $\pd{y^*}{a}$ and $\pd{y^*}{b}$  as $a$ and $b$ marginally change.
\item Using one of the envelope theorems find the rate of change of the value function $f(x^*,y^*,a,b)$ as the result of the marginal change in $a$ and $b$.
\end{enumerate}

\item Robinson Crusoe splits his time $\bar{L}$ hours a week between labor and leisure. His utility function is represented by $u(c,l)=\alpha \ln c+(1-\alpha)\ln l$, where $c$ is the amount of food and $l$ is leisure in hours and $0<\alpha<1$. The food is produced by him in accordance with the production function $c=\sqrt{\bar{L}-l}$.
\begin{enumerate}
\item Find Robinson’s optimal bundle $(c^*,l^*)$ that provides him with the maximum utility (welfare) possible. Justify your answer by checking second-order conditions or otherwise.
\item Let $\alpha=1/4$, $\bar{L}=168$. Using the envelope theorem, estimate the change in the maximum value of his welfare if his utility function has slightly changed to become $u(c,l)=\frac{1}{5} \ln c+\frac{4}{5}\ln l$.
\end{enumerate}

\end{enumerate}

\subsection{MFE, fall semester sols}

\begin{enumerate}
\item[2]
$f_{xx}=4h'(2x+y)-h'(x-3y)$,
$f_{xy}=2h'(2x+y)+3h'(x-3y)$,
$f_{yy}=h'(2x+y)-9h'(x-3y)$,


\item[7]
\begin{enumerate}
\item $x^*=1/2a$, $y^*=1/2b$
\item SOC for maximum: $a<0$, $b<0$
\item Convex if $a\geq 0$, $b\geq 0$, concave if $a \leq 0$, $b \leq 0$.
\item $\partial x^*/\partial b=0$, $\partial x^*/\partial a=-1/2a^2$,
\item  $\partial f^*/\partial a= x^{*2}=1/4a^2$
\end{enumerate}
\end{enumerate}

\subsection{MFE, fall semester retake, 25.01.12 }

Lecturer: K. Bukin. Classteachers: B. Demeshev, A. Kalchenko, S. Slavnov, D. Yesaulov \\

Section A. Answer all 6 questions from this section (10 marks each)

\begin{enumerate}

\item For the function $f(x,y)=2xy+3$ find the level curves and the equations for their tangents at the points $(1,2)$ and $(2, 2)$.

\item The population of a certain country grows exponentially, $N_t= N_{1990}\cdot \exp (r(t-1990))$. The population was 70 million in 1990 and 80 million in 2000,
what will be the population in 2013?

\item Use the chain rule to find $f'(x)$ and $f''(x)$ for $f(x)=u(a,b,x)$ where $a=\cos(x)$ and $b=x^3$.

\item The system of equations defines $x(z)$ and $y(z)$:
\begin{equation}
\begin{cases}
x^2+zxy+y^2+6z+y^3=10 \\
y^3x^2+3x+2y+z=7 \nonumber
\end{cases}
\end{equation}
Find $x'(z)$ and $y'(z)$ at the point $x=1$ and $y=1$.

\item Consider the function $f(x,y)=x^2+y^3-xy+3y$ at the point $(2;1)$. Find all the directions in which the  growth rate of the function constitutes $60\%$ of the maximal possible growth rate at that point.

\item Consider the objective function $f(x,y)=4kx^3+k^2xy+3ky^4-13x-13y$. The point $(x,y)=(1,1)$ is the maximum of the function.
\begin{enumerate}
\item Find the value of $k$
\item Find the approximate increase of the maximum value if $k$ will change by $\Delta k=0.01$
\end{enumerate}


\end{enumerate}

Section B. Answer both questions from this section (20 marks each)

\begin{enumerate}
\addtocounter{enumi}{6}

\item We wish to build a picnic zone for the travellers along a highway. The picnic zone should be rectangular with an area of 1000 m$^2$ and should have a fence on the three sides not adjacent to the highway. The price of one meter of fence is equal to \$ 20.
\begin{enumerate}
\item Find the dimensions of the picnic area that minimize the fencing costs.
\item Using hessian or otherwise check that you have found the costs-minimizing solution.
\item Using the Envelope theorem estimate the change in the costs if we increase the area of the picnic zone by 1 m$^2$.
\end{enumerate}

\item A monopolistic firm with the cost function $TC(Q)=30+15Q+Q^2$ sells a single product in two separate markets. The demand functions for these markets are given by $Q_1=25-P_1$, $Q_2=29-P_2$.
\begin{enumerate}
\item Find the optimal quantities $Q_1$, $Q_2$ to be supplied to the respective markets in order to maximize the profit. Using hessian or otherwise check the second order condition.
\item Calculate the point elasticity of demand for each of the three markets. Is it true that the optimal price is negatively related to the absolute value of elasticity at the optimal levels of output?
\item Using the Envelope theorem estimate the change in the optimal profit if the demand on the second market changes to $Q_2=29.2-1.1P_2$.
\end{enumerate}


\end{enumerate}




\subsection{MFE, mock exam, 02.04.12}


Time allowed 120 minutes.

Students should answer all of the following eight questions. Calculators are not permitted in the exam. Marks will be deducted for insufficient explanations within your answers.

Section A: 10 points  each question.
\begin{enumerate}
\item Determine whether the function $f(x,y)=\ln(5x+y)-5(x+y)^2$ is convex (concave up), concave (concave down), strictly convex, strictly concave or neither.

%Solution. Find Hesse matrix, $\Delta_1=-25/(5x + y)^2 - 10<0$, $\Delta_2=160/(5x + y)^2>0$.

%\item It is known that the functions $f_1(x)$ and $f_2(x)$ are concave up. Is it possible that the function $h(x)=\max\{f_1(x),f_2(x)\}$ is concave down?

\item Solve the differential equation $y^{(4)}-y=\cos(x)$. The $y^{(4)}$ denotes the forth derivative of $y$.
%\item The implicit function $z(x,y)$ is given by the equation $x-z=f(y-z)$ where $f$ is some unknown differentiable function. Find $\frac{\partial z}{\partial x}+\frac{\partial z}{\partial y}$.
\item Consider the monopolist producing two distinct goods. The cost function is given by $TC(q_1,q_2)=q_1+kq_2$, where the constant $k\in (0;1)$. And the demand functions are given by $q_1(p_1,p_2)=q_2(p_1,p_2)=(p_1p_2)^{-3}$.
\begin{enumerate}
\item Find the optimal production bundle for the monopolist.
\item For which values of $k$ one of the product is priced under marginal costs? %Explain why this can happen intuitively.
\end{enumerate}
\item The density of a standard normal random variable $X$ is given by $f(x)=\frac{1}{\sqrt{2\pi}}\exp(-x^2/2)$. Taking the fact that $\int_{-\infty}^{\infty}f(x)\,dx=1$ for granted calculate $E(X^2)$, $E(X^4)$. %, $E(X^6)$.

Hint: if you don't remember, $E(X^n)=\int_{-\infty}^{+\infty}x^nf(x)\,dx$
\item The point elasticity of demand for a good is given by $\varepsilon=p^2/(p^2+4p+3)$. Find the demand function $q(p)$ given the initial condition $q(1)=1$.

%\item Expand the function $f(x)=\sqrt[3]{1+5x}\cos(x^2)$ as a power series up to $x^4$. State the range for $x$ where your expansion is correct.

\item Find the values $a$ and $b$ such that the function $f$ is homogeneous:
\[f(x,y)=2x^{b-a}y^{b+2}+y^{a+1}x^{-3b}+y^{7b}x^{-2a}\]
For the values of $a$ and $b$ you have found expand the function \[h(x)=\sqrt{1+f(x,x)}\cdot (1-\cos(f(x,x))\] as a power series up to $x^4$. State the range for $x$  where your expansion is correct.

\end{enumerate}


Section B: 20 points each question.
\begin{enumerate}
\item It is known that $x_0=0$, $x_{100}=100k$ where $k\in \mathbb{Z}$ is constant and for any $n\in\{2,3,\ldots 100\}$ the following difference equation is satisfied:
\[ x_n-2x_{n-1}+x_{n-2}=-1 \]
\begin{enumerate}
\item Find the particular solution
\item Find the maximum value of $x_n$ for $n\in\{0,\ldots,100\}$ as a function of $k$.
\end{enumerate}

\item Using the Lagrange multiplier method without reducing the number of variables by substitution find the minimum of the function \[f(x,y,z)=2x^2+4y^2+xy+8z^2+2yz\] subject to $x+y+1.5z\geq 1.2$ and $x+y+z=1$.



\begin{comment}


\item The functions $y(t)$ and $x(t)$ satisfy the following system of equations
\[
\left\{
\begin{array}{l}
x'(t)=x(t)+3y(t) \\
y'(t)=3x(t)+y(t)
\end{array}
\right.
\]
with initial conditions $x(0)=a$ and $y(0)=b$.
\begin{enumerate}
\item Find the particular solution
\item State the conditions on $a$ and $b$ that must be satisfied if it is known that $x(t)$ decreases to a finite limit and $y(t)$ increases to a finite limit. Find these limits.
\end{enumerate}


\item Simplified Solow model. The amount of labor in the economy, $L(t)$, evolves according to the equation $L'(t)=\lambda L(t)$, where $\lambda>0$ is a constant. The amount of capital in the economy, $K(t)$, evolves according to the equation $K'(t)=sY(t)$, where the proportion of output which is invested into the capital, $s$, is a constant and the output $Y(t)$ is given by the Cobb-Duglas function $Y(t)=K^{1-a}(t)L^a (t)$.
\begin{enumerate}
\item Find the function $L(t)$ in terms of $L(0)$ and the parameters of the model.
\item Find the function $K(t)$ in terms of $L(0)$, $K(0)$ and the parameters of the model.
\item Find the limit $\lim_{t\to\infty} K(t)/L(t)$
\end{enumerate}


%Consider the following dynamic supply-demand model. The producer sets the pr



\item Consider the region $R$ described by the inequalities
\begin{equation}
\left\{
\begin{array}{l}
x^2+y^2\leq 4 \\
y\geq x \\
x,y \geq 0
\end{array}
\right.
\end{equation}
For each point $(a,b)$ in the first quadrant use the Lagrange method to find the closest point to $(a,b)$ in the region $R$.

\item Let $Y_t$, $C_t$, $I_t$ denote national income, consumption, and investment in period $t$ respectively. The economy is described by the system
\begin{equation}
\begin{cases}
Y_t=C_t+I_t\\
C_t=c+mY_t\\
Y_{t+1}=Y_{t}+rI_t \\
\end{cases},
\end{equation}
where $c$, $m$ and $r$ are positive constants and $m<1$.
\begin{enumerate}
\item Find the function $Y_t$
\item Find the asymptote of $\ln(Y(t))$ as $t$ tends to infinity.
\end{enumerate}

\item The function $f(x)$ is continuous and differentiable. Solve the following continuous analog of the simple regression problem:
\begin{equation}
\min_{a,b} \int_0^1 (f(x)-(a+bx)^2)\,dx
\end{equation}
\end{comment}


\end{enumerate}

\begin{comment}

\newpage
Mock exam on Mathematics for Economists, April 2nd, 2012.

Time allowed 120 minutes.

Students should answer all of the following eight questions. Calculators are not permitted in the exam. Marks will be deducted for insufficient explanations within your answers.

Section A: 10 points  each question.
\begin{enumerate}
\item Determine whether the function $f(x,y)=\ln(6x+y)-6(x+y)^2$ is convex (concave up), concave (concave down), strictly convex, strictly concave or neither.

%Solution. Find Hesse matrix, $\Delta_1=-25/(5x + y)^2 - 10<0$, $\Delta_2=160/(5x + y)^2>0$.

%\item It is known that the functions $f_1(x)$ and $f_2(x)$ are concave up. Is it possible that the function $h(x)=\max\{f_1(x),f_2(x)\}$ is concave down?

\item Solve the differential equation $y^{(4)}-y=\sin(x)$. The $y^{(4)}$ denotes the forth derivative of $y$.
%\item The implicit function $z(x,y)$ is given by the equation $x-z=f(y-z)$ where $f$ is some unknown differentiable function. Find $\frac{\partial z}{\partial x}+\frac{\partial z}{\partial y}$.
\item Consider the monopolist producing two distinct goods. The cost function is given by $TC(q_1,q_2)=q_1+kq_2$, where the constant $k\in (0;1)$. And the demand functions are given by $q_1(p_1,p_2)=q_2(p_1,p_2)=(p_1p_2)^{-3}$.
\begin{enumerate}
\item Find the optimal production bundle for the monopolist.
\item For which values of $k$ one of the product is priced under marginal costs? %Explain why this can happen intuitively.
\end{enumerate}
\item The density of a standard normal random variable $X$ is given by $f(x)=\frac{1}{\sqrt{2\pi}}\exp(-x^2/2)$. Taking the fact that $\int_{-\infty}^{\infty}f(x)\,dx=1$ for granted calculate $E(X^2)$, $E(X^4)$. %, $E(X^6)$.

Hint: if you don't remember, $E(X^n)=\int_{-\infty}^{+\infty}x^nf(x)\,dx$
\item The point elasticity of demand for a good is given by $\varepsilon=p^2/(p^2+3p+2)$. Find the demand function $q(p)$ given the initial condition $q(1)=1$.

%\item Expand the function $f(x)=\sqrt[3]{1+5x}\cos(x^2)$ as a power series up to $x^4$. State the range for $x$ where your expansion is correct.

\item Find the values $a$ and $b$ such that the function $f$ is homogeneous:
\[f(x,y)=2x^{1+b-a}y^{b+3}+y^{a-2}x^{-3b}+y^{7b}x^{7-2a}\]
For the values of $a$ and $b$ you have found expand the function \[h(x)=\sqrt{1+f(x,x)}\cdot (\cos(f(x,x)-1)\] as a power series up to $x^4$. State the range for $x$  where your expansion is correct.

\end{enumerate}


Section B: 20 points each question.
\begin{enumerate}
\item It is known that $x_0=0$, $x_{100}=100k$ where $k\in \mathbb{Z}$ is constant and for any $n\in\{2,3,\ldots 100\}$ the following difference equation is satisfied:
\[ x_n-2x_{n-1}+x_{n-2}=-1 \]
\begin{enumerate}
\item Find the particular solution
\item Find the maximum value of $x_n$ for $n\in\{0,\ldots,100\}$ as a function of $k$.
\end{enumerate}

\item Using the Lagrange multiplier method without reducing the number of variables by substitution find the minimum of the function \[f(x,y,z)=2x^2+4y^2+xy+8z^2+2yz\] subject to $x+y+1.5z\geq 1.2$ and $x+y+z=1$.

\end{enumerate}


\newpage
Mock exam on Mathematics for Economists, April 2nd, 2012.

Time allowed 120 minutes.

Students should answer all of the following eight questions. Calculators are not permitted in the exam. Marks will be deducted for insufficient explanations within your answers.

Section A: 10 points  each question.
\begin{enumerate}
\item Determine whether the function $f(x,y)=7(x+y)^2-\ln(7x+y)$ is convex (concave up), concave (concave down), strictly convex, strictly concave or neither.

%Solution. Find Hesse matrix, $\Delta_1=-25/(5x + y)^2 - 10<0$, $\Delta_2=160/(5x + y)^2>0$.

%\item It is known that the functions $f_1(x)$ and $f_2(x)$ are concave up. Is it possible that the function $h(x)=\max\{f_1(x),f_2(x)\}$ is concave down?

\item Solve the differential equation $y^{(4)}-y=-\sin(x)$. The $y^{(4)}$ denotes the forth derivative of $y$.
%\item The implicit function $z(x,y)$ is given by the equation $x-z=f(y-z)$ where $f$ is some unknown differentiable function. Find $\frac{\partial z}{\partial x}+\frac{\partial z}{\partial y}$.
\item Consider the monopolist producing two distinct goods. The cost function is given by $TC(q_1,q_2)=q_1+kq_2$, where the constant $k\in (0;1)$. And the demand functions are given by $q_1(p_1,p_2)=q_2(p_1,p_2)=(p_1p_2)^{-3}$.
\begin{enumerate}
\item Find the optimal production bundle for the monopolist.
\item For which values of $k$ one of the product is priced under marginal costs? %Explain why this can happen intuitively.
\end{enumerate}
\item The density of a standard normal random variable $X$ is given by $f(x)=\frac{1}{\sqrt{2\pi}}\exp(-x^2/2)$. Taking the fact that $\int_{-\infty}^{\infty}f(x)\,dx=1$ for granted calculate $E(X^2)$, $E(X^4)$. %, $E(X^6)$.

Hint: if you don't remember, $E(X^n)=\int_{-\infty}^{+\infty}x^nf(x)\,dx$
\item The point elasticity of demand for a good is given by $\varepsilon=p^2/(p^2+5p+4)$. Find the demand function $q(p)$ given the initial condition $q(1)=1$.

%\item Expand the function $f(x)=\sqrt[3]{1+5x}\cos(x^2)$ as a power series up to $x^4$. State the range for $x$ where your expansion is correct.

\item Find the values $a$ and $b$ such that the function $f$ is homogeneous:
\[f(x,y)=5x^{2+b}y^{b-a}+y^{a}x^{1-3b}+y^{7b-a-1}x^{1-a}\]
For the values of $a$ and $b$ you have found expand the function \[h(x)=\sqrt{1+f(x,x)}\cdot (\cos(f(x,x)-1)\] as a power series up to $x^4$. State the range for $x$  where your expansion is correct.

\end{enumerate}


Section B: 20 points each question.
\begin{enumerate}
\item It is known that $x_0=0$, $x_{100}=100k$ where $k\in \mathbb{Z}$ is constant and for any $n\in\{2,3,\ldots 100\}$ the following difference equation is satisfied:
\[ x_n-2x_{n-1}+x_{n-2}=-1 \]
\begin{enumerate}
\item Find the particular solution
\item Find the maximum value of $x_n$ for $n\in\{0,\ldots,100\}$ as a function of $k$.
\end{enumerate}

\item Using the Lagrange multiplier method without reducing the number of variables by substitution find the minimum of the function \[f(x,y,z)=2x^2+4y^2+xy+8z^2+2yz\] subject to $x+y+1.5z\geq 1.2$ and $x+y+z=1$.

\end{enumerate}



\newpage
Mock exam on Mathematics for Economists, April 2nd, 2012.

Time allowed 120 minutes.

Students should answer all of the following eight questions. Calculators are not permitted in the exam. Marks will be deducted for insufficient explanations within your answers.

Section A: 10 points  each question.
\begin{enumerate}
\item Determine whether the function $f(x,y)=8(x+y)^2-\ln(8x+y)$ is convex (concave up), concave (concave down), strictly convex, strictly concave or neither.

%Solution. Find Hesse matrix, $\Delta_1=-25/(5x + y)^2 - 10<0$, $\Delta_2=160/(5x + y)^2>0$.

%\item It is known that the functions $f_1(x)$ and $f_2(x)$ are concave up. Is it possible that the function $h(x)=\max\{f_1(x),f_2(x)\}$ is concave down?

\item Solve the differential equation $y^{(4)}-y=-\cos(x)$. The $y^{(4)}$ denotes the forth derivative of $y$.
%\item The implicit function $z(x,y)$ is given by the equation $x-z=f(y-z)$ where $f$ is some unknown differentiable function. Find $\frac{\partial z}{\partial x}+\frac{\partial z}{\partial y}$.
\item Consider the monopolist producing two distinct goods. The cost function is given by $TC(q_1,q_2)=q_1+kq_2$, where the constant $k\in (0;1)$. And the demand functions are given by $q_1(p_1,p_2)=q_2(p_1,p_2)=(p_1p_2)^{-3}$.
\begin{enumerate}
\item Find the optimal production bundle for the monopolist.
\item For which values of $k$ one of the product is priced under marginal costs? %Explain why this can happen intuitively.
\end{enumerate}
\item The density of a standard normal random variable $X$ is given by $f(x)=\frac{1}{\sqrt{2\pi}}\exp(-x^2/2)$. Taking the fact that $\int_{-\infty}^{\infty}f(x)\,dx=1$ for granted calculate $E(X^2)$, $E(X^4)$. %, $E(X^6)$.

Hint: if you don't remember, $E(X^n)=\int_{-\infty}^{+\infty}x^nf(x)\,dx$
\item The point elasticity of demand for a good is given by $\varepsilon=p^2/(p^2+6p+5)$. Find the demand function $q(p)$ given the initial condition $q(1)=1$.

%\item Expand the function $f(x)=\sqrt[3]{1+5x}\cos(x^2)$ as a power series up to $x^4$. State the range for $x$ where your expansion is correct.

\item Find the values $a$ and $b$ such that the function $f$ is homogeneous:
\[f(x,y)=x^{b-2a}y^{b+2+a}+3y^{a+3}x^{-3b-2}-y^{7b-a}x^{-a}\]
For the values of $a$ and $b$ you have found expand the function \[h(x)=\sqrt{1+f(x,x)}\cdot (1-\cos(f(x,x))\] as a power series up to $x^4$. State the range for $x$  where your expansion is correct.

\end{enumerate}


Section B: 20 points each question.
\begin{enumerate}
\item It is known that $x_0=0$, $x_{100}=100k$ where $k\in \mathbb{Z}$ is constant and for any $n\in\{2,3,\ldots 100\}$ the following difference equation is satisfied:
\[ x_n-2x_{n-1}+x_{n-2}=-1 \]
\begin{enumerate}
\item Find the particular solution
\item Find the maximum value of $x_n$ for $n\in\{0,\ldots,100\}$ as a function of $k$.
\end{enumerate}

\item Using the Lagrange multiplier method without reducing the number of variables by substitution find the minimum of the function \[f(x,y,z)=2x^2+4y^2+xy+8z^2+2yz\] subject to $x+y+1.5z\geq 1.2$ and $x+y+z=1$.

\end{enumerate}

\end{comment}



\subsection{MOR, 22.05.12 }

\textbf{Section A}. Solve \textbf{two} of the following \textbf{two} problems

\vspace{12pt}

\begin{enumerate}
\item Use Lagrange multipliers method to solve optimization problem
\[xy\to \max\]
subject to $x\geq 0$, $0\leq y\leq 3$, $x+2y\leq 8$, $y\geq \frac{x^2}{16}+1$.
\item Solve the linear program depending on the parameter $\beta$,
\[2x_1+4x_2+5x_3-x_4\to\min\]
subject to $x_1+x_3-x_4\geq 0$, $-x_1+x_2+\frac{1}{2}x_3+\beta x_4\geq 1$, $x_i\geq 0$.\\
For what values of $\beta$ the minimum of the objective function equals 3?

\end{enumerate}

\vspace{12pt}

\textbf{Section B}. Solve \textbf{two} of the following \textbf{three} problems

\vspace{12pt}

\begin{enumerate}[resume]
\item Find the general solution of the differential equation $y''+2y'+y=xe^{-x}+\cos(x)$
\item Solve the initial-value problem for the system of difference equations


$ \left\{ \begin{array}{l}
x_{t+1}=x_t+y_t \\
y_{t+1}=3x_t-y_t-5
\end{array} \right.$, where $x_0=y_0=0$.

\item In the model of interacting inflation and unemployment based on the Phillips relation,  both unemployment rate $U$ and expected inflation $\pi$ are the solutions of the system $\dot{\pi}=\frac{3}{4}(p-\pi)$, $\dot{U}=-\frac{1}{2}(m-p)$, where $m$ is exogenously defined positive rate of nominal money growth and $p$ is the posteriori observed inflation satisfying equation $p=\frac{1}{6}-3U+\pi$. Find the steady-state solutions for the inflation both expected and observed as well as the unemployment rate in terms of $m$. Explore the dynamic stability of solutions. What is the natural rate of unemployment?

\end{enumerate}

\vspace{12pt}

\textbf{Section C}. Solve \textbf{two} of the following \textbf{three} problems

\vspace{12pt}

\begin{enumerate}[resume]
\item Find all pure and mixed Nash equilibria in the following bimatrix game:


\begin{tabular}{c|ccc}
 & D & E & F \\
\hline
A & 3;4 & 1;3 & 1;0  \\
B & 2;7 & 3;6 & 0;3  \\
C & 0;2 & 2;1 & 5;6  \\
\end{tabular}
\item Two players play a version of Rock-Paper-Scissor game. Paper beats Rock, Rock beats Scissors, Scissors beats Paper. The two players simultaneously make their choice. The first player can choose any object. The second player can choose Rock or Paper. The winner receives 1 rouble from the loser. In case of a draw the wealth of a player does not change.
\begin{enumerate}
\item Construct the payoff matrix of the game.
\item Find all pure and mixed Nash equilibria
\end{enumerate}
\item Two players are trying to bribe the judge. The possible amount of bribe is any real number between 0 and 1 million roubles. The player who gives the biggest bribe is announced as the winner of the affair by the judge. The winner receives 1 million roubles. The loser gets nothing. Obviously bribes are not returned by the judge. In the case of equal bribes each player gets nothing.
\begin{enumerate}
\item Are there any pure Nash equilibria in this game?
\item Find at least one mixed Nash equilibrium.
\end{enumerate}

Short tips:
\begin{enumerate}
\item No
\item If the second player choses his move according continuous distribution function $F$ then the expected payoff of the first player for the bribe $b$ is equal to
\[
1\cdot P(b_2 \leq b) - b = 1\cdot F(b)-b
\]

If a rational player uses mixed strategies he is indifferend between the corresponding pure strategies. That means that $F(b)-b=const$ for pure strategies that are mixed. For pure strategies that are mixed the density function $f(b)=F'(b)=1$. Do I know such a random variable? Yes, I know! A uniform on $[0;1]$.

\end{enumerate}

\end{enumerate}


\subsection{MFE, retake exam, 19.09.2012}

Time allowed 120 minutes.

Students should answer all of the following eight questions. Calculators are not permitted in the exam. Marks will be deducted for insufficient explanations within your answers.

Section A: 10 points  each question.
\begin{enumerate}
%\item Determine whether the function $f(x,y)=\ln(5x+y)-5(x+y)^2$ is convex (concave up), concave (concave down), strictly convex, strictly concave or neither.

%Solution. Find Hesse matrix, $\Delta_1=-25/(5x + y)^2 - 10<0$, $\Delta_2=160/(5x + y)^2>0$.

\item It is known that the functions $f_1(x)$ and $f_2(x)$ are concave up. Is it possible that the function $h(x)=\max\{f_1(x),f_2(x)\}$ is concave down?

%\item Solve the differential equation $y^{(4)}-y=\cos(x)$. The $y^{(4)}$ denotes the forth derivative of $y$.
\item The implicit function $z(x,y)$ is given by the equation $x-z=f(y-z)$ where $f$ is some unknown differentiable function. Find $\frac{\partial z}{\partial x}+\frac{\partial z}{\partial y}$.
\item Consider the monopolist producing two distinct goods. The cost function is given by $TC(q_1,q_2)=q_1+kq_2$, where the constant $k\in (0;1)$. And the demand functions are given by $q_1(p_1,p_2)=q_2(p_1,p_2)=(p_1p_2)^{-3}$.
\begin{enumerate}
\item Find the optimal production bundle for the monopolist.
\item For which values of $k$ one of the product is priced under marginal costs? %Explain why this can happen intuitively.
\end{enumerate}
\item The density of an exponential random variable $X$ is given by $f(x)=\exp(-x)$ for $x>0$. Calculate $E(X)$, $E(X^2)$.

Hint: if you don't remember, $E(X^n)=\int_{-\infty}^{+\infty}x^nf(x)\,dx$
\item The point elasticity of demand for a good is given by $\varepsilon=p^2/(p^2+4p+3)$. Find the demand function $q(p)$ given the initial condition $q(1)=1$.

%\item Expand the function $f(x)=\sqrt[3]{1+5x}\cos(x^2)$ as a power series up to $x^4$. State the range for $x$ where your expansion is correct.

\item Find the values $a$ and $b$ such that the function $f$ is homogeneous:
\[f(x,y)=2x^{b-a}y^{b+2}+y^{a+1}x^{-3b}+y^{7b}x^{-2a}\]
For the values of $a$ and $b$ you have found expand the function \[h(x)=\sqrt{1+f(x,x)}\cdot (1-\cos(f(x,x))\] as a power series up to $x^4$. State the range for $x$  where your expansion is correct.

\end{enumerate}


Section B: 20 points each question.
\begin{enumerate}

\item Use Lagrange multipliers method to solve optimization problem
\[x+\ln y\to \max\]
subject to
$x\geq 0$, $x+y\leq 4$, $x+2y\leq 6$.

\item Let $Y_t$, $C_t$, $I_t$ denote national income, consumption, and investment in period $t$ respectively. The economy is described by the system
\begin{equation}
\left\{
\begin{array}{l}
Y_t=C_t+I_t\\
C_t=c+mY_t\\
Y_{t+1}=Y_{t}+rI_t
\end{array}
\right.
\end{equation},
where $c$, $m$ and $r$ are positive constants.
\begin{enumerate}
\item Find the function $Y_t$
\item Find the asymptote of $\ln(Y(t))$ as $t$ tends to infinity.
\end{enumerate}


\end{enumerate}



\begin{comment}

\item It is known that $x_0=0$, $x_{100}=100k$ where $k\in \mathbb{Z}$ is constant and for any $n\in\{2,3,\ldots 100\}$ the following difference equation is satisfied:
\[ x_n-2x_{n-1}+x_{n-2}=-1 \]
\begin{enumerate}
\item Find the particular solution
\item Find the maximum value of $x_n$ for $n\in\{0,\ldots,100\}$ as a function of $k$.
\end{enumerate}

\item Using the Lagrange multiplier method without reducing the number of variables by substitution find the minimum of the function \[f(x,y,z)=2x^2+4y^2+xy+8z^2+2yz\] subject to $x+y+1.5z\geq 1.2$ and $x+y+z=1$.


\item The functions $y(t)$ and $x(t)$ satisfy the following system of equations
\[
\left\{
\begin{array}{l}
x'(t)=x(t)+3y(t) \\
y'(t)=3x(t)+y(t)
\end{array}
\right.
\]
with initial conditions $x(0)=a$ and $y(0)=b$.
\begin{enumerate}
\item Find the particular solution
\item State the conditions on $a$ and $b$ that must be satisfied if it is known that $x(t)$ decreases to a finite limit and $y(t)$ increases to a finite limit. Find these limits.
\end{enumerate}


\item Simplified Solow model. The amount of labor in the economy, $L(t)$, evolves according to the equation $L'(t)=\lambda L(t)$, where $\lambda>0$ is a constant. The amount of capital in the economy, $K(t)$, evolves according to the equation $K'(t)=sY(t)$, where the proportion of output which is invested into the capital, $s$, is a constant and the output $Y(t)$ is given by the Cobb-Duglas function $Y(t)=K^{1-a}(t)L^a (t)$.
\begin{enumerate}
\item Find the function $L(t)$ in terms of $L(0)$ and the parameters of the model.
\item Find the function $K(t)$ in terms of $L(0)$, $K(0)$ and the parameters of the model.
\item Find the limit $\lim_{t\to\infty} K(t)/L(t)$
\end{enumerate}


%Consider the following dynamic supply-demand model. The producer sets the pr



\item Use Lagrange multipliers method to solve optimization problem
\[x+\ln y\to \max\]
subject to
$x\geq 0$, $x+y\leq 4$, $x+2y\leq 6$.


\item Let $Y_t$, $C_t$, $I_t$ denote national income, consumption, and investment in period $t$ respectively. The economy is described by the system
\begin{equation}
\left\{
\begin{array}{l}
Y_t=C_t+I_t\\
C_t=c+mY_t\\
Y_{t+1}=Y_{t}+rI_t
\end{array}
\right.
\end{equation},
where $c$, $m$ and $r$ are positive constants.
\begin{enumerate}
\item Find the function $Y_t$
\item Find the asymptote of $\ln(Y(t))$ as $t$ tends to infinity.
\end{enumerate}

\item The function $f(x)$ is continuous and differentiable. Solve the following continuous analog of the simple regression problem:
\begin{equation}
\min_{a,b} \int_0^1 (f(x)-(a+bx)^2)\,dx
\end{equation}
\end{comment}


\subsection{MOR, retake exam, 11.09.12}


\textbf{Section A}. Solve \textbf{two} of the following \textbf{two} problems

\vspace{12pt}

\begin{enumerate}
\item Use Lagrange multipliers method to solve optimization problem
\[x+\ln y\to \max\]
subject to
$x\geq 0$, $x+y\leq 4$, $x+2y\leq 6$.
\item Solve the linear program depending on the parameter $\beta$,
\[2x_1+4x_2+5x_3+x_4\to\min\]
subject to $x_1+x_3+x_4\geq 0$, $-x_1+x_2+\frac{1}{2}x_3-\beta x_4\geq 1$, $x_i\geq 0$.\\
For what values of $\beta$ the minimum of the objective function equals 3?

\end{enumerate}

\vspace{12pt}

\textbf{Section B}. Solve \textbf{two} of the following \textbf{three} problems

\vspace{12pt}

\begin{enumerate}[resume]
\item Find the general solution of the differential equation $y''+4y'+4y=xe^{-3x}+\cos(x)$
\item Consider the system of difference equations
$$
\left\{ \begin{array}{l}
x_{t+1}=2x_t-4y_t \\
y_{t+1}=x_t-3y_t+3
\end{array} \right.
$$
\begin{enumerate}
\item Solve the system
\item Find the equilibrium solution and check whether it's stable
\end{enumerate}

\item A policymaker desires to double in 10 periods of time the value of GDP $y_t$  produced in period $t$. Evolution of GDP over time is given by equation $4y_{t+2}-4y_{t+1}+y_t=2^t+t^2$. Is doubling of GDP feasible? If the answer is positive, is it possible to find the period $t$ when the value of $y_t$  will first exceed $2y_0$, where $y_0$ is the initial GDP?



\end{enumerate}

\vspace{12pt}

\textbf{Section C}. Solve \textbf{two} of the following \textbf{three} problems

\vspace{12pt}

\begin{enumerate}[resume]
\item Find all pure and mixed Nash equilibria in the following bimatrix game:


\begin{tabular}{c|ccc}
 & D & E & F \\
\hline
A & 5;5 & 2;4 & 2;1  \\
B & 3;8 & 4;7 & 1;4  \\
C & 1;3 & 3;2 & 6;7  \\
\end{tabular}
\item A man has two sons. When he dies, the value of his estate after tax is \$1000. In his will it states that the sons must specify the sum of money $s_i$ that they are willing to accept. If $s_1+2s_2\leq 1000$, then each gets the sum he asked for and the rest goes the cats’ shelter. If $s_1+2s_2> 1000$, then neither of them gets any money and the entire sum goes to the cats’ shelter. Assume that the sons only care about the money they will inherit and they ask for the whole dollars. Find the pure strategies Nash equlibria of this game.

\item Two players are trying to bribe the judge. The possible amount of bribe is any real number between 0 and 1 million roubles. The probability that the player will win is proportional to the amount of the bribe. In the case of zero bribes the probability is equal $1/2$ for each player. The winner receives 1 million roubles. The loser gets nothing. Obviously bribes are not returned by the judge.
\begin{enumerate}
\item Are there any pure Nash equilibria in this game?
\item Find at least one mixed Nash equilibrium.
\end{enumerate}

\end{enumerate}


