\documentclass[12pt]{article} % размер шрифта

\usepackage{tikz} % картинки в tikz
\usepackage{microtype} % свешивание пунктуации

\usepackage{array} % для столбцов фиксированной ширины

\usepackage{url} % для вставки ссылок \url{...}

\usepackage{indentfirst} % отступ в первом параграфе

\usepackage{sectsty} % для центрирования названий частей
\allsectionsfont{\centering} % приказываем центрировать все sections

\usepackage{amsthm} % теоремы и доказательства

\theoremstyle{definition} % прямой шрифт в условии теорем
\newtheorem{theorem}{Теорема}[section]


\usepackage{amsmath, amssymb} % куча стандартных математических плюшек

\usepackage[top=2cm, left=1.5cm, right=1.5cm, bottom=2cm]{geometry} % размер текста на странице

\usepackage{lastpage} % чтобы узнать номер последней страницы

\usepackage{enumitem} % дополнительные плюшки для списков
%  например \begin{enumerate}[resume] позволяет продолжить нумерацию в новом списке
\usepackage{caption} % подписи к картинкам без плавающего окружения figure


\usepackage{fancyhdr} % весёлые колонтитулы
\pagestyle{fancy}
\lhead{Math for economists}
\chead{Section A}
\rhead{2020-10-24, ICEF}
\lfoot{Variant $\mu$}
\cfoot{Good luck!}
\rfoot{}
\renewcommand{\headrulewidth}{0.4pt}
\renewcommand{\footrulewidth}{0.4pt}



\usepackage{todonotes} % для вставки в документ заметок о том, что осталось сделать
% \todo{Здесь надо коэффициенты исправить}
% \missingfigure{Здесь будет картина Последний день Помпеи}
% команда \listoftodos — печатает все поставленные \todo'шки

\usepackage{booktabs} % красивые таблицы
% заповеди из документации:
% 1. Не используйте вертикальные линии
% 2. Не используйте двойные линии
% 3. Единицы измерения помещайте в шапку таблицы
% 4. Не сокращайте .1 вместо 0.1
% 5. Повторяющееся значение повторяйте, а не говорите "то же"

\usepackage{fontspec} % поддержка разных шрифтов
\usepackage{polyglossia} % поддержка разных языков

\setmainlanguage{english}
\setotherlanguages{russian}

\setmainfont{Linux Libertine O} % выбираем шрифт
% если Linux Libertine не установлен, то
% можно также попробовать Helvetica, Arial, Cambria и т.Д.

% чтобы использовать шрифт Linux Libertine на личном компе,
% его надо предварительно скачать по ссылке
% http://www.linuxlibertine.org/index.php?id=91&L=1

% на сервисах типа sharelatex.com этот шрифт есть :)

\newfontfamily{\cyrillicfonttt}{Linux Libertine O}
% пояснение зачем нужно шаманство с \newfontfamily
% http://tex.stackexchange.com/questions/91507/

\AddEnumerateCounter{\asbuk}{\russian@alph}{щ} % для списков с русскими буквами
%\setlist[enumerate, 2]{label=\asbuk\cdot),ref=\asbuk\cdot} % списки уровня 2 будут буквами а) б) ...

%% эконометрические и вероятностные сокращения
\DeclareMathOperator{\Cov}{Cov}
\DeclareMathOperator{\Corr}{Corr}
\DeclareMathOperator{\Var}{Var}
\DeclareMathOperator{\E}{E}
\DeclareMathOperator{\grad}{grad}
\def \hb{\hat{\beta}}
\def \hs{\hat{\sigma}}
\def \htheta{\hat{\theta}}
\def \s{\sigma}
\def \hy{\hat{y}}
\def \hY{\hat{Y}}
\def \v1{\vec{1}}
\def \e{\varepsilon}
\def \he{\hat{\e}}
\def \z{z}
\def \hVar{\widehat{\Var}}
\def \hCorr{\widehat{\Corr}}
\def \hCov{\widehat{\Cov}}
\def \cN{\mathcal{N}}
\def \RR{\mathbb{R}}


\def \putyourname{\fbox{
    \begin{minipage}{42em}
      Name, group no:\vspace*{3ex}\par
      \noindent\dotfill\vspace{2mm}
    \end{minipage}
  }
}



\begin{document}

\putyourname
\begin{enumerate}
  \item (10 points) Consider the function $f(x, y) = x^3 +3a y^3 + 2axy$ where $a$ is a parameter. 
  Using the total differential find the approximate value of $f(1.98, 0.99)$.

\newpage
\putyourname
  \item (10 points) Consider the system
  \[
  \begin{cases}
  3x^3 + u(y) + u(z) = 5 \\
  x + x^3 + 2u(y^3x) = 4 \\
  \end{cases},
  \]
  where $u(x)$ is a differentiable function with $u(1)=1$. 


  \begin{enumerate}
    \item Clearly state conditions sufficient to guarantee that the system defines the functions $z(y)$ and $x(y)$ at a point $(1, 1, 1)$. 
    \item Find $z'(y)$ provided the conditions are met.
 \end{enumerate}

\newpage
\putyourname
 \item (10 points) Consider the function $f (x, y) = xy^3$  and the point $A = (−1, −2)$.
 
 Find the direction of the maximal rate of change of the function and this maximal rate.


\newpage
\putyourname
\item (10 points) For $x>0$, $y>0$ find the limit 
\[
u(x,y) = \lim_{t\to 0} \left(x^{\frac{t-1}{t}} + y^{\frac{t-1}{t}}  \right)^{\frac{t}{t-1}}
\]

\newpage
\putyourname
\item (10 points) Provide an explicit example of a non-convergent sequence $(x_n)$ in $\RR^2$ 
such that the sequences $y_n = \| x_n \|$ and $z_n = \| x_n  + 2x_{n+1}\|$ are convergent.

\newpage
\putyourname
\item (10 points) Let's consider the sets $A_n = \{ x\in \RR \mid x^2 n = 1\}$.

Describe the set $A = \cup_{n=1}^{\infty} A_n$: is it closed, open, bounded, compact?



\end{enumerate}



\newpage
\chead{Section B}
\putyourname

\begin{enumerate}[resume]
\item 
A curve represented by the equation $(x^2 + y^2)^2 = x^2 - y^2$ is called lemniscate. 

\begin{enumerate}
  \item (5 points) While solving for $y=y(x)$ implicit function defined by this equation 
  is it possible to use IFT in the neighborhood of the point $(0,0)$?
  \item (5 points) Show that in the first quadrant $\{(x,y) \mid x>0, y>0 \}$ 
  such implicit function $y=y(x)$ exists. Justify your reasoning. 
  \item (10 points) Find $y'(0)$ if it exists. 
\end{enumerate}

Hint: for c) it is convenient  to change Cartesian coordinates to polar coordinates 
following the formulas $x=r\cos \phi$ and $y=r\sin \phi$.


\newpage
\putyourname

\item Let $u$ be a composite function $u=g(x^2 + y^2)$, where $g(t) \in C^2$ for $t>0$. 

\begin{enumerate}
  \item (5 points) Is the formula 
  \[
du = g'(x^2 + y^2)(2xdx + 2ydy)
  \]
  for the total differential valid? Provide a clear argument. 
\item  (5 points) For higher order differentials we would like to continue in the same fashion:
\[
d^2u = g''(x^2 + y^2)(2xdx + 2ydy)^2
\]  
Does this method work? Justify your answer. 
\item (10 points) Let $g(t)=\sqrt{t}$. 
Prove that for the function $u(x,y)=\sqrt{x^2 + y^2}$ the second-order differential is non-negative.
\end{enumerate}




\end{enumerate}






%%%% вар!!!

\newpage
\chead{Section A}
\lfoot{Variant $\rho$}

\putyourname
\begin{enumerate}
  \item (10 points) Consider the function $f(x, y) = x^3 +4a y^3 + 2axy$ where $a$ is a parameter. 
  Using the total differential find the approximate value of $f(1.98, 0.99)$.

\newpage
\putyourname
  \item (10 points) Consider the system
  \[
  \begin{cases}
  3x^3 + u(y) + u(z) = 5 \\
  x + x^3 + 3u(y^3x) = 5 \\
  \end{cases},
  \]
  where $u(x)$ is a differentiable function with $u(1)=1$. 


  \begin{enumerate}
    \item Clearly state conditions sufficient to guarantee that the system defines the functions $z(y)$ and $x(y)$ at a point $(1, 1, 1)$. 
    \item Find $z'(y)$ provided the conditions are met.
 \end{enumerate}

\newpage
\putyourname
 \item (10 points) Consider the function $f (x, y) = xy^3$  and the point $A = (−1, −3)$.
 
 Find the direction of the maximal rate of change of the function and this maximal rate.


\newpage
\putyourname
\item (10 points) For $x>0$, $y>0$ find the limit 
\[
u(x,y) = \lim_{t\to 0} \left(x^{\frac{t-1}{t}} + y^{\frac{t-1}{t}}  \right)^{\frac{t}{t-1}}
\]

\newpage
\putyourname
\item (10 points) Provide an explicit example of a non-convergent sequence $(x_n)$ in $\RR^2$ 
such that the sequences $y_n = \| x_n \|$ and $z_n = \| x_n  + 3x_{n+1}\|$ are convergent.

\newpage
\putyourname
\item (10 points) Let's consider the sets $A_n = \{ x\in \RR \mid x^2 n = 2\}$.

Describe the set $A = \cup_{n=1}^{\infty} A_n$: is it closed, open, bounded, compact?



\end{enumerate}



\newpage
\chead{Section B}
\putyourname

\begin{enumerate}[resume]
\item 
A curve represented by the equation $(x^2 + y^2)^2 = x^2 - y^2$ is called lemniscate. 

\begin{enumerate}
  \item (5 points) While solving for $y=y(x)$ implicit function defined by this equation 
  is it possible to use IFT in the neighborhood of the point $(0,0)$?
  \item (5 points) Show that in the first quadrant $\{(x,y) \mid x>0, y>0 \}$ 
  such implicit function $y=y(x)$ exists. Justify your reasoning. 
  \item (10 points) Find $y'(0)$ if it exists. 
\end{enumerate}

Hint: for c) it is convenient  to change Cartesian coordinates to polar coordinates 
following the formulas $x=r\cos \phi$ and $y=r\sin \phi$.


\newpage
\putyourname

\item Let $u$ be a composite function $u=g(x^2 + y^2)$, where $g(t) \in C^2$ for $t>0$. 

\begin{enumerate}
  \item (5 points) Is the formula 
  \[
du = g'(x^2 + y^2)(2xdx + 2ydy)
  \]
  for the total differential valid? Provide a clear argument. 
\item  (5 points) For higher order differentials we would like to continue in the same fashion:
\[
d^2u = g''(x^2 + y^2)(2xdx + 2ydy)^2
\]  
Does this method work? Justify your answer. 
\item (10 points) Let $g(t)=\sqrt{t}$. 
Prove that for the function $u(x,y)=\sqrt{x^2 + y^2}$ the second-order differential is non-negative.
\end{enumerate}




\end{enumerate}
  




%%%% вар!!!

\newpage
\chead{Section A}
\lfoot{Variant $\xi$}

\putyourname
\begin{enumerate}
  \item (10 points) Consider the function $f(x, y) = x^3 +5a y^3 + 2axy$ where $a$ is a parameter. 
  Using the total differential find the approximate value of $f(1.98, 0.99)$.

\newpage
\putyourname
  \item (10 points) Consider the system
  \[
  \begin{cases}
  3x^3 + u(y) + u(z) = 5 \\
  x + x^3 + 4u(y^3x) = 6 \\
  \end{cases},
  \]
  where $u(x)$ is a differentiable function with $u(1)=1$. 


  \begin{enumerate}
    \item Clearly state conditions sufficient to guarantee that the system defines the functions $z(y)$ and $x(y)$ at a point $(1, 1, 1)$. 
    \item Find $z'(y)$ provided the conditions are met.
 \end{enumerate}

\newpage
\putyourname
 \item (10 points) Consider the function $f (x, y) = xy^3$  and the point $A = (−1, −4)$.
 
 Find the direction of the maximal rate of change of the function and this maximal rate.


\newpage
\putyourname
\item (10 points) For $x>0$, $y>0$ find the limit 
\[
u(x,y) = \lim_{t\to 0} \left(x^{\frac{t-1}{t}} + y^{\frac{t-1}{t}}  \right)^{\frac{t}{t-1}}
\]

\newpage
\putyourname
\item (10 points) Provide an explicit example of a non-convergent sequence $(x_n)$ in $\RR^2$ 
such that the sequences $y_n = \| x_n \|$ and $z_n = \| x_n  + 4x_{n+1}\|$ are convergent.

\newpage
\putyourname
\item (10 points) Let's consider the sets $A_n = \{ x\in \RR \mid x^2 n = 3\}$.

Describe the set $A = \cup_{n=1}^{\infty} A_n$: is it closed, open, bounded, compact?



\end{enumerate}



\newpage
\chead{Section B}
\putyourname

\begin{enumerate}[resume]
\item 
A curve represented by the equation $(x^2 + y^2)^2 = x^2 - y^2$ is called lemniscate. 

\begin{enumerate}
  \item (5 points) While solving for $y=y(x)$ implicit function defined by this equation 
  is it possible to use IFT in the neighborhood of the point $(0,0)$?
  \item (5 points) Show that in the first quadrant $\{(x,y) \mid x>0, y>0 \}$ 
  such implicit function $y=y(x)$ exists. Justify your reasoning. 
  \item (10 points) Find $y'(0)$ if it exists. 
\end{enumerate}

Hint: for c) it is convenient  to change Cartesian coordinates to polar coordinates 
following the formulas $x=r\cos \phi$ and $y=r\sin \phi$.


\newpage
\putyourname

\item Let $u$ be a composite function $u=g(x^2 + y^2)$, where $g(t) \in C^2$ for $t>0$. 

\begin{enumerate}
  \item (5 points) Is the formula 
  \[
du = g'(x^2 + y^2)(2xdx + 2ydy)
  \]
  for the total differential valid? Provide a clear argument. 
\item  (5 points) For higher order differentials we would like to continue in the same fashion:
\[
d^2u = g''(x^2 + y^2)(2xdx + 2ydy)^2
\]  
Does this method work? Justify your answer. 
\item (10 points) Let $g(t)=\sqrt{t}$. 
Prove that for the function $u(x,y)=\sqrt{x^2 + y^2}$ the second-order differential is non-negative.
\end{enumerate}




\end{enumerate}
  




%%%% вар!!!

\newpage
\chead{Section A}
\lfoot{Variant $\omega$}

\putyourname
\begin{enumerate}
  \item (10 points) Consider the function $f(x, y) = x^3 +6a y^3 + 2axy$ where $a$ is a parameter. 
  Using the total differential find the approximate value of $f(1.98, 0.99)$.

\newpage
\putyourname
  \item (10 points) Consider the system
  \[
  \begin{cases}
  3x^3 + u(y) + u(z) = 5 \\
  x + x^3 + 5u(y^3x) = 7 \\
  \end{cases},
  \]
  where $u(x)$ is a differentiable function with $u(1)=1$. 


  \begin{enumerate}
    \item Clearly state conditions sufficient to guarantee that the system defines the functions $z(y)$ and $x(y)$ at a point $(1, 1, 1)$. 
    \item Find $z'(y)$ provided the conditions are met.
 \end{enumerate}

\newpage
\putyourname
 \item (10 points) Consider the function $f (x, y) = xy^3$  and the point $A = (−1, −5)$.
 
 Find the direction of the maximal rate of change of the function and this maximal rate.


\newpage
\putyourname
\item (10 points) For $x>0$, $y>0$ find the limit 
\[
u(x,y) = \lim_{t\to 0} \left(x^{\frac{t-1}{t}} + y^{\frac{t-1}{t}}  \right)^{\frac{t}{t-1}}
\]

\newpage
\putyourname
\item (10 points) Provide an explicit example of a non-convergent sequence $(x_n)$ in $\RR^2$ 
such that the sequences $y_n = \| x_n \|$ and $z_n = \| x_n  + 2x_{n+1}\|$ are convergent.

\newpage
\putyourname
\item (10 points) Let's consider the sets $A_n = \{ x\in \RR \mid x^2 n = 4\}$.

Describe the set $A = \cup_{n=1}^{\infty} A_n$: is it closed, open, bounded, compact?



\end{enumerate}



\newpage
\chead{Section B}
\putyourname

\begin{enumerate}[resume]
\item 
A curve represented by the equation $(x^2 + y^2)^2 = x^2 - y^2$ is called lemniscate. 

\begin{enumerate}
  \item (5 points) While solving for $y=y(x)$ implicit function defined by this equation 
  is it possible to use IFT in the neighborhood of the point $(0,0)$?
  \item (5 points) Show that in the first quadrant $\{(x,y) \mid x>0, y>0 \}$ 
  such implicit function $y=y(x)$ exists. Justify your reasoning. 
  \item (10 points) Find $y'(0)$ if it exists. 
\end{enumerate}

Hint: for c) it is convenient  to change Cartesian coordinates to polar coordinates 
following the formulas $x=r\cos \phi$ and $y=r\sin \phi$.


\newpage
\putyourname

\item Let $u$ be a composite function $u=g(x^2 + y^2)$, where $g(t) \in C^2$ for $t>0$. 

\begin{enumerate}
  \item (5 points) Is the formula 
  \[
du = g'(x^2 + y^2)(2xdx + 2ydy)
  \]
  for the total differential valid? Provide a clear argument. 
\item  (5 points) For higher order differentials we would like to continue in the same fashion:
\[
d^2u = g''(x^2 + y^2)(2xdx + 2ydy)^2
\]  
Does this method work? Justify your answer. 
\item (10 points) Let $g(t)=\sqrt{t}$. 
Prove that for the function $u(x,y)=\sqrt{x^2 + y^2}$ the second-order differential is non-negative.
\end{enumerate}




\end{enumerate}



\end{document}
