\documentclass{article}
\usepackage{amsmath, amsfonts, amsthm}
\usepackage[cp1251]{inputenc}
\usepackage[russian]{babel}
\usepackage{lscape}
\usepackage{amssymb}
\voffset -1in
\hoffset -12mm
\pagestyle{empty}
\topmargin=1cm

\oddsidemargin=5mm
\textwidth = 17cm
\textheight=25cm
\usepackage[cp1251]{inputenc}
\usepackage[russian]{babel}
\begin{document}

\fontsize{14}{21}
\selectfont
\centerline{\textbf{Assignment 12 (Due on the week December 9 -- 14)}}
\fontsize{12}{18}
\selectfont
\begin{enumerate}
\item Use the Lagrange multiplier method to write the first-order conditions for the maximum of the function $f(x,y)=\sqrt{x}+\sqrt{y}$, subject to $ax+y=1$, where $a$ is a real parameter. For what values of $a$ solution exists? Check sufficiency condition.

\item Find the points of relative optimum and classify them using second-order conditions:\\ $u=x^2+y+2z\rightarrow extr$, s.t. $x^3yz^2=w$, where $w$ is a real parameter.

\item Find the points of relative optimum and classify them using second-order conditions:\\ $u=(x+z)y\rightarrow extr$, s.t. $x^2+y^2=2$, $y+z=2$, where all the variables are positive.
    
\item A firm with the smooth production function $Q(x,y)$ wants to find the least-cost input combination for a production of a specified level output $Q_0$ representing, say, a customer's special order. Show that at the point of optimal input combination, the input-price-marginal-product ratio must be the same for each input.
\item Show that the function $z=(1+e^y)\cos x-ye^y$ has an infinite number of points of maximum and no point of minimum.
\end{enumerate}
\end{document}