\subsection{MFE, mock, ??.11.08}

Calculators are not allowed. \\

Candidates should attempt:\\
- all questions from Part A \\
- two of three questions from Part B. \\


Part A. \\

Problem 1. $[10]$ \\
Find and sketch on the plane the sets:\\
a)	$([1;3)\times[-1;4))\cap ((2;5]\times(2;5])$\\
b)	$([1;3)\times[-1;4))\cup ((2;5]\times(2;5])$\\
c)	$([1;3)\times[-1;4))\backslash ((2;5]\times(2;5])$\\

Problem 2. $[10]$ \\
Calculate the directional derivative of the function $f(x,y)=2x^3+2y^2$ at the point $A(1;2)$ in the following directions:\\
a)	$\vec{l}=(1;3)$ \\
b)	$\vec{l}$ which is orthogonal to the curve given by the equation $x^2+y^2=5$\\
c)	Direction of the fastest growth of $f(x,y)$\\

Problem 3. $[10]$ \\
Consider the following system of equation:\\
$\left\{\begin{array}{l}
xyzw+2x^3y^3z^3+4w^3=7\\
x+y+z^3+w^3+w^2z^2=5\\
\end{array}\right.$\\
a)	Does this system define functions $z(x,y)$ and $w(x,y)$ at a point $x=1$, $y=1$, $z=1$, $w=1$?\\
b)	If it's possible find $\frac{\partial z}{\partial x}$ and $\frac{\partial w}{\partial y}$ at that point\\

Problem 4. $[10]$ \\
Find the Hesse matrix of the function $f(x,y)=(2+cos(x))^{sin(y)+5}$\\
Comment: clearly state Young theorem if you use it \\

Problem 5. $[10]$ \\
Find the total differential for the function $f(x,y)=x^2y^2+xy^2+2x+4y$\\
Using the total differential find approximately $f(1.001,1.999)$ \\

Problem 6. $[10]$ \\
Using the chain rule find all the first derivatives of  $p(a,b)$ where $p(a,b)=f(x(a,b),y(a,b))$, $x(a,b)=a^2+ab$ and $y(a,b)=b^3+ab$ \\


Part B. \\

Problem 7. $[20]$ \\
Find the critical points of the function $f(x,y,z)=x^3-y^3+9xy-z^2e^{-z^2}$\\
Classify them using the Hesse matrix. \\

Problem 8. $[20]$ \\
The individual lives for two periods.  He has a utility function $U(c_{1},c_{2})=u(c_{1})+\beta u(c_{2})$. In each period his wage is equal to $w$. His budget constraint requires that his period 1 consumption be his wage $w$ minus any savings, $c_{1}=w-s$. The government taxes savings at the rate $q$ per dollar saved. So his second period consumption will be $c_{2}= w + (1+ r)s(1-q)$.  The individual takes $q$, $r$, $w$ as given. The individual seeks to maximise his utility $U$.\\
a)	Find the first order condition for the optimal savings amount\\
b)	Using the implicit function theorem find $\frac{\partial s}{\partial q}$\\
c)	Find $\frac{\partial s}{\partial q}$ explicitly if $u(c)=ln(c)$\\

Problem 9. $[20]$ \\
A firm sells its output into a perfectly competitive market and faces a fixed price $p$. It hires labor in a competitive labor market at a wage $w$, and rents capital in a competitive capital market at rental rate $r$. The production function is $f(L, K)$. The firm seeks to maximize its profits. Both factors are used in positive amounts in the optimal point. \\
a)	Express the profit $\pi$ as a function of $L$ and $K$\\
b)	Find necessary first order conditions for a profit-maximizing point $(L^{*},K^{*})$\\
c)	Find the second order conditions sufficient for maximum (two inequalities)\\
d)	Using the implicit function theorem find $\frac{\partial K^{*}}{\partial w}$.  Is it positive or negative if you additionally know that $\frac{\partial^{2}\pi}{\partial L\partial K}>0$?\\



\subsection{MFE, fall semester exam, 21.01.2009}

Lecturer: K.A. Bukin, Classteachers: B.B. Demeshev, I.O. Kachkovski


SECTION A. Answer all FOUR questions from this section (60 marks in total)

\begin{enumerate}
\item Find the gradient of $f(x,y) =\frac{x^2y}{\sqrt{x^2+y^2}}$
at the point $M(1,3)$. Compute the derivative of $f$ at $M$ in direction of the vector $\{-1, 1\}$.

\item Calculate all partial derivatives of the first and second order of $u$ with respect to $x$ and $y$ if $u=f (\xi, \eta)$ and $\xi=x+xy$, $\eta=x/y$.

\item Consider the system of equations
\[
\left\{
\begin{array}{l}
x_1^2+x_1 y_1+x_2 y_2  = 3 \\
x_2^2+x_1 y_2 -x_2 y_1 =1
\end{array}
\right.
\]
Are implicit functions $y = y(x)$ and $x = x(y)$ defined around the point $M(1, 1, 1, 1)$?
Here $y$ denotes the vector with components $y_1$ and $y_2$ and the same notation holds for $x$.
If your answer is affirmative, find $\partial y_1/\partial x_1$ and $\partial x_2/\partial y_2$ at the point $M$.

\item Find all stationary points of $f(x,y) = x^4 + y^4-4xy + 1$. Classify them as local minimum, maximum or saddle point.
\end{enumerate}


SECTION B
Answer TWO of the three questions from this section (20 marks each)
\begin{enumerate}[resume]
\item Consider a problem of identifying Pareto–optimal allocations in the two goods
economy with the two agents having identical utility functions $U_1(x_1 , y_1)= x_1 y_1$ and
$U_2(x_2, y_2) = x_2 y_2$, where $x_i$ and $y_i$ denote respectively the amounts of goods $x$ and $y$ that are allocated to agent $i = 1, 2$. Given a weight $\alpha \in (0, 1)$, the solution of that problem can be found by maximizing the weighted utility function $\alpha U_1 + (1 - \alpha)U_2$ with respect to all the variables subject to the resource constraints $x_1 + x_2 = w_1$, $y_1 + y_2 = w_2$. In this economy $w_1$, $w_2$ are positive endowments.
\begin{enumerate}
\item Using the Lagrangean derive the necessary conditions for extremum and solve the
system of equations.
\item Without applying the second–order conditions explain why the solution
$(x_1^* , x_2^* , y_1^* , y_2^* )$
is the maximum of the problem stated above.
\item Calculate the value of the weighted utility function at the Pareto-optimal allocation and compare it with the case when all the endowment in $x$ and $y$ is given to a single
agent. How would you explain a difficulty you may find here?
\end{enumerate}

\item A firm’s inventory $I(t)$ is depleted at a constant rate per unit time, i.e. $I(t) = x-\delta t$, where $x$ is an amount of good reordered by the firm whenever the level of inventory is
zero. The order is fulfilled immediately. The annual requirement for the commodity is
$200$ units and the firm orders the commodity $n$ times a year where $200 = nx$. The firm
incurs two types of inventory costs: a holding cost and an ordering cost. Since the average
stock of inventory is $x/2$, the holding cost equals $C_h x/2$, the cost of placing one order is
$C_o$, and with $n$ orders a year the annual ordering cost equals $C_o n$.
\begin{enumerate}
\item Minimize the cost of inventory $C = C_h x/2 + C_o n$ by choice of $x$ and $n$ subject to
the constraint $nx = 200$ by the Lagrange multiplier method.
\item Use the envelope theorem to approximate the change in the minimal cost if the
requirement for the commodity rises to $204$ units.
\end{enumerate}


\item Use the Lagrange multiplier method to prove that the triangle with maximum area
that has a given perimeter $2p$ is equilateral. Be sure to justify that the extreme you find
is indeed maximum.


\textit{Hint 1:} Use Heron’s formula for the area:
\[
A=\sqrt{p(p - x)(p - y)(p - z)}
\]
where $x$, $y$ and $z$ are the lengths of the sides.


\textit{Hint 2:} You may find that maximizing $A^2$ instead of just $A$ will require somewhat more
pleasant calculations while giving the same answer.
\end{enumerate}
